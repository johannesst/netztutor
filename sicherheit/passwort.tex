
\subsection{Pa"sw"orter}
\label{passwort}

% Das Terminalprogramm \glossar telnet geh"ort zum Lieferumfang von s"amtlichen
% \glossar UNIX--Clones, Windows95/98 und Windows NT. F"ur andere Betriebssysteme
% muss es gegebenenfalls noch nachinstalliert werden. Prinzipiell kann man auch
% jedes andere Terminalprogramm verwenden, das eine DEC--vt100 Terminalemulation
% beherrscht.

\subsubsection*{Wie sicher ist das Pa"swort? Wie sicher sollte es sein?}

Die Sicherheit des Pa"sworts sollte nicht untersch"atzt werden. Erstens sind
manche Pa"sw"orter leichter zu knacken als andere, und zweitens kann mit einem
geklauten oder geknackten Pa"swort weitaus mehr Unfug getrieben werden als nur
fremde \glossar E-Mails zu lesen.

Wer eine Vorstellung davon bekommen m"ochte, wozu ein unbefugter Zugang
mi"sbraucht werden kann, dem sei das Buch "`Kuckucksei"' von Clifford Stoll
empfohlen. Es handelt sich dabei "ubrigens nicht um ein Informatik--Fachbuch,
sondern um einen tatsachenbasierten Roman. Man erf"ahrt auf sehr humorvolle und
auch f"ur Laien verst"andliche Weise, wie Computersysteme angegriffen und
mi"sbraucht werden k"onnen. 

Das Pa"swort sollte also m"oglichst schwer zu knacken sein. Gegen systematische
alphabetische Angriffe (Ausprobieren aller Kombinationen aller verf"ugbaren
Zeichen) ist im Grunde kein Kraut gewachsen. Allerdings ist der Zeitaufwand
derart hoch, da"s diese Angriffsart kaum praktikabel ist. Die
Wahrscheinlichkeit, das richtige Pa"swort zu erwischen, ist geringer als die,
da"s das Pa"swort inzwischen ge"andert wurde. H"aufiger sind dagegen die
sogenannten W"orterbuchattacken, bei denen nur solche Pa"sw"orter ausprobiert
werden, die auch sinnvolle W"orter ergeben, wie sie eben in W"orterb"uchern einer
beliebigen Sprache stehen (deutsch oder englisch zum Beispiel). Auch
Namenslisten (menschliche Vornamen, Firmennamen, etc.) werden oft als
Grundlage benutzt.

\subsubsection*{Und wie w"ahle ich nun mein Pa"swort?}

In der letzten Frage wurde gekl"art, welche Zeichenketten nicht als Pa"sw"orter
benutzt werden sollen (Echte deutsche oder englische W"orter, Eigennamen,
Firmennamen usw). Es gibt mehrere Regeln, die ein Pa"swort gegen"uber den
"ublichen Attacken praktisch immun machen: 

\begin{enumerate}
  \item Gro"s- und Kleinschreibung gemischt verwenden (wird unterschieden)
  \item Zahlen mit einbauen
  \item Sonderzeichen mit einbauen (allerdings keine speziellen deutschen
    Sonderzeichen, siehe oben)
  \item Pa"swort nicht zu kurz w"ahlen (8 Zeichen sind optimal) 
\end{enumerate}

Als n"achstes stellt sich m"oglicherweise die Frage, wie man sich ein solches
sicheres Pa"swort noch merken k"onnen soll. Es gibt verschiedene Eselsbr"ucken
und der Phantasie sind keine Grenzen gesetzt. Eine M"oglichkeit: Man nimmt sich
eine Zeile aus seinem Lieblingsgedicht, -song, oder was auch immer und greift
sich die Anfangsbuchstaben heraus. Aus "`We don't need no Education"' ergibt
sich beispielsweise das Pa"swort "`WdnnE"' --- leicht zu merken und schwer zu
erraten bzw.\  zu knacken.
