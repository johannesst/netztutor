\subsection{Allgemeine Hinweise}
\label{allgemeines_sicher}
Grundsätzlich ist jeder Benutzer im Netzwerk selbst für die Sicherheit seines Rechners verantwortlich. Dazu gehört insbesondere, diesen von Viren und Bots frei zu halten. Daher sollte man stets das Betriebssystem auf einen aktuellen Stand halten und grundsätzlich nur mit aktiver Firewall und einen Virenscanner am Netz sein. Dies gilt für ALLE Betriebssysteme. Zwar ist die Situation unter Linux und Mac OS X nicht so gravierend, wie unter Windows, bis sich das ändert, dürfte aber nur eine Frage der Zeit sein. \\
Sollte ein Rechner von einen Bot und/oder Virus befallen sein, der von
dort dann weitere Rechner angreift, wird dieser im Regelfall vom
Rechenzentrum gesperrt. Freigeschaltet wird er erst wieder, nachdem
man seinen Rechner neu eingerichtet und sechs Euro Strafgebühr bezahlt
hat. Man sollte also im eigenen Interesse seinen Rechner
sauberhalten. Auf den nächsten Seiten gibt es nun
Hintergrundinformationen zu den einzelnen Schädlingsarten, sowie zur
sicheren Wahl eines Passwortes.

%%% Local Variables: 
%%% mode: latex
%%% TeX-master: "../Netzeinfuehrung"
%%% End: 
