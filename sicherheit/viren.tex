
\subsection{Viren, Trojaner, Bugs}
\label{viren}

\begin{quote}
"`The degree to which you take security seriously and invest
in it should be proportional to the value and sensitivity of
your system and its data."' (aus \url{http://www.nwi.net/~pchelp/security/advice.htm})
\end{quote}

Dank Computer-BILD wei"s es jede/r: Im Internet warten Tausende 
von Kriminellen nur darauf, da"s du deinen Computer an das Netz anschlie"st.
Wenn du also vermeiden willst, da"s diese b"osen Jungs und M"adchen sich an 
deinem Home-Banking-Account, deiner Diplomarbeit oder gar deinen 
Quake-Spielst"anden zu schaffen machen, solltest Du einige Sachen beachten. 

Damit die Angreifer (f"alschlicherweise oft auch Hacker genannt) auf deinen
Computer bzw. deine Daten zugreifen k"onnen, muss ein bestimmtes Programm auf 
deinem Computer laufen. Dabei gibt es mehrere M"oglichkeiten:

\subsubsection{Computerviren}
Ein Computervirus ist eine Sequenz von Programmcode, welche 
in anderen, n"utzlichen Code eingef"ugt ist und mit diesem ausgef"uhrt wird. 
Dabei versucht sich der Code in andere Programme zu kopieren, d.h. dieses zu 
infizieren. Viren befinden sich also immer in einem "`Wirtsprogramm"'. 
Bootsektor--Viren setzen sich in den Bootsektor von Disketten oder Festplatten
fest. Dieser Programmcode wird nach dem Booten des Rechners direkt gestartet.
Damit ein Virus von einer geliehenen Diskette nicht gleich gestartet
wird, stelle im BIOS die Boot-Sequenz zuerst auf Laufwerk C.
Weiterhin kann sich ein Virus in einem Anwendungsprogramm (z.B. MS-Word) oder
in einem Spiel verstecken. Allerdings k"onnen auch andere Dateien ausf"uhrbaren
Code enthalten: MIME--encoded Mail, WWW-Seiten mit JavaScript oder VBScript,
Postscript Dateien, Word-Dateien,~\dots Gegen Viren hilft ein (besser zwei) gutes
und aktuelles Virenkiller-Programm und seine h"aufige Anwendung.

\subsubsection{Trojanische Pferde (Trojan Horses)} 
Analog zur griechischen Mythologie wird
hier ein K"oder ausgelegt. Neben diesem erh"alt man jedoch unerw"unschten
Programmcode. Trojaner k"onnen sich prinzipiell in den gleichen Dateitypen
wie Viren aufhalten. Starte daher niemals (!) Programme unbekannter Herkunft 
auf deinem Computer. Gleiches gilt nat"urlich auch f"ur den Crack von
\url{www.evilhacker.org} (gibt's die wirklich?) oder Programme die du auf einem
Rechner hier im lokalen Netz gefunden hast. Wenn Dich Dein Mailprogramm fragt,
ob es ein Programm ausf"uhren soll, und du bist dir nicht absolut sicher, warum,
antworte mit nein! Eines der ausgereiftesten Trojaner ist z.Z. SubSeven: mehr
Infos dazu unter \url{http://home.t-online.de/home/TschiTschi/subseven.htm}.

\subsubsection{Bugs}
Fehlerhafte Programme die du bereits installiert hast, k"onnen offene
Angriffspunkte enthalten. Softwarehersteller 
bringen h"aufig Updates zu ihren Betriebssystemen oder Anwendungsprogrammen heraus. 
Halte dich auf den Laufendem und aktualisiere auf neuere Versionen, insbesondere, 
wenn Sicherheitsl"ucken bekannt geworden sind.  

Sicherheit im Computerbereich ist ein Katz--und--Maus--Spiel. Informiere dich
deshalb laufend "uber die aktuellen Entwicklungen. Sicherheit ist kein Produkt
was man fertig installieren kann, sondern ein st"andiger Proze"s. Sei skeptisch,
wenn Dir jemand sagt, mit seinem Produkt bist Du sicher. Auch unsere Firewall
bietet nur begrenzten Schutz gegen Angriffe von "`au"sen"', gegen Angriffe aus 
dem Wohnheim ist sie v"ollig nutzlos.  

Infos zum Thema Sicherheit findet man z.B. unter \url{http://www.insecure.org/} im Internet.

Zum Schluss noch eine Bitte: Wenn ihr eine Email erhaltet, in der vor einer
anderen mit dem Betreff: XYZ gewarnt wird und man euch auffordert, diese an 
m"oglichst viele Leute weiter zu senden, dann leitet diese NICHT weiter! 
Solange man Emails nur liest und nicht ausf"uhrt, k"onnen sie niemals Schaden
anrichten!

\subauthor{Richard Karsch}
