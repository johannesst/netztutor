\documentclass[12pt,titlepage,twoside]{scrartcl}

\usepackage[ngerman]{babel}
\usepackage[]{graphicx}
\usepackage[T1]{fontenc}
\usepackage[utf8]{inputenc}
%\usepackage{times}
\usepackage{fancyhdr}
%\usepackage{amsmath}
%\usepackage{path}
\usepackage{url}
\usepackage{eurosym}
\usepackage{calc}

\tolerance=2000
\emergencystretch=20pt

%\textwidth16cm
%\textheight23cm       %21cm
\topmargin0.0cm      %0.5cm
%\headheight1.0cm
%\headsep0.5cm         %1.2cm
%\footskip0.7cm          %0cm
%\footheight0cm        %0cm
\oddsidemargin0.2cm   %0.5cm
\evensidemargin0.2cm  %0.5cm
\parindent0cm
\parskip1.5ex plus0.5ex minus0.5ex

\pagestyle{fancy}
\lhead[\rmfamily \thepage]{\nouppercase{\slshape \leftmark}}
\rhead[\nouppercase{\slshape \leftmark}]{\rmfamily \thepage}
\chead{}
\lfoot{}
\cfoot{}
\rfoot{}

%\headrulewidth0pt
%\footrulewidth0pt
%\plainheadrulewidth0pt
%\plainfootrulewidth0pt

%\addtocounter{secnumdepth}{2}
%\addtocounter{tocdepth}{2}
\newcommand{\settocdepth}[1]{\addtocontents{toc}{\protect\setcounter{tocdepth}{#1}}}

% Paragraph-Ueberschrift nicht eingebettet
% ========================================
%\makeatletter
%\renewcommand\paragraph{\@startsection{paragraph}{4}{\z@}%
%                                     {-3.25ex\@plus -1ex \@minus -.2ex}%
%                                     {1.5ex \@plus .2ex}%
%                                     {\normalfont\normalsize\bfseries}}
%\renewcommand\subparagraph{\@startsection{subparagraph}{5}{\z@}%
%                                     {-3.25ex\@plus -1ex \@minus -.2ex}%
%                                     {1.5ex \@plus .2ex}%
%                                     {\normalfont\normalsize\bfseries}}
%\makeatother

\newcounter{para_nr}
\newcommand{\Paragraph}[1]{{\large\bf\S{}\sf\textbf{\/\stepcounter{para_nr}\arabic{para_nr} #1}}}
\newcommand{\tm}{$^{TM}$}
\newcommand{\snev}{\emph{SchunterNet~e.V.} }
\newcommand{\glossar}{
\unitlength1.5mm
\begin{picture}(2,2)
%\thicklines
\put(0,0){\vector(2,1){2.5}}
\end{picture}
}
\newcommand{\subauthor}[1]{
\begin{flushright}
\footnotesize \copyright\ by #1
\end{flushright}
}
\newcommand{\centergraphics}[2][]{
\begin{center}
\includegraphics[#1]{#2}
\end{center}
}

\newfont{\bigbold}{cmssbx10 scaled 4299}

\begin{document}

\title{%\includegraphics[width=8cm]{logo.ps}\\[2cm]
Einf"uhrung in die Nutzung des\\[5mm]
\bigbold SchunterNet\\[5mm]
\normalsize 3. "uberarbeitete Auflage -- Oktober 2009 \vspace{3.3cm}}

\date{\small \sf SchunterNet e.V. \quad Braunschweig\\ \small \sf Bienroder Weg 54 \hfill 38108 Braunschweig \hfill Telefon/Fax: (0531) 2\,35\,14\,59 \hfill www.schunternet.de}
\author{}

\uppertitleback{\small \glossar\ \dots\ Erl"auterung im Glossar

MS-DOS, Windows, Windows 95/98, Windows NT sind eingetragene Warenzeichen der
    Microsoft Corporation. Alle Warennamen werden ohne Gew"ahrleistung der
    freien Verwendbarkeit benutzt und sind m"oglicherweise eingetragene
    Warenzeichen.}
\lowertitleback{\small \textbf{Impressum:}

Herausgeber: \parbox[t]{8cm}{SchunterNet e.V.\\ Bienroder Weg 54\\ 38108
  Braunschweig}

Autoren: \parbox[t]{\textwidth-15.6mm}{Ronny Heidenreich, Uwe Gre"shake unter
  Verwendung der "`Wohnheimnetz FAQ"' des Studentenwohnheimes Sch"utzenweg 42
  in Oldenburg (\url{http://www.Fortytwo.Uni-Oldenburg.de/~waterman/FAQ/})
  sowie der "`Technischen Hinweise f"ur Informationsanbieter im World Wide
  Web"' des Rechenzentrums der Technischen Universit"at Braunschweig
  (\url{http://www.tu-bs.de/www/techn_hinweise/})\\
  Dank an Andreas L"ubbecke f"ur zahlreiche Korrekturen und Erg"anzungen sowie
  an Oliver B"ohnke f"ur die Informationen zum Macintosh.\\
  Aktualisiert und neugeschrieben von Alexander Riemer und Johannes Starosta
  }

Satz: \LaTeX}

\maketitle

\thispagestyle{empty}

\tableofcontents

\clearpage

\section{Der Verein SchunterNet e.V.}

Um die Realisierung der Vernetzung des Studentenwohnheims An der Schunter
voranzutreiben, wurde am 28. August 1997 der Verein \snev gegr"undet. In diesem
Rahmen wurde in Zusammenarbeit mit dem Studentenwerk, dem Rechenzentrum und
nicht zuletzt verschiedenen IT-Unternehmen ein Konzept f"ur ein Wohnheimnetz
erarbeitet. Im November 1997 erhielt der Verein vom Studentenwerk ein Darlehen
und den Auftrag, das Netz zu realisieren.

Am 1. Juli 1998 waren die Kabel im Haus 1 verlegt und die aktiven Komponenten
angeschlossen, so da"s der Probebetrieb beginnen konnte. Seit September 1998
ist auch die Funkanbindung an das Rechenzentrum funktionsf"ahig und das
\emph{SchunterNet} offiziell in Betrieb. Im Sommersemester 1999 wurden
schlie"slich auch die "ubrigen Geb"aude an das Netzwerk angeschlossen.

Der \snev betreibt das Netz, stellt die \glossar Systemadministration und
verwaltet die Teilnehmerdaten.

W"ochentlich wird eine \glossar{Sprechstunde} im Clubhaus angeboten. 
Hier k"onnen Probleme beim Netzbetrieb besprochen oder der Netzantrag
abgeholt und eingereicht werden. 
Mindestens zweimal j"ahrlich findet zudem eine Mitgliederversammlung
statt.

Da all dies von den Bewohnern neben ihrem Studium erledigt wird und die zu
investierende Zeit daher begrenzt ist, sind weitere helfende H"ande nat"urlich
immer willkommen. Der \snev ist dabei nicht als ein au"senstehender Anbieter
eines Netzzugangs zu betrachten, sondern vielmehr als eine rechtlich
notwendige Organisationsform aller Netzteilnehmer, respektive Teil der
Bewohnerschaft des Wohnheims. Jedem sollte bewusst sein, da"s ohne die aktive
Mitarbeit \emph{jedes} Nutzers dieses Netz nicht entstanden w"are und in
Zukunft auch nicht weiter betrieben werden kann.

\newpage
\section{Anmeldung und Netzanschluss}

\subsection{Nutzervertrag}

Um einen Zugang zum \emph{SchunterNet} zu erhalten, muss zun"achst ein
Antrag gestellt werden. Dieser dient zum einen der Registrierung der
pers"onlichen und technischen Daten der Nutzer bzw.\  deren Computer im
\snev und im Gauss--IT--Zentrum der TU zum Zwecke der \glossar Administration und
zum anderen als Nutzervertrag der Anerkennung der Teilnahmeregeln durch den
Nutzer. Gleichzeitig erfolgt die Aufnahme in den \snev als passives Mitglied.
% Schön wäre es... Anpassung an den Stand Umbau Haus3?
\bf Den Antrag erhält man beim Einzug vom Hausmeister oder in der
wöchentlichen Sprechstunde des SchunterNet e.V.% den Prospektst"andern entnommen werden, welche sich in jedem
%Haus neben den Briefk"asten befinden, und 
Eine
PDF-Version ist auch unter
\url{http://www.schunternet.de/SchunterNet/Verein/Dokumente/} zu
finden. \\
Er ist ist vollst"andig ausgef"ullt und unterschrieben in den
Briefkasten des Vereins beim
 Clubhaus einzuwerfen. % Foto vom Briefkasten einfügen?

\rm Es folgen einige Hinweise zum Ausf"ullen des Antrags und zu den Hintergr"unden.
Bei auftretenden Fragen, werden wir gerne 


\subsection{Netzwerkkarte und MAC--Adresse}
In jedem Zimmer befindet sich eine Anschlussdose. Sie befindet sich
meistens in Nähe der Steckdosen und sieht wie folgt aus:

\bf{FIX ME: Bild einer typischen Netzwerkdose einfügen :D}

\rm Zum Anschluss ist ein Netzwerkadapter (Ethernet--Karte) mit RJ45--Buch\-se und
ein Anschlusskabel (Twisted Pair, RJ45, \textbf{nicht} crossed over) der
ben"otigten L"ange erforderlich. Ein entsprechender Adapter ist in den
gängigen Computern heutzutage fest verbaut. Das dazugehörige Kabel
kann, wenn nicht schon vorhanden, in der Sprechstunde des Vereins zum
Selbstkostenpreis erworben werden.
%Die
%Bandbreite der Netzkarte bzw.\  der Netzanbindung kann 10~MBit/s (10BaseT) oder 100~MBit/s (100BaseT) betragen. Macs haben seit 1999 10/100BaseT Ethernet eingebaut, das gilt auch
%f"ur iMacs/iBooks/PowerBooks.

Jede Ethernet--Karte bzw.\ --Chip hat eine weltweit eindeutige Hardware--(oder
\glossar MAC--)Adresse, die fest in der Karte eingestellt ist. Bei "alteren
Ethernet--Karten ist sie auf einen der Chips aufgeklebt, bei neueren
erf"ahrt man sie entweder aus der mitgelieferten Dokumentation, oder
durch ein Service--Programm.%, welches auf der mitgelieferten Diskette
%zu finden ist. Dazu wird der Rechner im MS--DOS--Mode gestartet (echtes DOS ist
%wichtig, also keine MS--DOS--Eingabeaufforderung oder DOS--Box unter Windows!)
%und das Programm \verb#A:\setup# aufgerufen. Unter "`View
%Configuration"' steht die Hardwareadresse (6 Byte, hexadezimal, jeweils durch
%Doppelpunkt getrennt).
Die Adresse kann z.B. so aussehen: \texttt{00:12:6b:9f:20:cc}
Je nach Betriebssystem gibt es mehrere Möglichkeiten, die MAC-Adresse
herauszufinden.

\subsubsection*{Windows 95/98/NT}
\textbf{FIXME: Kann das raus? Oder gibts da noch welche?}\\
Benutzer von Windows 95/98 haben  eine weitere recht einfache
M"oglichkeit. Das Programm winipcfg.exe im Windows--Verzeichnis (meist
\verb#C:\WINDOWS#) zeigt Informationen "uber den Netzwerkanschluss
an, unter anderem auch die Hardwareadresse. Wichtig ist hierbei, da"s im
entsprechenden Auswahlfeld die Ethernet-Netzwerkkarte angew"ahlt ist (die
PPP-Adresse 44:45:53:54:00:00 ist \textbf{nicht} die gesuchte)
Unter Windows NT findet sich das dazu n"otige Programm im Startmen"u unter
Programme/Verwaltung (Allgemein)/Windows NT-Diagnose. Hier steht die
MAC-Adresse auf der Seite "`Netzwerk"' und da "uber den Button "`Transporte"'.

\subsubsection*{Windows XP}
Unter Windows XP wählt man den Punkt Startmenü. Anschließend wählt man
den Menüpunkt ,,Ausführen'' und tippt dadrin \texttt{cmd} ein. In die sich
nun öffnende Eingabeaufforderung gibt man den Befehl \texttt{ipconfig /all}
ein. %Illustrierende Bilder
\textbf{FIXME: Bilder sagen mehr als tausend worte}

\subsubsection*{Windows Vista/ 7}

\textbf{FIXME: Bilder sagen mehr als tausend worte}


\subsubsection*{Mac OS}

% In einer Sprechstunde sah ich, wie eine Benutzerin die MAC über den
% Apple System Profiler herausfand, würde das gerne mal anhand des
% Minis nachvollziehen und fotografisch aufbereiten :)
Unter MacOS X kann die MAC-Adresse mit den ,,Apple System Profiler''
ermittelt werden: %screenshot mini
Unter älteren Systemen (Mac OS 9.22 und dessen Vorgänger) kann 
die MAC--Adresse ermittelt werden, indem man das
Kontrollfeld "`TCP/IP"' aufruft und dessen INFO-Button anw"ahlt. Die hier
angegebene "`Hardware Adresse"' ist die MAC--Adresse. Alternativ kann die
MAC--Adresse auch mit dem "`Apple System Profiler"' im Apple--Men"u ermittelt
werden. 

\textbf{FIXME: Bilder sagen mehr als tausend worte}

\subsubsection*{Unix/Linux}
Zunächst öffnet man ein Terminal. Dadrin gibt man dann den Befehl 
. Unter
\glossar UNIX/Linux öffnet man zunächst ein Terminal. Dort gibt man
dann den Befehl \texttt{ifconfig} ein. Unter Umständen muss der
absolute Pfad \texttt{ifconfig} benutzt werden. Die Ausgabe könnte so
aussehen: %insert screenshot mit englischer und deutscher Form!

\textbf{FIXME: Bilder sagen mehr als tausend worte}


\subsection{IP--Adressen und Namen}
\label{namen}

Damit ein Rechner im Internet identifiziert werden kann, ben"otigt er
eine weltweit eindeutige IP--Adresse, die euch von der \glossar
Netzadministration zugeteilt wird. F"ur das SchunterNet stehen Adressen in den
Bereichen 134.169.168.xxx und 134.169.169.xxx zur Verf"ugung. 

Da diese Zahlenkolonnen recht unhandlich sind, erh"alt jeder Rechner im
Wohnheim zus"atzlich einen eindeutigen Namen, welcher aus einem Host-- und einem
Domain--Teil besteht. Die Domain unseres Wohnheims ist {\tt
  schunter.etc.tu-bs.de}. Der Hostname kann frei gew"ahlt werden und darf aus
Kleinbuchstaben (keine Umlaute) und Zahlen bestehen, muss aber mit einem
Buchstaben beginnen. Nat"urlich darf er im Wohnheim auch nur
h"ochstens einmal vorkommen. (Bsp.: scotty logan5 \dots).
Falls es euren gew"ahlten Rechnernamen schon gibt, m"usst ihr
einen anderen Namen w"ahlen.

Aus Gr"unden der Handhabbarkeit wird eine Bezeichnung empfohlen, die
"ubersichtlich, reproduzier-- und merkbar ist. "Uberlange Zeichenfolgen, die
keinen Sinn ergeben, sind das zum Beispiel nicht.

Um einen Zugang auf den Benutzerserver (Linux--PC) zu erhalten (z.B.\  um dort
eine \glossar Homepage einzurichten oder die eingegangene \glossar E-Mail
abzuholen), ben"otigt ihr noch einen Loginnamen. Auch dieser kann relativ frei
gew"ahlt werden, darf jedoch nicht mehr als 8 Zeichen enthalten und sollte in
einem gewissen Zusammenhang mit dem zugeh"origen Benutzer stehen (Vorname,
Spitzname etc.) Bei der Wahl des Loginnamens sollte man beachten, da"s dieser
au"serdem den ersten Teil der E-Mail--Adresse darstellen wird, also
\url{loginname@schunter.etc.tu-bs.de}. Daher gilt hier die obige Empfehlung
f"ur die Zeichenfolge des Rechnernamens besonders.

\subsection{Kosten}

Das \emph{SchunterNet} wurde mit einem zinsfreien Kredit des Studentenwerks
finanziert, welcher im Zeitraum von f"unf Jahren in monatlichen Raten
zur"uckzuzahlen war. Diese Kosten wurden auf eine monatliche Nutzungsgeb"uhr
nach der Geb"uhrenordnung [Anhang \ref{gebuehr}] umgelegt. Inzwischen sind die laufenden Kosten 
(Glasfaserleitung zum APM/IT--Zentrum und Systemwartung) in die Nutzungsgeb"uhr enthalten. 

Um den Aufwand f"ur den Kassenwart in Grenzen zu halten, ist die Zahlung durch 
Einzugserm"achtigung der Regelfall, ein Vordruck befindet sich am Netzantrag.
Bitte beachtet, da"s die Banken bei nicht gedecktem Konto oder aus anderen
Gr"unden nicht ausgef"uhrten Lastschriften Bearbeitungsgeb"uhren erheben, die bis
zu \EUR{10,--} betragen k"onnen und euch durch den Verein in voller H"ohe
berechnet werden. In Ausnahmefällen kann die Nutzungsgebühr während der Sprechstunde in bar 
beglichen werden. [Anhang \ref{gebuehr}]

\subsection{Minimale Netzkonfiguration}

Bei fast allen Betriebssystemen m"ussen w"ahrend der Installation ein paar
Angaben gemacht werden, die "uberall gleichartig sind:
\begin{description}
  \item[TCP/IP--Adresse bzw. Host--Adresse] hier muss die sogenannte IP--Adresse
    des eigenen Rechners eingegeben werden, z.B. 134.169.168.255. 

    Jeder Computer im Wohnheim hat eine eigene IP--Adresse. Dies sind vier
    Zahlen von 0 bis 255 die jeweils durch einen Punkt getrennt werden. Die
    ersten beiden Zahlen sind innerhalb des Wohnheims (und innerhalb der TU)
    immer gleich und zwar 134.169. Die dritte Zahl ist im Wohnheim entweder
    168 oder 169. 
%    Zun"achst werden die Adressen aus dem Bereich 134.169.168
%    vergeben, der Bereich 134.169.169 ist f"ur wachsende Teilnehmerzahlen
%    reserviert. 
    Die IP--Adresse identifiziert genau einen Computer, also hat jeder Rechner
    eine andere IP--Adresse. Diese wird euch bei Annahme des Antrags von
    der \glossar Administration mitgeteilt.

    Hier wurde von dem Beispiel 134.169.168.255 ausgegangen. Anstatt der 255
    muss dann die passende/richtige Endung eingesetzt werden. 

  \item[Subnet mask] Hier muss korrekterweise 255.255.254.0 eingegeben
    werden.

  \item[Netzadresse, Broadcast] Wenn danach gefragt wird, muss als Netzadresse
    134.169.168.0 und als Broadcastadresse 134.169.169.255 eingetragen werden.

  \item[DNS/Name Server, Router, Gateway] Hier muss immer die
    IP--Adresse 134.169.168.1 eingegeben werden.
 
  \item[TCP/IP Domain Name] Hier muss \url{schunter.etc.tu-bs.de} eingegeben
    werden.

  \item[Host Name oder Rechnername] Hier muss der im Antrag angegebene
    Rechnername (siehe auch \ref{namen}) eingetragen werden.
 
  \item[DHCP, BOOTP und DDNS] Falls hiernach gefragt wird, dann sollte dies
    nicht ausgew"ahlt bzw.\  mit nein beantwortet werden.
\end{description}

Ansonsten folgen hier noch ein paar Hinweise, die betriebssystemspezifisch
sind:


\subsubsection{Windows95/98}

Damit also auf zur Maus--Klick--Orgie: Es wird davon ausgegangen, da"s das
Fenster "`Systemsteuerung"' ge"offnet ist (zu finden "uber das Men"u \fbox{Start}
unter \fbox{Einstellungen}).

Zun"achst zur Netzwerkkarte und damit zum
schwierigsten Teil: Wann die Karte konfliktfrei l"auft und Windows95 auch
"uberzeugt davon ist, ist nur sehr schwer festzustellen. Das hier sind einige
Anhaltspunkte.

\centergraphics[width=13.5cm]{W95Netzkarte}

Die Karte funktioniert mit Sicherheit noch nicht, wenn
\begin{itemize}
  \item sie unter \fbox{System} $\Rightarrow$ \fbox{Ger"atemanager}
    $\Rightarrow$ \fbox{Netzwerkkarten} gar nicht auftaucht oder
  \item das Netzwerkkartensymbol an besagter Stelle mit einem gelb unterlegten
    Rufzeichen \raisebox{0.4mm}{$\bigcirc$}\hspace{-2.7mm}!\ \ versehen ist.
\end{itemize}

Die Karte funktioniert mit sehr gro"ser Wahrscheinlichkeit, wenn 
\begin{itemize}
  \item sie unter \fbox{System} $\Rightarrow$ \fbox{Ger"atemanager}
    $\Rightarrow$ \fbox{Netzwerkkarten} auftaucht und
  \item nicht mit dem gelben Symbol versehen ist und 
  \item sie unter \fbox{System} $\Rightarrow$ \fbox{Ger"atemanager}
    $\Rightarrow$ \fbox{Computer} $\Rightarrow$ \fbox{Eigenschaften} nicht mit
    dem kleinen runden \raisebox{0.2mm}{$\bigcirc$}\hspace{-2.7mm}i~--Symbol
    (blau auf wei"sem Grund) versehen ist.
\end{itemize}

\centergraphics{W95TCPIP}

Falls alle Bedingungen bis auf die letzte erf"ullt sind, kann die Karte
m"oglicherweise auch funktionieren. In dem Fall hilft nur Ausprobieren. Aus
meinen Erfahrungen kann ich leider keine zielgerichtete Anleitung
zusammenstricken, wie auftretende Konflikte beseitigt werden, aber
funktioniert hat es letztendlich doch immer irgendwie.
Oft ist auch Gl"uck bzw. Willk"ur seitens des Betriebssystemherstellers dabei.
Wenn die Karte einmal installiert ist, ist der Rest ein
Kinderspiel. 

Unter \fbox{Netzwerk} wird durch \fbox{Hinzuf"ugen} $\Rightarrow$
\fbox{Protokoll} $\Rightarrow$ \fbox{Hinzuf"ugen} $\Rightarrow$
\fbox{Microsoft} $\Rightarrow$ \fbox{TCP/IP} $\Rightarrow$ \fbox{OK} das
TCP/IP--Protokoll installiert. "Uber \fbox{TCP/IP} $\Rightarrow$
\fbox{Eigenschaften} werden die oben genannten Einstellungen (IP--Adresse,
\glossar Subnetmask, Gateway, \glossar DNS--Server, Host-- und Domain--Name)
vorgenommen. Alle anderen Felder bleiben zun"achst leer. Das NETBEUI--Protokoll
sollte entfernt werden, ebenso IPX, wenn es nicht aus bestimmten Gr"unden ganz
unbedingt ben"otigt wird. Der "`\glossar Client f"ur Netware--Netzwerke"' ist
ebenfalls f"urs Wohnheim ohne Bedeutung und kann entfernt werden. Wichtig ist
der "`\glossar Client f"ur Microsoft--Netzwerke"', dieser sollte auch unter
"`Prim"are Netzanmeldung"' (oder so "ahnlich) gew"ahlt werden. In der
Registerkarte "`Identifikation"' sollte nochmals der Rechnername und als
Arbeitsgruppe "`\texttt{schunter}"' angegeben werden.

\begin{sloppypar}
Abgeschlossen wird jeweils mit \fbox{OK}. Dazwischen liegen beliebig viele
Reboot--(=~Rech\-nerneustart--)Vorg"ange, je nachdem, wie Windows sich gerade
f"uhlt. Wenn Windows dann zum ersten Mal nach Benutzername und Kennwort fragt,
ist es ratsam, als Benutzernamen den Loginnamen (siehe \ref{namen}) f"ur den
Wohnheimserver zu w"ahlen. Das Pa"swort ist an dieser Stelle egal und kann
getrost auch weggelassen werden.
\end{sloppypar}

Bei der gesamten Installation ist es extrem ratsam, die
Windows95/98--Installations--CD in greifbarer N"ahe zu haben, da von dieser CD
die Netzwerktreiber installiert werden.

\subsubsection{WindowsNT}

Bei der Installation des Betriebssystems ist darauf zu achten, da"s der
ganze Netzwerkkram mitinstalliert wird (ich wei"s auch nicht, wie man das
nachinstalliert, wenn es nicht schon da ist\dots). Zumindest sollte dann der
Ordner "`Netzwerkumgebung"' auf dem Desktop vorhanden sein.

Wenn man dort (mit rechter Maustaste) \fbox{Eigenschaften} ausw"ahlt, erscheint
ein Requester, aus dem der Computer-Name und die Arbeitsgruppe/Dom"ane
hervorgeht. Durch \fbox{"Andern} kommt man in folgende Eingabemaske:

\centergraphics{Identifikation}

Dort ist nun der richtige Rechnername einzutragen. Des weiteren ist der
Knopf f"ur Arbeitsgruppe wichtig (eine Dom"ane ist nach Windows-Nomenklatur
nicht das, was man darunter normalerweise versteht -- und setzt einen
Dom"anencontroller voraus, den es im SchunterNet nicht gibt). Welche
Arbeitsgruppe man eintr"agt ist an sich egal, allerdings bekommt man sp"ater
nur die Rechner sofort zu sehen, die in der selben Arbeitsgruppe sind. Im
SchunterNet sind die meisten Rechner in der Gruppe \url{SCHUNTER} (ein paar
auch in \url{Arbeitsgruppe} :-), zumindest ist der Server \url{Jupiter} dort zu
finden). Das kann man jetzt mit \fbox{OK} best"atigen.

\centergraphics{protokolle}

Als n"achstes ist die Seite "`Protokolle"' anzuw"ahlen (s.o.). Hier sollte
zumindest "`TCP/IP-Protokoll"' aufgef"uhrt sein. Wenn nicht, kann man es "uber
"`Hinzuf"ugen"' dto. Andere Protokolle wie zu Beispiel NetBIOS oder IPX/SPX
sind auch noch "ublich und werden haupts"achlich von Spielen verwendet
(sprich: ganz wichtig, aber nicht f"ur diese Brosch"ure). Sodann w"ahlt man
"`TCP/IP"' aus und geht auf \fbox{Eigenschaften\dots}. Jetzt sollte sich
folgendes Bild zeigen:

\centergraphics{IPAdresse}

Im SchunterNet gibt es keinen DHCP-Server, sprich automatische Zuteilung
von IP-Adressen, also den Radiobutton "`IP-Adresse angeben"' w"ahlen.
Nun tr"agt man nun die vom SchunterNet zugeteilte IP-Adresse, SubNetMask
und Standard Gateway (=Default-Gateway) ein. Unter \fbox{Optionen\dots} finden
sich noch verschiedene Dinge, die man braucht, wenn man selbst einen Gateway
einrichten will, aber das soll hier nicht das Thema sein (genauer gesagt,
ich hab es auch noch nicht probiert) -- also da braucht man nichts extra
eintragen. Weiter oben ist auch immer die Netzwerkkarte genannt --
wiederum nur von Interesse, wenn man mehrere davon haben sollte.

Als n"achstes ist noch der Nameserver einzutragen. Das geschieht auf der
Seite "`DNS"'.

\centergraphics{DNS}

Hier ist nochmals der Host (=Rechnername) erw"ahnt. Die Dom"ane muss noch
eingetragen werden: \url{schunter.etc.tu-bs} Bei "`Suchreihenfolge f"ur DNS"'
ist noch mittels \fbox{Hinzuf"ugen\dots} die IP-Adresse des Nameserver
einzutragen, also im SchunterNet die 134.169.168.1 (ja, ja, das ist der selbe
Rechner wie das Gateway)

Die Einstellungen f"ur die WINS-Adresse auf der folgenden Seite bleiben
leer. Der beim Best"atigen der "Anderungen der Netzwerkeigenschaften
erfolgende Warnhinweis, da"s keine prim"are WINS-Adresse eingegeben wurde,
kann getrost ignoriert werden.

\subauthor{Roland Damm}

\subsubsection{UNIX (Linux)}

Zun"achst muss die Netzwerkkarte vom Betriebssystemkernel erkannt
werden. Angenommen, die eigene IP--Adresse lautet 134.169.168.255, und die
Netzwerkkarte wurde als Interface eth0 erkannt. Dann sollten die drei Zeilen

\begin{verbatim}
root@erde:~ # ifconfig eth0 134.169.168.255 netmask 255.255.254.0 \
              broadcast 134.169.169.255
root@erde:~ # route add 134.169.168.0
root@erde:~ # route add default gw 134.169.168.1
\end{verbatim}

ausreichen, um den Netzwerkzugriff zu erhalten. Praktischerweise werden diese
Zeilen in einem der Boot--Startup--Skripte untergebracht.
Abh"angig von der vorliegenden \glossar UNIX--Version m"ussen noch die Dateien
\texttt{/etc/hosts}, \texttt{/etc/host.conf} und \texttt{/etc/resolv.conf}
angepa"st werden, um fremde Rechner auch "uber Namen und nicht nur "uber
IP--Adressen ansprechen zu k"onnen.

So sollte das zum Beispiel aussehen: 

\begin{verbatim}
# /etc/hosts
127.0.0.1       localhost
134.169.168.255 erde.schunter.etc.tu-bs.de      erde
# end of /etc/hosts

# /etc/host.conf
order hosts bind
multi on
# end of /etc/host.conf

# /etc/resolv.conf
search schunter.etc.tu-bs.de tu-bs.de
nameserver 134.169.168.1
# end of /etc/resolv.conf
\end{verbatim}

Der Vollst"andigkeit halber: Ein Loopback--Device muss ebenfalls konfiguriert
sein. Bei den meisten \glossar UNIX--Systemen sollte dieser Vorgang aber
bereits in der Standardinstallation integriert sein. Ansonsten helfen die
Manual--Pages zu ifconfig(8) und route(8) weiter.

Bei aktuellen Linux--Distributionen vereinfacht sich noch einiges. Kernel mit
den notwendigen eincompilierten Treibern sind meist vorhanden. Falls die
Netzwerkkarte dennoch nicht gefunden wird, kann mit dem Bootparameter ether
bzw.\  mit den jeweiligen Modul--Parametern nachgeholfen werden (siehe Handbuch
der Distribution). Die Interface- und Routingkonfiguration sowie die
\glossar DNS--Einstellungen werden von den Administrations- und
Installationswerkzeugen (z.B. YaST von S.u.S.E) weitestgehend automatisch
vorgenommen. Falls Probleme auftauchen, sollte die umfangreiche Dokumentation
von Linux so ziemlich alle Fragen beantworten. Wenn das zur Distribution
geh"orige Handbuch nicht reicht, sind die Linux--HOWTOs die n"achste
Anlaufstelle. Bei jeder guten Distribution liegen diese Dokumente nach der
Installation irgendwo unter /usr/doc.

% Raus!
% \subsubsection{OS/2}

% Bei der Installation von OS/2 muss die Installation von TCP/IP--Diensten
% ausgew"ahlt werden. Falls OS/2 bereits installiert ist, k"onnen die
% TCP/IP--Dienste auch nachinstalliert werden, indem nacheinander der Ordner
% "`System"', der Ordner "`Systemkonfiguration"' und schlie"slich der Ordner
% "`Installieren/Entfernen"' ge"offnet werden. Im letzteren findet sich das
% Programm "`Netzwerkinstallation anpassen"' (oder so "ahnlich). Mit diesem
% k"onnen die TCP/IP--Dienste nachinstalliert werden. Bei der Installation wird
% der Benutzer dann nach den oben genannten und folgenden Angaben gefragt:

% \begin{description}
%   \item[(Netzwerk-)Adapter] hier findet die Auswahl des %oben genannten
%     Hardwaretreibers f"ur die Netzwerkkarte statt. Falls eine NE2000--kompatible
%     Netzkarte eingebaut werden soll, kann hier der Treiber f"ur
%     Eagle--Technology NE2000plus Ethernet Adapter ausgew"ahlt werden. Allerdings
%     kann dieser Treiber scheinbar keine Plug and Play--Karten automatisch
%     erkennen. Deswegen muss eine solche Karte vor der Installation auf manuell
%     eingestellt werden, siehe dazu Benutzeranleitung der Karte. "Uber
%     Einstellungen gibt es die M"oglichkeit die sogenannte Unterbrechungsebene
%     (IRQ) und Basisadresse zu kontrollieren (wovon unbedingt Gebrauch gemacht
%     werden sollte) oder gar zu ver"andern (durch Auswahl/Anklicken der
%     Einstellung IRQ oder Basisadresse und Anklicken von "Andern). Normalerweise
%     ist f"ur eine Netzwerkkarte auch nur ein Hardwaretreiber erforderlich und
%     kein weiterer.
%   \item[Protokoll] hier sollte einfach TCP/IP ausgew"ahlt
%     werden. Normalerweise ist nur dieses eine \glossar Protokoll erforderlich
%     und kein anderes und weitere Einstellungen sind ebenfalls nicht n"otig.
% \end{description}

% Eigentlich sollte beim Installieren von OS/2 bzw.\  den Hardware- und
% Netzwerktreibern alles automatisch geschehen. Der Benutzer muss bei
% Aufforderung nur die richtigen Daten eingeben.

\subsubsection{Macintosh}

Die Konfiguration des Macintosh erfolgt analog zu der f"ur Windows 95/98
usw. beschriebenen. Die dazu aufzurufenden Kontrollfelder (Apfel--Men"u
$\Rightarrow$ Kontrollfelder) hei"sen:
"`TCP/IP"' und "`Internet"'. Alternativ kann der "`Internet Assistent"' (im
Ordner "`Internet"' auf der Bootplatte) benutzt werden, der Schritt f"ur
Schritt durch die Konfiguration f"uhrt. Die Netzwerkeinstellungen k"onnen
jederzeit ge"andert werden und sind ohne Neustart/Reboot sofort g"ultig.

\subsection{Abmeldung}

Damit die Beendigung der Mitgliedschaft reibungslos vonstatten geht und nicht
zu viele Nutzungsgeb"uhren eingezogen werden, sollte die Abmeldung mindestens
sechs Wochen im voraus, sp"atestens bis zum 25. des vorletzten Monats der
Teilnahme, bei der Netzverwaltung eingereicht werden. Dies kann
formlos (mit Unterschrift), mittels des beim Verein erh"altlichen oder auf dem
Webserver
(\url{http://www.schunter.etc.tu-bs.de/SchunterNet/Verein/Dokumente/Abmeldung.ps})
zu findenden Vordrucks oder durch die auf dem Webserver zu findende
Eingabemaske \url{http://www.schunter.etc.tu-bs.de/SchunterNet/abmeldung.html}
erfolgen. Dort kann dann auch eine neue \glossar E--Mail--Adresse und/oder die
URL einer neuen \glossar Homepage angegeben werden, auf welche dann in den auf
den Austritt folgenden sechs Monaten eingehende Nachrichten oder Anfragen
weitergeleitet werden.
% Eine Abmeldung per \glossar E-Mail ist nicht ratsam, da
%nicht gesichert ist, da"s diese auch vom vorgeblichen Absender stammt.


\section{Netzdienste}

\subsection{Electronic Mail}

Der Loginname des Benutzerservers bildet den ersten Teil der \glossar
E-Mail--Adresse, welche dann folgenderma"sen aussieht:
\url{loginname@schunter.etc.tu-bs.de}
Um die Merkbarkeit zu vereinfachen, wird au"serdem ein Mail--Alias der Form
\url{Vorname.Nachname@schunter.etc.tu-bs.de} eingerichtet. Umlaute werden zu
ae, oe, ue usw. 



Zum Senden und Empfangen von E-Mail im SchunterNet sind z.B. unter Netscape
folgende Einstellungen notwendig:

\fbox{Bearbeiten} $\Rightarrow$ \fbox{Einstellungen} $\Rightarrow$ \fbox{Mail
  \& Diskussionsforen} $\Rightarrow$ \mbox{Mail--Server:} $\Rightarrow$
  \mbox{Server f"ur eingehende Mail:} \url{pop.schunter.etc.tu-bs.de}

Unter \fbox{Bearbeiten\dots} kann der Benutzername und das Pa"swort
gespeichert werden (siehe dazu auch \ref{viren} und \ref{passwort}).

Server f"ur ausgehende Mail: \url{mail.schunter.etc.tu-bs.de}

\subsubsection*{Wie komme ich an meine E-Mail in der Uni?}

Mehrere E-Mail--Adressen bei verschiedenen Service--Providern zu haben, ist
keine Seltenheit mehr. Daraus erw"achst die Frage, wie es m"oglich ist, alle
\glossar E-Mails von einem einzigen Arbeitsplatz aus abzurufen und zu
bearbeiten. Dabei gibt es grunds"atzlich zwei Strategien. 

\begin{enumerate}
  \item S"amtliche \glossar E-Mail wird zun"achst automatisch an eine einzige
    Adresse weitergeleitet und auf dem empfangenden \glossar Server gesammelt
    zwischengelagert. Von dort kann sie auf einmal abgerufen werden. 
  \item Die \glossar E-Mail bleibt auf dem\glossar  Server, der zu der
    jeweiligen Adresse geh"ort. Der Arbeitsplatzrechner muss dann immer von
    mehreren \glossar Servern die \glossar E-Mail abholen.
\end{enumerate} 

Die erste Methode kann dadurch eingeschr"ankt werden, da"s man als normaler
Benutzer nicht auf allen \glossar Serversystemen die M"oglichkeit hat, eine
automatische Weiterleitung der \glossar E-Mail zu veranlassen. Die zweite
M"oglichkeit bereitet Probleme, wenn die Software auf dem Arbeitsplatzrechner
nicht darauf ausgerichtet ist, von mehr als einem \glossar Server \glossar
E-Mail abzurufen. Folgende Beispiele sollen helfen, das richtige Verfahren zu
w"ahlen. Die Mail--Adresse im Wohnheim sei \url{loginname@schunter.etc.tu-bs.de}.

\paragraph*{Forwarding ins Wohnheim (1.Strategie)}

%Angenommen, es besteht ein zweiter \glossar Account auf den \glossar
%UNIX--Systemen im Hochschulnetz (Institute etc.). Dann kann auf den dortigen
%\glossar UNIX--Systemen eine Datei mit Namen "`\path{.forward}"' angelegt
%werden, die die Adresse im Wohnheim enth"alt (z.B. mit dem Kommando: \texttt{\#
%  echo \url{loginname@schunter.etc.tu-bs.de} > .forward}). S"amtliche \glossar E-Mail
%an die Hochschul--Adresse wird dann an das Wohnheim weitergeleitet und kann
%dort zusammen mit der restlichen \glossar E-Mail abgerufen werden. Leider
%funktioniert das seit der Umstellung auf das Filesystem afs im Rechenzentrum
%nicht mehr.

Der an der TU vergebene Mailalias der Form \url{V.Nachname@tu-bs.de} kann
direkt auf den Wohnheimserver umgeleitet werden. Dies kann jeder "uber den
Benutzerdatendienst des IT--Zentrums der TU--Braunschweig
(\url{https://www2.tu-bs.de/it/services/benutzer/bdd}) selbst erledigen.

Zun"achst meldet man sich "uber den Link "`LOGIN"' mit dem RZ--login
("ublicherweise eine y--Nummer) an. Dann f"uhrt ein Klick auf den gelben
Button im Feld "`Mailbox"' zu einem "Anderungsformular f"ur die
Mailadresse. Dort wird z.B. die Mail--Adresse
"`\url{loginname@schunter.etc.tu-bs.de}"' eingegeben und mit \fbox{abschicken}
best"atigt.

\paragraph*{Abruf von mehreren Servern (2. Strategie)}

Hier bleibt die Organisation ganz dem Arbeitsplatzrechner "uberlassen. Die
jeweilige Mail--Software muss so eingestellt werden, da"s von allen
Mail--Servern, bei denen ein Zugang besteht, die \glossar E-Mail abgeholt
wird. Zu jedem Mail--Zugang geh"ort ein POP3--Server mit Benutzerkennung und
Pa"swort. Nicht jedes Mail--Programm bietet diese M"oglichkeit.


% \subsection{WWW und HTTP--Proxy}

% Im \snev wird ein eigener \glossar Server f"ur das \glossar World Wide Web
% betrieben. Dieser dient unter anderem der Information "uber das Wohnheim An der
% Schunter, vorhandene M"oglichkeiten zum Arbeiten und Leben sowie die dort
% stattfindenden Veranstaltungen. Die Website ist unter dem URL 
% \url{http://www.schunternet.de/} zu erreichen. 

% Um den Datenverkehr %"uber die Funkbridge zum RZ 
% gering zu halten und den
% Zugriff auf h"aufig genutzte Informationsquellen im Internet zu beschleunigen,
% sollten alle Zugriffe auf \glossar WWW- und \glossar FTP-Server "uber den
% lokalen \glossar Proxy-Cache erfolgen. Dazu muss dieser im verwendeten Browser
% eingestellt werden, z.B.:

% \begin{description}
%   \item[Mozilla Firefox:] \mbox{~}\\
%     \fbox{Edit oder Extras} $\Rightarrow$ \fbox{Preferences} $\Rightarrow$
%     \fbox{Advanced} $\Rightarrow$ \fbox{Proxies} $\Rightarrow$ \fbox{Manual
%     proxy configuration} $\Rightarrow$ \fbox{View}\\
%     FTP Proxy, Gopher Proxy und HTTP Proxy: \url{proxy.schunter.etc.tu-bs.de}\\
%     Port: \verb#3128#\\
%     No proxy for: \url{schunter.etc.tu-bs.de} \url{schunternet.de}
%       \url{schuntille.de} \url{schunterkino.de}
%   \item[MS Internet Explorer:] \mbox{~}\\
%     \fbox{Ansicht} $\Rightarrow$ \fbox{Optionen} $\Rightarrow$
%     \fbox{Verbindung} $\Rightarrow$ \fbox{Verbindung "uber einen Proxy-Server}
%     $\Rightarrow$ \fbox{Einstellungen}\\
%     Adresse des Proxy-Servers: \url{proxy.schunter.etc.tu-bs.de}\\
%     Anschluss: \verb#3128#\\
%     Ausnahmen: \url{schunter.etc.tu-bs.de} \url{schunternet.de}
%       \url{schuntille.de} \url{schunterkino.de}
% \end{description}

%Wer sich von der steigenden Anzahl an Werbebannern im Web gest"ort f"uhlt,
%kann auch den Port/Anschluss \verb#8000# eintragen und sich von dem dort
%h"orenden Junkbuster einen Gro"steil der Banner ausfiltern lassen.

\subsection{Usenet News}

Zur Teilnahme am weltweiten Informationsaustausch "uber das \glossar Usenet
kann im \snev der \glossar Server\footnote{Tats"achlich l"auft im lokalen
  Netz gar kein Newsserver im eigentlichen Sinne (abgesehen von dem f"ur die
  lokalen Gruppen), sondern lediglich ein \glossar Proxy
  (\texttt{newscache}). Das bedeutet, dem \glossar Client wird ein
  vollwertiger Newsserver suggeriert, aber die Daten werden von verschiedenen
  \glossar Servern geladen und an den Client weitergereicht sowie lokal
  gecached. Vorteil dieser Methode ist, da"s weniger Festplattenplatz
  vorgehalten werden muss und dennoch eine gro"se Auswahl an Newsgruppen
  angeboten werden kann, wobei zweite und folgende Zugriffe auf dieselben
  Nachrichten beschleunigt erfolgen. Nachteilig sind fehlender Einfluss auf
  die Haltezeiten und langsamerer Erstzugriff auf die Informationen.}
\url{news.schunter.etc.tu-bs.de} genutzt werden (Standard--Port ist 119).
Dieser stellt neben den im Rechenzentrum (\url{news.tu-bs.de}) aufliegenden
Newsgruppen %auch die Hierarchien corel.*, intel.*, linux.*, microsoft.*,
%
auch die Hierachie schunter.* zur Verf"ugung.

Letztere ist die lokale Hierarchie unseres Wohnheims mit den Gruppen 
\begin{tabbing}
  schunter.general.testnetmm\=\kill
  \url{schunter.general} \> Allgemeines im Wohnheim an der Schunter. \\
  \url{schunter.net} \> Das Netzwerk. \\
  \url{schunter.test} \> Tests nur hier!. 
\end{tabbing}

Im Usenet herrschen andere Gepflogenheiten als in Chatrooms oder auf
WWW--Boards. Empfohlen wird, zun"achst 
\begin{tabbing}
  schunter.general.testnetmm\=\kill
  \url{de.newusers.infos} \> Infos und periodische Postings fuer neue User. (Moderated)
\end{tabbing}
aufzusuchen und sich die dort st"andig neu geposteten 15 Artikel f"ur
Einsteiger durchzulesen. Fragen dazu kann man in
\begin{tabbing}
  schunter.general.testnetmm\=\kill
  \url{de.newusers.questions} \> Neue Benutzer im Netz fragen, Experten antworten.
\end{tabbing}
stellen, dort wird einem von kompetenten Leuten geholfen. Diese Informationen
und die Einhaltung der dortigen Empfehlungen k"onnen eine Menge "Arger
ersparen und halten so zumindest den deutschsprachigen Teil des Usenets (de.*)
nutzbar.

Zwingend erforderlich ist die Verwendung einer g"ultigen E-Mail--Adresse nach
\glossar RfC 1036. Um sich vor unerw"unschter Werbemail (Bulk E-Mail, SPAM) zu
sch"utzen ist es eine sehr gute Idee, sich z.B. entweder
\begin{enumerate}
  \item[a)] eine kostenlose Extra--Mailadresse f"ur das Usenet einzurichten
    (\url{www.hotmail.com}) die man dann gar nicht weiter nutzt, oder aber
  \item[b)] sich eine kostenlose E-Mail--Adresse bei \url{www.gmx.de} f"ur
    Mail \& News zu besorgen.
\end{enumerate}
Auf gar keinen Fall darf mit einer ung"ultigen bzw. nicht erreichbaren
E-Mail--Adresse gearbeitet werden.

Eine GMX-Adresse (mit automatischer Weiterleitung nach
\url{schunter.etc.tu-bs.de}) hat die Vorteile, da"s
\begin{enumerate}
  \item Man nicht st"andig das "`Konto"' wechseln bzw. beim Mailen statt News
    posten die Mail--Adresse nicht "andern muss.
  \item sehr effektive automatische Filter bei GMX vor SPAM sch"utzen, und
  \item beim Auszug aus dem Wohnheim bzw. K"undigung des Netzzugangs die Mails
    an GMX auf den k"unftigen Server umgeleitet werden k"onnen oder aber bei
    GMX verbleiben k"onnen, diese Adresse also stets g"ultig ist und nicht
    st"andig umge"andert werden muss.
\end{enumerate}

%\subsection{IRC (Internet Relay Chat)}

 \subsection{ICQ (I Seek You)}
\textbf{FIXME: Komplett entfernt, DAS sollte mittlerweile jeder
  kennen. Vielleicht durch Hinweis auf jabberserver ersetzen?}

% \subsubsection*{Was ist ICQ?}

% ICQ ist ein kleines n"utzliches Internet Tool, welches dich dar"uber
% informiert, wer gerade online ist und es einem erm"oglicht, mit denjenigen in
% Kontakt zu treten. ICQ "ubernimmt f"ur dich die Internetsuche nach einem
% Bekannten und informiert dich, wenn der Betreffende gerade online ist. Man
% kann mit ICQ Chatten, Mitteilungen, Dateien und URL's schicken oder einfach
% nur mit Bekannten rumh"angen, w"ahrend man durchs WWW surft.

% \subsubsection*{Wie funktioniert es?}

% Wenn du ICQ installierst (nachdem du es von
% \url{http://www.icq.com/download/step-by-step.html} heruntergeladen geladen
% hast), fragt dich das Programm danach, dich auf einem Server zu registrieren,
% welcher mit dem weltumspannenden ICQ-Netzwerk verbunden ist. Zum Zeitpunkt der
% Registrierung erh"altst du eine eindeutige ICQ-Nummer (die UIN -- Universal
% Internet Number -- am besten irgendwo notieren!), weiterhin gibt dir ICQ die
% M"oglichkeit, pers"onlich Informationen unter deiner ICQ\# zu speichern. Dies
% erm"oglicht anderen ICQ-Nutzern, dich eindeutig zu erkennen, wenn du dich
% einloggst. 

% Wenn du dich einmal registriert hast, kannst du eine Liste "uber
% Freunde und Bekannte f"ur dich erstellen, welche ICQ dann dazu benutzt, diese
% Bekannten f"ur dich zu suchen. W"ahrenddessen l"auft ICQ im Hintergrund, ohne
% andere laufende Anwendungen zu unterbrechen. Sobald du dich im Internet
% einloggst, informiert ICQ alle Bekannten, da"s du anwesend bist, und
% informiert dich dar"uber, wenn Bekannte sich ein-- oder ausloggen. Sobald du
% wei"st, wer online ist, reicht ein Rechtsklick auf den Namen des Betreffenden
% um einen Chat zu starten, eine Mitteilung zu schicken, Dateien auszutauschen,
% oder andere peer-to-peer Anwendungen zu starten.

% Weitere Tips gibt es unter \url{http://www.icq.com/icqtour/} 

% \subsubsection*{Firewall/Router-Einstellungen}

% \begin{sloppypar}
% Auf den Button \fbox{ICQ/Preferences \& Security/Preferences} klicken. Danach
% das Feld \fbox{Connection} ausw"ahlen. Nun auf \fbox{I'm using a permanent
%   internet connection (LAN)}, weiterhin \fbox{I am behind a firewall or proxy}
% $\Rightarrow$ \fbox{Firewall Settings}. Nun \fbox{I don't use a SOCKS Proxy
%   server on my Firewall or I am using another Proxy server}. \fbox{Next}
% w"ahlen, nun \fbox{Use dynamically allocated port Numbers (Default)}
% $\Rightarrow$ \fbox{Next} $\Rightarrow$ \fbox{Check My Firewall / Proxy
%   Setting}. Wenn bei "`Status"' "`Success"' steht, l"auft alles. Herzlichen
% Gl"uckwunsch! Nur noch auf den Button unten rechts (das Bl"umchen!)
% klicken. Ist es gr"un, seid ihr online.
% \end{sloppypar}

% Falls nicht, wendet euch doch bitte an die Newsgroup \url{schunter.general}
% auf dem Server \url{news.schunter.etc.tu-bs.de}.

% \subauthor{EWU (ICQ\# 37759570)}

\subsection{FTP}
\textbf{FIXME: ftp zeigt mittlerweile auf jupiter, ohne anonymen
  zugang, kann ergo raus, ggf. durch hinweis auf ,,spielwiese'' neptun
ersetzen, falls da jemand was draus machen mag}
%Auf dem lokalen \glossar ftp-Server \url{ftp://ftp.schunter.etc.tu-bs.de/}
%kann frei nutzbare Software zur Verf"ugung gestellt werden.

\subsection{NTP --- Time Service}

Zur Synchronisation der Rechneruhren im SchunterNet mittels des Network Time
Protocol (NTP) kann der \glossar Server \url{time.schunter.etc.tu-bs.de}
verwendet werden, welcher seinerseits mit prim"aren Timeservern im Internet
abgeglichen wird. Zur Einrichtung sei auf die Dokumentation der erforderlichen
\glossar Client--Software verwiesen.


\section{Benutzerserver (\texttt{jupiter.schunter.etc.tu-bs.de})}

\subsection{Zugriff auf das Homedirectory}

Wie so oft gibt es auch hier mehrere M"oglichkeiten: 

\subsubsection{\dots\ per ftp (File Transfer Protocol)}

Mit einem beliebigen \glossar ftp--Programm k"onnen Dateien in das
\glossar Home--Directory eingespielt werden. Auch die meisten anderen
Dateioperationen wie L"oschen, Verzeichnisse anlegen usw.\  k"onnen im
allgemeinen mit solchen Programmen durchgef"uhrt werden.

Beim Einlogproze"s m"ussen Benutzerkennung und Pa"swort angegeben werden, ein
Zugriff auf die \glossar Home--Directorys per anonymous--ftp ist nicht
m"oglich. Aus diesem Grund ist der Netscape Navigator f"ur diese Zugriffsart
nicht so gut geeignet.
Je nach verwendeten FTP-Client wird das Passwort außerdem im Klartext
übertragen, daher raten wir von einer Verwendung von ftp ab.

\subsubsection{\dots\ durch Einloggen auf dem Server} 

\textbf{FIXME: Komplett durch SSH Beschreibung ersetzen, ggf. mit
  Hinweis auf scp Beschränkung}
%Mit rlogin oder einem beliebigen Terminalprogramm (vorzugsweise\glossar
%telnet) kann man sich direkt auf dem \glossar Server (Name: "`jupiter"')
%einloggen und in einer \glossar UNIX--Shell--Umgebung (bash) arbeiten. Als
%Texteditoren stehen z.B. vi (bzw.\  vim, elvis) und joe sowie emacs zur
%Verf"ugung.

\subsubsection{\dots\ per SMB--Mount} 

F"ur Windows --Benutzer ist diese Variante vermutlich die komfortabelste
Zugriffsm"oglichkeit. Dies funktioniert folgenderma"sen:

\begin{sloppypar}
Unter \fbox{Systemsteuerung} $\Rightarrow$ \fbox{Netzwerk} wird "uber
\fbox{Hinzuf"ugen\dots} $\Rightarrow$ \fbox{Dienst} "`Datei und
Druckerfreigabe f"ur Windows--Netzwerke"' installiert, falls dies nicht schon
zuvor geschehen ist. Dann kann mit dem Windows--Explorer "uber \fbox{Extras}
$\Rightarrow$ \fbox{Netzlaufwerk verbinden} mit der Angabe
"`\url{\\jupiter\<username>}"' das \glossar Homedirectory auf einen neuen
Laufwerksbuchstaben abgebildet werden. Es ist sinnvoll, die Option
"`Verbindung beim Start wiederherstellen"' zu aktivieren. Das Pa"swort ist in
diesem Fall das Server--Pa"swort. Der Datei- und Druckerfreigabedienst kann
"ubrigens sp"ater wieder entfernt werden. Wenn er allerdings nie installiert
wurde, klappt auch der \glossar SMB--Zugriff auf fremde Laufwerke nicht (eine
Windows--Kuriosit"at).
\end{sloppypar}

Auch f"ur Linux--Benutzer besteht die M"oglichkeit, mit Hilfe von
\path{smbmount} diese Zugriffsmethode zu benutzen. Ein Nachteil dabei ist, da"s
normalerweise bei jedem Mount--Vorgang das Pa"swort eingegeben werden muss. Man
kann sich zwar leicht ein Skript schreiben, das diese Aufgabe "ubernimmt, aber
dann muss man auch das Pa"swort im Klartext in dem Skript unterbringen, was eine
potentielle Sicherheitsl"ucke darstellt. 

\subsubsection{\dots\ per NFS--Mount}
\textbf{FIXME: Überhaupt noch relevant?}

%F"ur Linux und alle anderen \glossar UNIX--Varianten ist dies die richtige
%L"osung. Da sie im Wohnheim relativ selten genutzt wird, ist eine Freigabe des
%\glossar Home--Directorys per \glossar NFS nicht initial f"ur alle Benutzer
%eingerichtet. Wer hiervon Gebrauch machen m"ochte, sollte eine kurze Mitteilung
%an \url{admin@jupiter} mit Angabe des Loginnamens und der eigenen IP--Adresse
%schicken.

%Ein Tip: Wer sein Server--Home--Directory in den Home--Directory--Bereich eines
n%ormalen Benutzers auf der eigenen Maschine mounten m"ochte, sollte daf"ur
s%orgen, da"s die UID und GID auf beiden Rechnern "ubereinstimmen. Das vermeidet
P%robleme mit File--Permissions. 


\subsection{Private Homepages}

Jeder Benutzer kann in seinem HOME--Bereich auf dem Benutzerserver ein
Unterverzeichnis mit dem Namen \path{public_html} anlegen (auf  
\url{jupiter.schunter.etc.tu-bs.de} sollte dieses bereits existieren), das der
\glossar WWW-Server als Do\-ku\-men\-ten--Verzeichnis f"ur pers"onliche \glossar
WWW--Seiten interpretiert. Eine dort liegende Datei mit dem Namen
\path{index.html} wird als \glossar Homepage interpretiert (alternativ ist
auch \path{index.htm} m"oglich). Der URL dieser \glossar Homepage besteht aus
dem Namen des \glossar WWW-Servers, dem Tilde-Zeichen (\verb#~#) und dem Login
des Benutzers (Beispiel: \url{http://www.schunter.etc.tu-bs.de/~user/}). Es
sollte euch bewusst sein, da"s dieses Verzeichnis mittels \glossar WWW-Server
weltweit eingesehen werden kann.

Das Erzeugen eines Unterverzeichnisses \path{public_html} geht (unter Linux)
so:
%\begin{itemize}
%  \item Einloggen auf \url{jupiter.schunter.etc.tu-bs.de} mittels Kommando:
%    \texttt{telnet jupiter} oder: \verb#ftp jupiter# und anschlie"sender
%    \glossar Authentifizierung
%  \item Kommando: \verb#mkdir ~/public_html# zur Erzeugung des
%    Unterverzeichnisses, in dem die \glossar WWW--Seiten liegen sollen 
%  \item Kommando: \verb#chmod 755 ~/public_html#, damit jedermann (!)
%    dieses Verzeichnis lesen kann
%  \item Kommando: \verb#chmod 711 ~#, damit jedermann ein vorhandenes
%    Verzeichnis im HOME--Bereich finden kann
%\end{itemize}

Das Verzeichnis kann aber auch vom PC unter Windows per \glossar ftp mit
graphischer Oberfl"ache angelegt werden, z.B. mit WS\underline{\ \
}FTP oder per SMB-Mount. Auch
dazu m"usst ihr euch auf \url{jupiter} einloggen. Ein Verzeichnis k"onnt ihr dort
per Knopfdruck anlegen.

Legt im Verzeichnis \path{public_html} eine Datei mit dem Namen
\path{index.html} an (absoluter Pfad:
\path{~/public_html/index.html}). Diese Datei findet man mit dem URL:
\url{http://www.schunter.etc.tu-bs.de/~/user/}

Nat"urlich m"ussen auch alle Dateien in diesem Verzeichnis zum Lesen freigegeben
werden. 
%\begin{itemize}
%  \item Kommando: \verb#chmod 644 ~/public_html/*#
%\end{itemize}

Zus"atzlich kann im selben Verzeichnis (\path{~/public_html/}) eine Datei
\path{description.txt} mit einer kurzen Beschreibung des Inhalts der \glossar
Homepage (eine Zeile, maximal 50 Zeichen) angelegt werden. Diese wird dann auf
der "Ubersichtsseite der privaten \glossar Homepages im SchunterNet mit angezeigt.


\section{Sicherheit}

\subsection{Viren, Trojaner, Bugs}
\label{viren}

\begin{quote}
"`The degree to which you take security seriously and invest
in it should be proportional to the value and sensitivity of
your system and its data."' (aus \url{http://www.nwi.net/~pchelp/security/advice.htm})
\end{quote}

Dank Computer-BILD wei"s es jede/r: Im Internet warten Tausende 
von Kriminellen nur darauf, da"s du deinen Computer an das Netz anschlie"st.
Wenn du also vermeiden willst, da"s diese b"osen Jungs und M"adchen sich an 
deinem Home-Banking-Account, deiner Diplomarbeit oder gar deinen 
Quake-Spielst"anden zu schaffen machen, solltest Du einige Sachen beachten. 

Damit die Angreifer (f"alschlicherweise oft auch Hacker genannt) auf deinen
Computer bzw. deine Daten zugreifen k"onnen, muss ein bestimmtes Programm auf 
deinem Computer laufen. Dabei gibt es mehrere M"oglichkeiten:

\subsubsection{Computerviren}
Ein Computervirus ist eine Sequenz von Programmcode, welche 
in anderen, n"utzlichen Code eingef"ugt ist und mit diesem ausgef"uhrt wird. 
Dabei versucht sich der Code in andere Programme zu kopieren, d.h. dieses zu 
infizieren. Viren befinden sich also immer in einem "`Wirtsprogramm"'. 
Bootsektor--Viren setzen sich in den Bootsektor von Disketten oder Festplatten
fest. Dieser Programmcode wird nach dem Booten des Rechners direkt gestartet.
Damit ein Virus von einer geliehenen Diskette nicht gleich gestartet
wird, stelle im BIOS die Boot-Sequenz zuerst auf Laufwerk C.
Weiterhin kann sich ein Virus in einem Anwendungsprogramm (z.B. MS-Word) oder
in einem Spiel verstecken. Allerdings k"onnen auch andere Dateien ausf"uhrbaren
Code enthalten: MIME--encoded Mail, WWW-Seiten mit JavaScript oder VBScript,
Postscript Dateien, Word-Dateien,~\dots Gegen Viren hilft ein (besser zwei) gutes
und aktuelles Virenkiller-Programm und seine h"aufige Anwendung.

\subsubsection{Trojanische Pferde (Trojan Horses)} 
Analog zur griechischen Mythologie wird
hier ein K"oder ausgelegt. Neben diesem erh"alt man jedoch unerw"unschten
Programmcode. Trojaner k"onnen sich prinzipiell in den gleichen Dateitypen
wie Viren aufhalten. Starte daher niemals (!) Programme unbekannter Herkunft 
auf deinem Computer. Gleiches gilt nat"urlich auch f"ur den Crack von
\url{www.evilhacker.org} (gibt's die wirklich?) oder Programme die du auf einem
Rechner hier im lokalen Netz gefunden hast. Wenn Dich Dein Mailprogramm fragt,
ob es ein Programm ausf"uhren soll, und du bist dir nicht absolut sicher, warum,
antworte mit nein! Eines der ausgereiftesten Trojaner ist z.Z. SubSeven: mehr
Infos dazu unter \url{http://home.t-online.de/home/TschiTschi/subseven.htm}.

\subsubsection{Bugs}
Fehlerhafte Programme die du bereits installiert hast, k"onnen offene
Angriffspunkte enthalten. Softwarehersteller 
bringen h"aufig Updates zu ihren Betriebssystemen oder Anwendungsprogrammen heraus. 
Halte dich auf den Laufendem und aktualisiere auf neuere Versionen, insbesondere, 
wenn Sicherheitsl"ucken bekannt geworden sind.  

Sicherheit im Computerbereich ist ein Katz--und--Maus--Spiel. Informiere dich
deshalb laufend "uber die aktuellen Entwicklungen. Sicherheit ist kein Produkt
was man fertig installieren kann, sondern ein st"andiger Proze"s. Sei skeptisch,
wenn Dir jemand sagt, mit seinem Produkt bist Du sicher. Auch unsere Firewall
bietet nur begrenzten Schutz gegen Angriffe von "`au"sen"', gegen Angriffe aus 
dem Wohnheim ist sie v"ollig nutzlos.  

Infos zum Thema Sicherheit findet man z.B. unter \url{http://www.insecure.org/} im Internet.

Zum Schluss noch eine Bitte: Wenn ihr eine Email erhaltet, in der vor einer
anderen mit dem Betreff: XYZ gewarnt wird und man euch auffordert, diese an 
m"oglichst viele Leute weiter zu senden, dann leitet diese NICHT weiter! 
Solange man Emails nur liest und nicht ausf"uhrt, k"onnen sie niemals Schaden
anrichten!

\subauthor{Richard Karsch}

\subsection{Pa"sw"orter}
\label{passwort}

\subsubsection*{Wie "andere ich das Pa"swort?}
% Geht mittlerweile nur noch via ssh
\textbf{FIXME: Geht nur via kpasswd und nur via ssh, das wir ja
  eigentlich nicht mehr zulassen wollen-> ggf. Skript entwickeln und
  auf Webseite packen?}
% Um das Pa"swort zu "andern, muss zun"achst ein Terminalprogramm gestartet werden,
% z.B. \glossar telnet. Als Zieladresse wird der Rechner jupiter angegeben
% (IP--Adresse ist 134.169.168.3, falls der \glossar DNS--Server nicht benutzt
% werden kann). Man erh"alt dann folgende (oder "ahnliche) Begr"u"sungsmeldung: 

% \begin{verbatim}
% Welcome to S.u.S.E. Linux 5.3 (i386) - Kernel 2.0.36 (ttyp5).

% jupiter login: 
% \end{verbatim}

% Es folgt die Eingabe der Benutzerkennung und des (alten) Pa"sworts. Bei der
% Eingabe des Pa"sworts erfolgt kein Echo auf dem Bildschirm, auch nicht
% symbolisch (z.B. durch *). Danach erscheint folgende (oder eine "ahnliche)
% Meldung:

% \begin{verbatim}
% Last login: Thu Mar 20 15:30:06 on ttyp1 from erde.schunter.et
% No mail.
% loginname@jupiter:/home/users/loginname >
% \end{verbatim}

% Man befindet sich nun in einer \glossar UNIX--Shell--Umgebung. Zum "Andern des
% Pa"sworts reicht es, das Kommando \verb#kpasswd# zu kennen. Nach dessen Aufruf
% wird man zun"achst nach dem alten, und dann zweimal nach dem neuen Pa"swort
% gefragt. Falls bei der Frage nach dem neuen Pa"swort zwei unterschiedliche
% Zeichenketten angegeben wurden (Tippfehler), erscheint die Meldung

% \verb#They don't match; try again.#

% und man darf noch mal ein neues Pa"swort angeben (wiederum zweimal). Wichtig:
% In der \glossar UNIX--Welt wird grunds"atzlich zwischen Gro"s- und
% Kleinschreibung unterschieden, also auch beim Pa"swort. Wenn alles geklappt
% hat, kann man die Umgebung mit dem Kommando \verb#exit# wieder verlassen. Auf
% spezielle deutsche Sonderzeichen (Umlaute und Esszett) sollte beim Pa"swort
% verzichtet werden. Auch wenn die \glossar UNIX--Anmeldung und das \glossar
% E-Mail abholen noch klappt, so gibt es doch sp"atestens bei \glossar
% SMB--Zugriffen (\glossar Homedirectory abbilden, Drucken "ubers Netz)
% Schwierigkeiten.

% Das Terminalprogramm \glossar telnet geh"ort zum Lieferumfang von s"amtlichen
% \glossar UNIX--Clones, Windows95/98 und Windows NT. F"ur andere Betriebssysteme
% muss es gegebenenfalls noch nachinstalliert werden. Prinzipiell kann man auch
% jedes andere Terminalprogramm verwenden, das eine DEC--vt100 Terminalemulation
% beherrscht.

\subsubsection*{Wie sicher ist das Pa"swort? Wie sicher sollte es sein?}

Die Sicherheit des Pa"sworts sollte nicht untersch"atzt werden. Erstens sind
manche Pa"sw"orter leichter zu knacken als andere, und zweitens kann mit einem
geklauten oder geknackten Pa"swort weitaus mehr Unfug getrieben werden als nur
fremde \glossar E-Mails zu lesen.

Wer eine Vorstellung davon bekommen m"ochte, wozu ein unbefugter Zugang
mi"sbraucht werden kann, dem sei das Buch "`Kuckucksei"' von Clifford Stoll
empfohlen. Es handelt sich dabei "ubrigens nicht um ein Informatik--Fachbuch,
sondern um einen tatsachenbasierten Roman. Man erf"ahrt auf sehr humorvolle und
auch f"ur Laien verst"andliche Weise, wie Computersysteme angegriffen und
mi"sbraucht werden k"onnen. 

Das Pa"swort sollte also m"oglichst schwer zu knacken sein. Gegen systematische
alphabetische Angriffe (Ausprobieren aller Kombinationen aller verf"ugbaren
Zeichen) ist im Grunde kein Kraut gewachsen. Allerdings ist der Zeitaufwand
derart hoch, da"s diese Angriffsart kaum praktikabel ist. Die
Wahrscheinlichkeit, das richtige Pa"swort zu erwischen, ist geringer als die,
da"s das Pa"swort inzwischen ge"andert wurde. H"aufiger sind dagegen die
sogenannten W"orterbuchattacken, bei denen nur solche Pa"sw"orter ausprobiert
werden, die auch sinnvolle W"orter ergeben, wie sie eben in W"orterb"uchern einer
beliebigen Sprache stehen (deutsch oder englisch zum Beispiel). Auch
Namenslisten (menschliche Vornamen, Firmennamen, etc.) werden oft als
Grundlage benutzt.

\subsubsection*{Und wie w"ahle ich nun mein Pa"swort?}

In der letzten Frage wurde gekl"art, welche Zeichenketten nicht als Pa"sw"orter
benutzt werden sollen (Echte deutsche oder englische W"orter, Eigennamen,
Firmennamen usw). Es gibt mehrere Regeln, die ein Pa"swort gegen"uber den
"ublichen Attacken praktisch immun machen: 

\begin{enumerate}
  \item Gro"s- und Kleinschreibung gemischt verwenden (wird unterschieden)
  \item Zahlen mit einbauen
  \item Sonderzeichen mit einbauen (allerdings keine speziellen deutschen
    Sonderzeichen, siehe oben)
  \item Pa"swort nicht zu kurz w"ahlen (8 Zeichen sind optimal) 
\end{enumerate}

Als n"achstes stellt sich m"oglicherweise die Frage, wie man sich ein solches
sicheres Pa"swort noch merken k"onnen soll. Es gibt verschiedene Eselsbr"ucken
und der Phantasie sind keine Grenzen gesetzt. Eine M"oglichkeit: Man nimmt sich
eine Zeile aus seinem Lieblingsgedicht, -song, oder was auch immer und greift
sich die Anfangsbuchstaben heraus. Aus "`We don't need no Education"' ergibt
sich beispielsweise das Pa"swort "`WdnnE"' --- leicht zu merken und schwer zu
erraten bzw.\  zu knacken.

% \subsection{ssh --- Die Secure Shell}
%in benutzerserver bereich verlagert.
% Bei konventionellen Methoden des Arbeitens auf entfernten Rechnern
% (\glossar \texttt{telnet}, \texttt{rlogin}, \texttt{rsh}) werden Eingaben und
% Ausgaben unverschl"usselt "uber das Netz "ubertragen. Das schlie"st bereits die
% \glossar Authentifizierung ein, d.h.\  das Pa"swort 
% wird im Klartext "ubertragen und erst auf dem Zielrechner verschl"usselt und
% verifiziert. Dadurch hat jeder, der sich auf einem System befindet, welches
% auf die "Ubertragungswege zugreifen kann, die M"oglichkeit, diese Daten
% mitzuh"oren und bspw. Pa"sw"orter auszufiltern. 
% Auch wenn dies innerhalb unseres Wohnheimnetzes aufgrund der physikalischen
% Struktur nicht m"oglich ist, sollte es dennoch ber"ucksichtigt werden. 

% Eine M"oglichkeit, die Kommunikation sicherer zu machen, ist die Secure Shell,
% ein Programm zum Login und zur Ausf"uhrung von Kommandos auf entfernten
% Maschinen. Es soll \texttt{rlogin} und \texttt{rsh} ersetzen und stellt einen
% sicheren verschl"usselten Kommunikationsweg zwischen zwei Rechnern "uber ein
% unsicheres Netzwerk zur Verf"ugung. Auch X11--Ausgaben und beliebige
% TCP/IP--Verbindungen k"onnen "uber den verschl"usselten Kanal "ubertragen werden.

%\subsection{Kerberos}

%\url{http://www.winfile.com/}

\begin{appendix}
\settocdepth{1}

\clearpage

\setcounter{para_nr}{0}

\section{Satzung des Vereins SchunterNet e.V.}
\label{satzung}

{\small Stand: 05. August 1997}

\Paragraph{Name und Sitz}
\begin{enumerate}
\item Der Verein f"uhrt den Namen "`\snev"'
\item Der Verein hat seinen Sitz in Braunschweig.
\item Der Verein ist eingetragener Verein im Vereinsregister beim Amtsgericht 
Braunschweig.
\end{enumerate}

\Paragraph{Vereinszweck}
\begin{enumerate}
\item Zweck des Vereins \snev ist die Planung, der Aufbau
    und der Betrieb eines Computernetzwerkes im "`Studentenwohnheim An der
    Schunter"' einschlie"slich dessen Anbindung an das Netz der Technischen
    Universit"at Braunschweig, sowie die F"orderung der Kommunikation von
    Studenten im nationalen und internationalen Rahmen.
\item Der Verein verfolgt ausschlie"slich gemeinn"utzige Zwecke und ist nicht
    eigenwirtschaftlich t"atig.
\end{enumerate}

\Paragraph{Mitgliedschaft}\label{mitglied}
\begin{enumerate}
\item Mitglied kann jeder Bewohner der Wohnheimanlage "`Studentenwohnheim An
  der Schunter"' werden.
\item Der Verein besteht aus aktiven und passiven Mitgliedern.
\item Passives Mitglied kann werden, wer einen Antrag auf einen
  SchunterNet--Anschluss gestellt hat und die Satzung sowie deren
  Erg"anzungsordnungen anerkennt.
\item Aktives Mitglied kann werden, wer einen Antrag auf einen
  SchunterNet--Anschluss gestellt hat, die Satzung und deren
  Erg"anzungsordnungen anerkennt und aktiv am Aufbau und Betrieb des Netzes
  mitarbeitet.
\item \label{aufnahme} "Uber die Aufnahme aktiver Mitglieder entscheidet die
  Mitgliederversammlung mit einfacher Mehrheit. Die Mitgliedschaft beginnt mit
  der Aufnahme durch die Mitgliederversammlung.
\item Ein passives Mitglied kann in jeder Mitgliederversammlung einen Antrag
  auf aktive Mitgliedschaft stellen. Die Aufnahme erfolgt entsprechend
  Ziffer \ref{aufnahme}.
\item Die Mitgliedschaft endet durch Verlust der Gesch"aftsf"ahigkeit, Auszug
  aus dem "`Studentenwohnheim An der Schunter"' oder Austrittserkl"arung.
\item Der Austritt ist schriftlich gegen"uber dem Vorstand zu erkl"aren 
  und wird einen Monat nach Eingang der schriftlichen Austrittserkl"arung bei 
  dem Vorstand wirksam.
\item \label{ausschluss}Mitglieder, die dem Zweck und Ansehen des Vereins
  zuwider handeln oder gegen Bestimmungen der g"ultigen Satzung oder der
  Erg"anzungsordnungen versto"sen, k"onnen durch Beschluss des Vorstandes aus dem
  Verein ausgeschlossen werden. Widerspricht der Betroffene innerhalb eines
  Monats, entscheidet die Mitgliederversammlung "uber den Ausschluss mit 2/3
  Mehrheit. Bis zur Entscheidung der Mitgliederversammlung ruhen die
  Mitgliedschaftsrechte des betroffenen Mitglieds.
\end{enumerate}

\Paragraph{Beitr"age}
\begin{enumerate}
\item[] Von den Mitgliedern werden keine Beitr"age erhoben.
\end{enumerate}

\Paragraph{Organe}
\begin{enumerate}
\item[] Die Organe des Vereins sind:
  \begin{enumerate}
  \item die Mitgliederversammlung
  \item der Vorstand
  \end{enumerate}
\end{enumerate}

\Paragraph{Mitgliederversammlung}
\begin{enumerate}
\item[]
  \begin{sloppypar}
    Die Mitgliederversammlung bestimmt auf der Grundlage des Vereinszwecks 
    die Richtlinien f"ur die T"atigkeit des Vereins.
  \end{sloppypar}

  Sie ist im "ubrigen insbesondere zust"andig f"ur:
  \begin{enumerate}
  \item Die Entgegennahme des Jahresberichtes des Vorstandes
  \item Die Erteilung von Entlastungen
  \item Die Wahl des Vorstandes
  \item Die Wahl der Systemverwaltung
  \item Satzungs"anderungen und Aufl"osung des Vereins
  \item Aufnahme von Mitgliedern und, im Falle des Widerspruchs gegen den den
    Ausschluss aussprechenden Vorstandsbeschluss, f"ur den Ausschluss von
    Mitgliedern.
\end{enumerate}
\end{enumerate}

\Paragraph{Einberufung der Mitgliederversammlung}
\begin{enumerate}
\item Die ordentliche Mitgliederversammlung muss mindestens einmal im Semester
  stattfinden. Der Termin der Mitgliederversammlung ist mindestens eine Woche
  vorher bekannt zu geben. Die Bekanntgabe des Termins erfolgt durch Aushang
  am Anschlagbrett der Heimselbstverwaltung oder E-Mail.
\item Au"serordentliche Mitgliederversammlungen finden statt:
  \begin{enumerate}
  \item auf Beschluss des Vorstandes oder
  \item wenn dies 10\% der Mitglieder unter Angabe des Zwecks verlangen.
  \end{enumerate}
  Die Versammlung wird vom Vorstand durch Aushang am schwarzen Brett oder
  E-Mail mit einer Ladungsfrist von einer Woche unter Mitteilung der
  Tagesordnung einberufen. 
\item Die Mitgliederversammlung beschlie"st mit einfacher Mehrheit der 
  Stimmen. Stimmberechtigt ist jedes aktive Mitglied.

  Satzungs"anderungen, die vorzeitige Abwahl des Vorstandes und die 
  Entscheidung "uber den Ausschluss von Mitgliedern nach \S{}3
  Ziffer \ref{ausschluss} dieser Satzung erfordern eine Mehrheit von 2/3 der
  abgegebenen Stimmen.
\item Jedes aktive und passive Mitglied hat in der Mitgliederversammlung
  Rederecht und darf Antr"age stellen. Werden gegen einen Beschluss die
  Unterschriften von mehr als der H"alfte aller aktiven und passiven
  Mitglieder vorgelegt, so gilt dieser als nicht gefa"st.
\end{enumerate}

\Paragraph{Beschlussf"ahigkeit der Mitgliederversammlung}
\begin{enumerate}
\item Jede ordnungsgem"a"s einberufene Mitgliederversammlung ist
  beschlussf"ahig, wenn mindestens 2/3 der aktiven Mitglieder anwesend 
  sind.
\item Im Falle der Beschlussunf"ahigkeit ist die Mitgliederversammlung
  innerhalb eines Monats erneut einzuberufen. Diese ist dann ohne R"ucksicht
  auf die Anzahl der erschienenen Mitglieder beschlussf"ahig.
\end{enumerate}

\Paragraph{Vorstand}
\begin{enumerate}
\item Die Zahl der Vorstandsmitglieder bestimmt die Mitgliederversammlung.\\
  Der Vorstand besteht jedoch mindestens aus drei Mitgliedern, n"amlich dem 
  Vorsitzenden, dem stellvertretenden Vorsitzenden und dem Kassenwart.
\item Je zwei Vorstandsmitglieder vertreten den Verein gemeinsam.
\item Die Vorstandsmitglieder werden von der Mitgliederversammlung einzeln und
  auf die Dauer von zwei Studiensemestern gew"ahlt.
  
  Jedes Vorstandsmitglied bleibt im Amt, bis die Amtszeit des neugew"ahlten
  Nachfolgers beginnt oder die Mitgliederversammlung beschlossen hat, sein Amt
  nicht wieder zu besetzen. Eine Wiederwahl ist m"oglich.

  Die vorzeitige Abwahl eines Vorstandsmitgliedes kann nur mit 2/3 Mehrheit der
  ordnungsgem"a"s einberufenen Mitgliederversammlung erfolgen. Die
  Abwahl eines Vorstandsmitgliedes wird erst wirksam, wenn sich die 
  Mitgliederversammlung zugleich auf einen Nachfolger geeinigt oder beschlossen
  hat, sein Amt nicht wieder zu besetzen.
\end{enumerate}

\Paragraph{Aufgaben des Vorstandes}
\begin{enumerate}
\item[] Der Vorstand sorgt f"ur die Durchf"uhrung der Beschl"usse der
  Mitgliederversammlung und f"ur die Information der Mitglieder.
\end{enumerate}

\Paragraph{Beurkundung von Beschl"ussen}
\begin{enumerate}
\item[] Der Schriftf"uhrer fertigt "uber Beschl"usse der Mitgliederversammlung
  Protokolle an, die vom Versammlungsleiter und ihm unterschrieben werden.
\end{enumerate}

\Paragraph{Schlussbestimmungen}
\begin{enumerate}
\item Im Falle einer Auf\/l"osung des Vereins f"allt das Vereinsverm"ogen der
  Heimkasse des Wohnheims "`An der Schunter"' zu.
\item Diese Satzung tritt mit dem Beschluss der Gr"undungsversammlung vom
  23. April 1997 in Kraft.
\end{enumerate}

\clearpage

\setcounter{para_nr}{0}

\section{Benutzerordnung des SchunterNet e.V.}
\label{nutzerordnung}

{\small Stand: 16. Februar 2000}

\mbox{ }

{\large\sf\textbf{Pr"aambel}}

Das im Studentenwohnheim \emph{An der Schunter} durch den
\snev betriebene Netzwerk soll allen Mietern die
M"oglichkeit bieten, mit ihren Heimrechnern (Mac, PC, etc.) einfach und
kosteng"unstig an moderner Datenkommunikation zu partizipieren. Die
gemeinsame Nutzung von Ressourcen steht dabei im lokalen Netz im
Mittelpunkt. Dazu geh"oren der Datenaustausch "uber das Netz, die
Nutzung zentraler oder privater Peripherieger"ate (Drucker, Scanner,
Streamer, etc.) sowie die Bereitstellung nichtkommerzieller Software
auf File--Servern.

Mit der Anbindung an das Hochschulnetz erh"alt der Teilnehmer die
M"oglichkeit, seinen PC als Terminal f"ur eine UNIX--Workstation
einzusetzen. Durch die Verbindung mit dem weltweiten Internet
bieten sich den Studenten letztendlich ganz neue Wege der Recherche
und Informationsbeschaffung vom eigenen Schreibtisch aus. Zentraler
Punkt ist hierbei das World Wide Web, welches in den letzten Jahren
die Popularit"at des Internet erst begr"undet hat.

\mbox{ }

\Paragraph{G"ultigkeit}

\begin{enumerate}
  \item Die folgenden Regelungen gelten f"ur alle Benutzer des Netzes
      des \snev im Studentenwohnheim \emph{An der Schunter}, Braunschweig. Sie
      erg"anzen die Benutzungsordnung f"ur das Rechenzentrum der Technischen
      Universit"at Braunschweig. [Anhang \ref{RZordnung}]
  \item Diese Ordnung tritt mit ihrer Ver"offentlichung in Kraft. Sie
      verliert ihre G"ultigkeit bei Inkrafttreten einer neuen Benutzerordnung.
\end{enumerate}

\Paragraph{Allgemeine Bestimmung}

\begin{enumerate}
  \item Die Teilnahme an Datennetzen verlangt von jedem einzelnen einen
      verantwortungsvollen Umgang mit diesem Medium. Die Benutzerordnung wurde
      geschaffen, um die Funktionsf"ahigkeit des Netzwerkes und ein geregeltes
      Miteinander der Teilnehmer zu gew"ahrleisten.
  \item Jeder Benutzer verpflichtet sich, diese Ordnung anzuerkennen.
  \item F"ur die Nutzung der Ressourcen des Hochschulnetzes (TUBS--Net
      und Zugang zum Internet) ist dar"uber hinaus  die
      Benutzungsordnung f"ur das Rechenzentrum der Technischen
      Universit"at Braunschweig [Anhang \ref{RZordnung}] verbindlich.
  \item Betriebs- und Hardwarekosten werden entsprechend der
      Geb"uhrenordnung [Anhang \ref{gebuehr}] des \snev auf die Nutzer
      umgelegt.
\end{enumerate}

\Paragraph{Zulassung der Benutzer}

\begin{enumerate}
  \item Grunds"atzlich ist jeder Bewohner des Studentenwohnheims
      \emph{An der Schunter} berechtigt, sich an das Wohnheimnetz
      anzuschlie"sen, sofern er sich mit den hier aufgef"uhrten
      Regelungen einverstanden erkl"art und dem \snev beitritt.
  \item Einschr"ankungen werden im Einzelfall durch den \snev ausgesprochen.
\end{enumerate}

\Paragraph{An- und Abmeldung}

\begin{enumerate}
  \item Zur Anmeldung ist der Antrag auf Netzanschluss sowie Mitgliedschaft
      im \snev auszuf"ullen und unterschrieben beim Vorstand des \snev
      einzureichen. Dieser stellt einen Nutzervertrag des Teilnehmers mit dem
      Betreiber dar.
  \item "Anderungen der Benutzerdaten sind dem Verein unverz"uglich
      mitzuteilen.
  \item Die Teilnahme kann durch Auszug aus dem Wohnheim, Abmeldung
      oder Ausschluss (s. \S{}8) beendet werden. Auszug oder Abmeldung sind dem
      Verein mindestens sechs Wochen im voraus anzuk"undigen.
\end{enumerate}

\Paragraph{Rechte des Benutzers}

\begin{enumerate}
  \item Jeder Benutzer hat das Recht, den ihm zur Verf"ugung gestellten
      Netzanschluss zu jeder Zeit im Rahmen dieser Benutzerordnung zu
      nutzen.
  \item Grunds"atzlich kann jeder Benutzer alle zur Verf"ugung
      gestellten Dienste des Netzes in Anspruch nehmen.
  \item Der Benutzer wird im Rahmen der M"oglichkeiten durch die
      Vertreter des \snev beraten und betreut. Dies
      ist zu den festgelegten Sprechstunden in den R"aumlichkeiten des
      Vereins m"oglich.
\end{enumerate} 

\Paragraph{Bereitgestellte Dienste des Netzes}

\begin{description}
  \item[Prim"arer Anschluss] Der prim"are Anschluss an
      der im Zimmer des Benutzers vorhandenen Netzwerkdose wird zur
      Verf"ugung gestellt und auf Antrag freigeschaltet. Der \snev
      ist stets um einen sicheren und unterbrechungsfreien Betrieb des
      Wohnheimnetzes bem"uht, soweit dies beim Stand der Technik und im
      zeitlichen Rahmen der Mitglieder m"oglich ist.
  \item[Sekund"arer Anschluss] Auf gesonderten Antrag
      kann bei ausreichenden Ressourcen der in der Dose vorhandene
      Zweitanschluss f"ur Netzwerknutzung zus"atzlich freigeschaltet
      werden.
  \item[Protokoll des Anschlusses]  Der Anschluss erfolgt "uber
      10 MBit/s Ethernet (10BaseT) und erm"oglicht je nach Netzwerkadapter des
      Benutzers Half-- bzw. Full--Duplex--Betrieb.
  \item[MAC Adressen der Netzwerkadapter] Jedem
      Anschluss werden aus Gr"unden der Sicherheit bis maximal vier vom
      Benutzer angegebene MAC--Adressen (Media Access Control,
      eindeutige Identifizierungsnummer eines Netzwerkadapters) fest
      zugeordnet.
  \item[Protokolle] Im internen Netz wird
      grunds"atzlich jedes Protokoll (TCP/IP, IPX/ODI\tm, NetBIOS\tm,
      appletalk\tm, etc.) durchgeschaltet. Die Verbindung zum
      Hochschulnetz erfolgt ausschlie"slich "uber das Internetprotokoll
      (IP).
  \item[IP--Adresse] Jedem Anschluss wird eine statische
      IP--Adresse zur Verf"u\-gung gestellt, die den Rechner im gesamten
      Internet eindeutig identifiziert. Hierdurch wird es grunds"atzlich
      m"oglich, die Dienste des Internet in Anspruch zu nehmen und von au"sen
      (z.B. vom Rechenzentrum aus) auf die Ressourcen des eigenen Rechners
      zuzugreifen. Einschr"ankungen oder Erweiterungen dieser
      Zugriffsm"oglichkeiten k"onnen nach den W"unschen des Benutzers vom
      Administrator eingestellt werden.
  \item[E-Mail] Jeder Benutzer erh"alt eine eigene
      Mailbox mit Adresse, auf die er "uber POP3 oder Imap zugreifen
      kann.
  \item[WWW-- und FTP--Zugang] Der Zugang zum
      World Wide Web und zu FTP--Servern wird "uber einen Proxyserver
      gew"ahrleistet.
  \item[Usenet News] Je nach vorhandenen Ressourcen
      werden Newsgruppen nach den W"un\-schen der Benutzer lokal zur
      Verf"ugung gestellt.
  \item[lokaler NTP Service] Es wird eine zeitgenaue
      Synchronisierung der Systemuhren der eigenen Rechner erm"oglicht.
  \item[Benutzerserver] Jedem Teilnehmer wird ein
      Arbeitsverzeichnis auf dem Benutzerserver (Linux--Workstation)
      zur Verf"ugung gestellt. Nach Wunsch k"onnen projektbezogene
      Benutzergruppen eingerichtet werden. Jeder Benutzer darf bis zu neun
      weitere Accounts mit eingeschr"ankten M"oglichkeiten beantragen.
  \item[private WWW--Homepages] Jeder Benutzer kann im
      Rahmen des zur Verf"u\-gung gestellten Speicherplatzes eigene \glossar
      Homepages erstellen und ver"offentlichen.
  \item[weitere Dienste] Der Verein kann dar"uber
      hinaus weitere Dienste zur Verf"u\-gung stellen.
\end{description}

\Paragraph{Pflichten des Benutzers}

Jeder Teilnehmer ist f"ur seinen Rechner und den Netzzugang "uber selbigen
voll verantwortlich. Das bedeutet:

\begin{enumerate}
  \item Der Teilnehmer hat die ihm zur Verf"ugung gestellten Betriebsmittel
      und Dienste sorgf"altig und ihren Bestimmungen entsprechend zu benutzen.
  \item Jeder Teilnehmer hat Ma"snahmen zum Schutz vor
      unbefugter Nutzung seines Anschlusses und der zur Verf"ugung gestellten
      Dienste durch Dritte zu ergreifen.
  \item Jeder Verdacht auf Mi"sbrauch von Ressourcen ist der Netzverwaltung
      unverz"uglich zu melden.
  \item Bauliche Ver"anderungen an der Netzwerkinstallation d"urfen nur mit
      schriftlicher Genehmigung des \snev vorgenommen werden.
  \item Die St"orung oder Beeintr"achtigung des Netzbetriebs durch unsachgem"a"sen
      Einsatz von Hard- und Software ist zu vermeiden. St"orungen jeder Art
      sind unverz"uglich dem \snev zu melden.
  \item Es ist dem Teilnehmer verboten,  eine andere als die ihm zugewiesene
      IP--Adresse im Netz zu benutzen (Address--Spoofing) oder Masquerading zu
      betreiben.
  \item Der am Netz angeschlossene Rechner darf grunds"atzlich nicht f"ur
      Routingzwecke verwendet werden. Ausnahmen bed"urfen der schriftlichen
      Genehmigung des \snev
  \item Jede Art des Mith"orens von Daten"ubertragungen, des unberechtigten
      Zugriffs auf fremde Daten oder des unberechtigten Zugangs zu fremden
      Rechnern ist zu unterlassen. Schon der Versuch ist strafbar.
  \item Die Bereitstellung und Nutzung von Software und Dokumentationen ist
      nur im Rahmen der ma"sgeblichen Lizenzbestimmungen zul"assig.
  \item Das Beziehen oder Verbreiten strafrechtlich relevanter Daten ist zu
      unterlassen.
  \item Der Teilnehmer ist dazu verpflichtet regelm"a\ss ig, nach M"oglichkeit
      t"aglich,  seine SchunterNet-EMails zu lesen. Er hat weiterhin daf"ur zu
      sorgen,  dass  er die EMails auch korrekt empfangen kann. Dabei ist es
      unerheblich, ob die EMails direkt vom SchunterNet-Mailserver oder "uber
      eine Weiterleitung bezogen werden.
\end{enumerate}

\Paragraph{Verfahren bei Verst"o"sen gegen die Benutzerordnung}

\begin{enumerate}
  \item Benutzer, die gegen die Benutzerordnung versto"sen, werden von den
      Vertretern des \snev auf den Versto"s hingewiesen.
  \item Bei schweren oder wiederholten Verst"o"sen gegen die Bestimmungen der
      Benutzerordnung wird der betreffende Teilnehmer von der weiteren Nutzung
      ausgeschlossen. Werden Belange des Zusammenlebens im Wohnheim ber"uhrt,
      kann zus"atzlich ein Heimratsverfahren angestrengt werden.
  \item Die Ger"ate und Anlagen werden in funktionsf"ahigem Zustand
      "ubergeben. Durch unsachgem"a"se Behandlung eingetretene Sch"aden hat
      der Nutzer in vollem Umfang zu tragen. Bei Beendigung der Nutzung,
      sp"atestens beim Auszug, wird von Vertretern des \snev der Zustand
      kontrolliert und ein Abnahmeprotokoll erstellt. Der Nutzer bleibt f"ur
      entstehende Sch"aden haftbar, solange er diese Abnahme nicht durchf"uhren
      lassen hat.
  \item Wird durch Verst"o"se zus"atzlicher administrativer Aufwand zur
      Wiederherstellung oder Bewahrung der Funktion und Sicherheit des Systems
      notwendig, so hat der Verursacher die entstehenden Kosten sowie die
      Arbeitsleistung entsprechend den in der Geb"uhrenordnung [Anhang
      \ref{gebuehr}] festgelegten Tarifen zu tragen.
  \item Wer "uber diese Bestimmungen hinaus gegen die Benutzungsordnung f"ur
      das Rechenzentrum der Technischen Universit"at Braunschweig [Anhang
      \ref{RZordnung}], Interessen dritter, nationales oder internationales
      Recht verst"o"st, hat mit Meldung an die zust"andigen Stellen bis hin zur
      Anzeige zu rechnen.
\end{enumerate}

\Paragraph{Haftungsausschluss}
\begin{enumerate}
  \item Ein Anspruch auf ununterbrochene Funktion des Netzes besteht
      nicht. Schadenersatzanspr"uche des Benutzers gegen"uber den Betreibern
      k"onnen nicht geltend gemacht werden.
  \item F"ur Sch"aden an Hardware, Software oder Daten des Benutzers, die
      durch die Teilnahme am Netzbetrieb entstehen, "ubernimmt der Betreiber
      keine Haftung.

\end{enumerate}

\clearpage

\section{Geb"uhrenordnung des SchunterNet e.V.}
\label{gebuehr}

{\small Stand: 1. November 1999}

\begin{enumerate}
\item Eine Anschlussgeb"uhr wird nicht erhoben.

\item Die Nutzungsgeb"uhr (Monatsgeb"uhr) wird dynamisch erhoben. Sie
  wird abh"angig vom Finanzbedarf des Vereins sowie der Anzahl der
  Nutzer angepa"st und sollte einen Betrag von 20~DM (in Worten:
  zwanzig Deutsche Mark) nicht "uberschreiten.
  Derzeit ist die Geb"uhr auf \EUR{10,--} (in Worten: zehn Euro) festgelegt. 

\item Bei der Freischaltung des Zugangs wird f"ur den laufenden Monat
  eine volle Monatsgeb"uhr erhoben. Gleichzeitig ist die Nutzungsgeb"uhr
  f"ur den folgenden Monat im voraus zu entrichten. Zur Absicherung eventueller
  durch den Nutzer verursachter Forderungen Dritter an den Verein, ist eine
  Kaution von \EUR{10,--} zu hinterlegen.

\item Alle Folgezahlungen sind bis 15. des Vormonats f"ur den
  entsprechenden Monat einzuzahlen (jeweils eine Monatsgeb"uhr).

\item Die Zahlung der Nutzungsgeb"uhr erfolgt ausschlie"slich per
  Lastschrifteinzugsverfahren.

\item Bei nicht gedecktem Konto kann die K"undigung fristlos zum Ende
  des laufenden Monats erfolgen.

\item Nach jeder K"undigung betr"agt die Wiederanschlussgeb"uhr drei
  Monatsgeb"uhren.

\item Die Nutzungsgeb"uhren werden in erster Linie zur R"uckzahlung des
  Kredits des Studentenwerkes verwendet. Eventuelle "Ubersch"usse
  werden f"ur Reparaturen und weitere Investitionen genutzt.

\item Wird ein Zweitanschluss beantragt, so sind die Kosten der
  Umr"ustung von Anschlussdose und Patchfeld (2xTP/TP--Einsatz) durch den
  Antragsteller zu tragen.

\item Durch Benutzer verursachter zus"atzlicher Arbeitsaufwand im Sinne \S{}8,
  Absatz 4 der Benutzerordnung [Anhang \ref{nutzerordnung}] wird zu
  einem Stundensatz von \EUR{12,80} in Rechnung gestellt.

\item Sch"aden an vereinseigener Hardware, die durch unsachgem"a"sen Umgang
  fahrl"assig oder vors"atzlich entstanden sind, gehen in vollem Umfang zu
  Lasten des Verursachers.

\item "Anderungen dieser Geb"uhrenordnung sind in den halbj"ahrlichen
  Mitgliederversammlungen m"oglich.
\end{enumerate}

\clearpage

\setcounter{para_nr}{0}
\renewcommand{\thesubsection}{\Roman{subsection}}

\section[RZ-Benutzungsordnung der TU Braunschweig]{Benutzungsordnung f"ur das
  Rechenzentrum der Technischen Universit"at Braunschweig}
\label{RZordnung}

{\small Stand: Dezember 1994}

\subsection{Allgemeines}

\Paragraph{Aufgaben des Rechenzentrums}

Das Rechenzentrum ist eine zentrale Einrichtung der Technischen
Universit"at Braunschweig, der alle Datenverarbeitungsanlagen und
Datenkommunikationsnetze innerhalb der Universit"at zugeordnet sind.
Dem Rechenzentrum obliegen folgende Aufgaben:

\begin{enumerate}
 \item Der Betrieb der Datenverarbeitungsanlagen und des
       Datenkommunikationsnetzes zur Erf"ullung von Aufgaben der
       Universit"at in Forschung, Lehre und Studium sowie zur
       Erledigung von Verwaltungsaufgaben.
 \item die Beratung und Unterst"utzung f"ur die Nutzung der
       Datenverarbeitungsanlagen, des Datenkommunikationsnetzes und der
       Rechnerprogramme,
 \item die Betreuung aller der Hochschule verf"ugbaren
       Datenverarbeitungskapazit"aten und Datenkommunikationsnetze
       sowie die betriebsfachliche Aufsicht "uber alle
       Datenverarbeitungsanlagen der Hochschule,
 \item die Koordination der Beschaffung und Erg"anzung von
       Datenverarbeitungsanlagen, Datenkommunikationsnetzen und
       Rechnerprogrammen.
\end{enumerate}

\Paragraph{Leistungen und Inanspruchnahme des Rechenzentrums}

\renewcommand{\labelenumi}{(\arabic{enumi})}
\renewcommand{\labelenumii}{\arabic{enumii}.}
\begin{enumerate}
  \item Das Leistungsangebot des Rechenzentrums umfa"st insbesondere:
    \begin{enumerate}
      \item Die Bereitstellung von DV--Ger"aten,
      \item die Bereitstellung einer allgemeinen Datenkommunikation,
        insbesondere "uber das Hochschulnetz,
      \item Zugang zu Informationsdiensten,
      \item Dienstleistungen im Zusammenhang mit der DV--Versorgung.
    \end{enumerate}

    Das Leistungsangebot des Rechenzentrums im einzelnen wird in
    Online--Do\-ku\-mentationen, Handb"uchern und Mitteilungen des
    Rechenzentrums gesondert bekanntgegeben.

  \item Die Inanspruchnahme des Rechenzentrums bedarf der Zulassung und erfolgt
    nach Ma"sgabe dieser Benutzungsordnung und ihrer Anlagen:
\end{enumerate}

\begin{itemize}
 \item Richtlinien f"ur das Hochschulnetz (Anlage 1 [Anhang
   \ref{TUBSnet-Richtlinien}]).
 \item Richtlinien zur Kontingentierung der DV--Kapazit"aten der
       zentralen Anlagen (Anlage 2).
 \item Richtlinien zum Betrieb von www--Servern und zur Nutzung von
       www--Diensten (Anlage 3 [Anhang \ref{www-Richtlinien}]).
\end{itemize}

\Paragraph{Nutzungsberechtigte}

Folgende Personen bzw. Institutionen k"onnen auf Antrag die Leistungen
des Rechenzentrums in Anspruch nehmen:

\begin{enumerate}
 \renewcommand{\labelenumi}{\arabic{enumi}.}
 \item \label{mitgl} Die Mitglieder und Angeh"origen der Universit"at, alle
       Fachbereiche, wissenschaftliche Einrichtungen, Betriebseinheiten
       usw., die im Ausstattungsplan, im Organisationsplan oder im
       Haushaltsplan einer nieders"achsischen Hochschule oder einer
       Hochschule au"serhalb des Landes gef"uhrt sind.
 \item \label{wiss} Andere wissenschaftliche Einrichtungen, die ganz oder
       "uberwiegend aus "offentlichen Mitteln finanziert
       werden.
 \item Sonstige Einrichtungen und Personen, die nicht unter
       Ziffer \ref{mitgl} oder Ziffer \ref{wiss} fallen.
\end{enumerate}

\Paragraph{DV--Benutzer und DV--Beauftragte}

Benutzerinnen bzw. Benutzer sind diejenigen Personen, die die Leistungen
des Rechenzentrums unmittelbar in Anspruch nehmen. Die von den
Einrichtungen der Hochschule mit der Abwicklung ihrer
Datenverarbeitungsvorhaben beauftragten Mitarbeiter/-innen hei"sen
DV--Beauftragte.


\subsection{Benutzungserlaubnis}

\Paragraph{Zulassungsverfahren}

\begin{enumerate}
  \item Wer die Leistungen des Rechenzentrums in Anspruch nehmen will, bedarf
    der Zulassung (Benutzungserlaubnis). Mit der Zulassung wird das
    Benutzungsverh"altnis begr"undet. Die Benutzer haben die
    Benutzungsordnung und deren Anlagen zu beachten.

  \item Die Benutzungserlaubnis ist --- einschlie"slich der erforderlichen
    Benutzeridentifikation --- schriftlich zu beantragen. Der Antrag soll auf
    den Vordrucken des Rechenzentrums gestellt werden und hat folgende
    Angaben zu enthalten:

    \begin{enumerate}
      \item Angaben zur beantragenden Person oder Einrichtung,
      \item Angaben, die eine Zuordnung zu Rangstufen der Bearbeitung
        gem"a"s \S{} 8 erm"oglichen,
      \item Angaben zu Art, Umfang und Zweck der beabsichtigten Nutzung,
      \item Abgabe der verlangten Erkl"arungen.
    \end{enumerate}

  \item Nutzungsberechtigte, die die Genehmigung erhalten, eigene Rechner
    bzw. Subnetze am Hochschulnetz anzuschlie"sen und zu betreiben,
    k"onnen Mitgliedern und Angeh"origen der TU Braunschweig unter
    Beachtung der "`Richtlinien f"ur das Hochschulnetz"' [Anhang
    \ref{TUBSnet-Richtlinien}] Zugang zum Hochschulnetz gew"ahren.
\end{enumerate}

\Paragraph{Erlaubniserteilung}

\begin{enumerate}
  \item Die Benutzungserlaubnis, mit der auch die Benutzeridentifikation
    vergeben wird, wird vom Leiter des Rechenzentrums schriftlich erteilt. Sie
    ist auf die beantragte und bewilligte Nutzungsart beschr"ankt. Mit der
    Erlaubniserteilung erfolgt die Einstufung in die jeweilige Rangstufe (\S{} 8)
    und Kostengruppe. Die Erlaubnis kann zeitlich befristet, eingeschr"ankt
    und unter Auflagen und Bedingungen erteilt werden. Die Benutzungserlaubnis
    ist nicht "ubertragbar.

  \item Die Berechtigung zur Nutzung bestimmter Leistungen kann vom
    Rechenzentrum insbesondere mit Bezug auf folgende Gesichtspunkte
    eingeschr"ankt oder versagt werden:

    \begin{itemize}
      \item Rangstufenfolge gem"a"s \S{} 8 (Nutzungspriorit"at),
      \item Vorrang von Arbeiten in Lehre und Forschung der
        Universit"atseinrichtungen,
      \item Zweckbestimmung der betreffenden Ger"ate bzw. Rechnersysteme,
      \item Lizenzbestimmungen,
      \item Wirtschaftlichkeit bzw. Verh"altnism"a"sigkeit der
        Verfahren,
      \item Leistungsverm"ogen und Auslastung der betreffenden
        Ger"ate bzw. Rechnersysteme.
    \end{itemize}

 \item Der/die Nutzungsberechtigte hat wesentliche "Anderungen seiner im
   Antrag gemachten Angaben unverz"uglich dem Rechenzentrum mitzuteilen;
   insbesondere ist er/sie verpflichtet, die Beendigung der Nutzung
   unverz"uglich bekanntzugeben.
\end{enumerate}

\Paragraph{Beendigung des Benutzungsverh"altnisses}

\begin{enumerate}
  \item Die Benutzungserlaubnis erlischt mit Beendigung des
    Benutzungsverh"altnisses:
    \begin{itemize}
      \item Nach Ablauf der erteilten Frist.
      \item Aufgrund einer entsprechenden Mitteilung des Nutzungsberechtigten
        oder der/des DV--Beauftragten.
      \item Sobald der Nutzungsberechtigte aus der TU Braunschweig bzw.\  aus
        derjenigen Einrichtung ausscheidet, die die Benutzungserlaubnis
        beantragt hat.
      \item Durch Ausschluss gem"a"s \S{} 12 (2).
    \end{itemize}

  \item Der Nutzungsberechtigte verpflichtet sich, bei Beendigung des
    Benutzungsverh"altnisses:
    \begin{itemize}
      \item alle ihn betreffenden bzw.\  von ihm genutzten Datenbereiche und
        Adressen freizugeben,
      \item die vom Rechenzentrum zur Verf"ugung gestellten Arbeitsmittel
        zur"uckzugeben,
      \item alle sonstigen Anspr"uche des Rechenzentrums, die aus dem
        Benutzungsverh"altnis entstanden sind, zu erf"ullen.
    \end{itemize}
\end{enumerate}

\Paragraph{Rangstufen}

Die Datenverarbeitungsvorhaben werden nach der Zugeh"origkeit der
sie durchf"uhrenden Nutzungsberechtigten (siehe \S{} 3) in Gruppen
gegliedert, denen Rangstufen zugeordnet sind. Die Rangstufenzuordnung
legt die jeweilige Nutzungspriorit"at fest und richtet sich nach
den vorl"aufigen Grunds"atzen f"ur die Errichtung und den
Betrieb von Hochschulrechenzentren in Niedersachsen --- bekanntgegeben
mit RdErl.\  d. MWK vom 19.9.1978 -- 1053 -- 02 804 -- G"ultL MWK 60/55 ---.

\subsection{Benutzungsregeln}

\Paragraph{Rechte und Pflichten der Benutzer}

\begin{enumerate}
  \item Benutzerinnen und Benutzer sind verpflichtet, diese Benutzungsregeln
    einzuhalten. F"ur die Nutzung des Hochschulnetzes gelten
    zus"atzlich die speziellen, in den "`Richtlinien f"ur das
    Hochschulnetz"' (Anlage 1 [Anhang \ref{TUBSnet-Richtlinien}])
    aufgef"uhrten Regelungen.

  \item Die Nutzungsberechtigten k"onnen diejenigen Leistungen des
    Rechenzentrums in Anspruch nehmen, f"ur die sie eine
    Benutzungserlaubnis haben. Die bereitgestellten Ressourcen, die durch
    Kontingente (gem"a"s \S{} 14) begrenzt sein k"onnen, sind in
    wirtschaftlicher und dem Nutzungszweck angemessener Weise zu nutzen.
    Im "ubrigen haben die Benutzer darauf zu achten, da"s sie
    die Nutzungsm"oglichkeiten anderer nicht in unangemessener Weise
    beeintr"achtigen.

  \item Die Nutzung der DV--Einrichtungen f"ur kommerzielle oder private
    Zwecke ist nur mit schriftlicher Zustimmung des Rechenzentrums und
    nach Festlegung der Entgelte zul"assig.

  \item Der Nutzungsberechtigte hat daf"ur Sorge zu tragen, da"s die
    ihm zugeteilten Benutzeridentifikationen nicht an andere weitergegeben
    oder in sonstiger Weise mi"sbr"auchlich verwendet werden.

  \item Bei der Nutzung von R"aumen bzw. Ger"aten des Rechenzentrums
    sind die Bedienungsanleitungen, allgemeinen Sicherheitsvorschriften
    und die Vorschriften der Hausordnung zu beachten. Beim Umgang mit
    Einrichtungen und Ger"aten des Rechenzentrums ist die gebotene
    Sorgfalt aufzuwenden.

  \item Benutzerinnen und Benutzer d"urfen Software und Dokumentationen,
    die ihnen vom Rechenzentrum direkt oder indirekt zur Verf"ugung
    gestellt werden, nicht ohne Genehmigung des Rechenzentrums kopieren,
    an Dritte weitergeben oder Dritten zug"anglich machen oder an
    anderen Prozessoren als denen verwenden, f"ur die das Rechenzentrum
    die Software bestimmt hat. Im "ubrigen sind die f"ur die
    zur Verf"ugung gestellte Software ma"sgeblichen
    Lizenzbestimmungen einzuhalten.

  \item Es ist untersagt, Manipulationen an der Betriebssystem--Software
    und an Benutzerverzeichnissen vorzunehmen oder Zugriff auf
    Benutzerbereiche auszuf"uhren, f"ur die keine Berechtigung
    vorliegt.

  \item Nach Aufforderung durch das Rechenzentrum ist der Nutzungsberechtigte
    verpflichtet, einen Bericht "uber die Benutzung der Rechenanlagen
    und die dabei gewonnenen Erfahrungen abzugeben.
\end{enumerate}

\Paragraph{Sicherheit des Datenmaterials}

\begin{enumerate}
  \item Das Rechenzentrum sorgt im allgemein "ublichen Rahmen f"ur
    die Sicherung der Daten, die die Benutzer-/innen auf elektronischen
    Datentr"agern des Rechenzentrums speichern.

  \item Das Rechenzentrum bewahrt Medien, die mit Daten von Benutzern
    beschrieben sind, w"ahrend einer festgelegten Frist auf. Die
    innerhalb dieser Frist nicht abgeholten Medien k"onnen vom
    Rechenzentrum vernichtet werden.
\end{enumerate}

\Paragraph{Verarbeitung schutzw"urdiger Daten}

Die Verarbeitung und "Ubertragung von Daten, die schutzbed"urftig
im Sinne der Datenschutzbestimmungen sind, ist nur nach vorheriger R"ucksprache
mit dem Rechenzentrum und nur unter Einhaltung der vorgeschriebenen
Sicherheitsma"snahmen gestattet.

\Paragraph{Ordnungsma"snahmen}

\begin{enumerate}
  \item Verst"o"st ein Nutzungsberechtigter gegen diese
    Benutzungsordnung und deren Anlagen, insbesondere gegen die sich aus
    \S{} 9 ergebenden Pflichten, kann der Leiter des Rechenzentrums die
    Benutzungserlaubnis vor"ubergehend einschr"anken bzw.\  in
    wiederholten oder schwerwiegenden F"allen die Benutzung f"ur
    einen bestimmten Zeitraum untersagen. Der betreffende Benutzer
    muss davon unter Angabe der Gr"unde in Kenntnis gesetzt werden.

  \item In besonders schwerwiegenden F"allen kann dem betreffenden Benutzer
    nach vorheriger Anh"orung und mit Zustimmung der "`Senatskommission
    f"ur die elektronische Datenverarbeitung"' die Benutzungserlaubnis
    entzogen werden. Bei einem schwerwiegenden Versto"s wird der
    Leiter des Rechenzentrums pr"ufen, ob strafrechtliche oder
    zivilrechtliche Schritte einzuleiten sind.

  \item Der Leiter des Rechenzentrums "ubt das Hausrecht aus und ist
    weisungsberechtigt.
\end{enumerate}

\Paragraph{Haftung}

\begin{enumerate}
  \item Die Benutzer haften f"ur die von ihnen schuldhaft verursachten
    Sch"aden sowie f"ur Verluste und Ver"anderungen der Daten
    des Rechenzentrums oder Dritter. Sie stellen das Rechenzentrum (bzw.\ 
    die Universit"at) vor Anspr"uchen Dritter frei, sofern etwaige
    Sch"aden auf Verst"o"se gegen diese Benutzungsordnung,
    insbesondere gegen Lizenzbestimmungen Dritter zur"uckzuf"uhren
    sind.

  \item Das Rechenzentrum haftet f"ur die von seinen Mitarbeiterinnen und
    Mitarbeitern in Aus"ubung ihrer Dienstpflichten vors"atzlich
    oder grob fahrl"assig verursachten Sch"aden. Eine Haftung des
    Rechenzentrums f"ur fehlerhafte Rechenergebnisse, f"ur die
    Zerst"orung von Daten und die Besch"adigung von Datentr"agern
    sowie f"ur Termin"uberschreitungen bei Rechenarbeiten ist ---
    soweit rechtlich zul"assig --- ausgeschlossen.
\end{enumerate}

\subsection{Bewirtschaftung von Betriebsmitteln}

\Paragraph{Kontingentierung}

\begin{enumerate}
  \item Da Rechenzeiten und Betriebsmittel in beschr"anktem Umfang zur
    Verf"ugung stehen, werden sie gegebenenfalls in Form von
    Kontingenten an die Benutzer verteilt. Betriebsmittel im Sinne dieser
    Ordnung sind insbesondere Speicherbereiche, Drucker, "Ubertragungswege
    auf Datenleitungen und Rechner--Arbeitspl"atze.

  \item Die Zuteilung der Kontingente erfolgt gem"a"s einem
    Verteilungsschl"ussel, der die Zugeh"origkeit der
    Nutzungsberechtigten (gem"a"s \S{} 8) sowie den Aufgabenbezug,
    insbesondere den Vorrang von Lehre und Forschung, ber"ucksichtigt.
    Die jeweiligen Kontingente setzen sich aus einem festen Sockelbetrag
    und einem bedarfsbezogenen Anteil zusammen. Die Kontingente werden in
    der Regel jeweils zu Beginn eines Quartals neu berechnet.

  \item Die Regelungen zur Kontingentierung im einzelnen werden durch
    Senatsbeschluss festgelegt.

  \item Die Verteilung der Rechenzeiten der zentralen Rechenanlagen geschieht
    im einzelnen nach Ma"sgabe der "`Richtlinien zur Kontingentierung der
    DV--Kapazit"at der zentralen Anlagen der TU Braunschweig"' (Anlage 2).
\end{enumerate}

\Paragraph{Entgelte und Kostenermittlung}

\begin{enumerate}
  \item F"ur die Inanspruchnahme der Leistungen des Rechenzentrums sind
    je nach Rangstufe folgende Entgelte zu entrichten:

    \begin{tabular}{lll}
      Rangstufen 1, 2&Angeh"orige von Hochschulen des Landes&unentgeltlich\\
      Rangstufe 3&Angeh"orige von Hochschulen&Selbstkosten Land\\
      &au"serhalb des Landes&\\
      Rangstufe 4&Hochschulbedienstete bei Nebent"atigkeit&Marktpreise.\\
      &und sonstige Nutzer&\\
    \end{tabular}

    Die Nutzung von Rangstufe 1 und 2 erfolgt grunds"atzlich unentgeltlich;
    Aufwendungen im Sinne von \S{} 61 Abs. 1 Satz 2 LHO (Landeshaushaltsordnung)
    sind zu erstatten. In den Rangstufen 3 und 4 werden Entgelte erhoben.

  \item F"ur Leistungen, die den im Rechenzentrum "ublichen Rahmen
    "uberschreiten, k"onnen zus"atzliche Entgelte erhoben werden.
    Diese legt das Rechenzentrum fest und teilt sie auf Anfrage mit.

  \item Grundlage f"ur die Bemessung der in Anspruch genommenen Leistungen
    sind die Betriebsunterlagen des Rechenzentrums. An Hand dieser Unterlagen
    ermittelt das Rechenzentrum die zu zahlenden Entgelte. Den
    Nutzungsberechtigten werden regelm"a"sig Nachweise "uber
    die entstandenen Kosten zugestellt und, falls nach der
    Landeshaushaltsordnung Kosten zu erheben sind, diese in Rechnung
    gestellt.

  \item Benutzungsentgelte sind auch dann zu entrichten, wenn Programme
    ergebnislos oder fehlerhaft durchgef"uhrt werden, es sei denn,
    der Fehler ist nachweislich und aufgrund grober Fahrl"assigkeit
    vom Rechenzentrum zu vertreten und das Benutzungsentgelt ist erheblich.
    Bei Inanspruchnahme sonstiger Leistungen des Rechenzentrums gilt
    Satz 1 entsprechend.

  \item Das N"ahere zu den Entgelten einschlie"slich der Sonderleistungen
    und der Kostenermittlung ist in der "`Entgeltordnung"' geregelt. Es gelten
    die jeweils vom Pr"asidenten genehmigten S"atze f"ur die
    Kategorien "`Selbstkosten Land"' und "`Marktpreise"'.
\end{enumerate}

\Paragraph{Inkrafttreten}

Diese Benutzungsordnung tritt am Tage nach ihrer hochschul"offentlichen
Bekanntmachung in Kraft. Gleichzeitig treten fr"uhere
Benutzungsordnungen au"ser Kraft.

\clearpage

\renewcommand{\thesubsection}{\arabic{subsection}}
\renewcommand{\labelenumi}{\arabic{subsection}.\arabic{enumi}}

\section[Richtlinien f"ur das TUBS-Net]{Richtlinien f"ur das Hochschulnetz der
  Technischen Universit"at Braunschweig}
\label{TUBSnet-Richtlinien}

{\small Stand: Dezember 1994}

\subsection{Allgemeines}

\begin{enumerate}
  \item Hochschulnetz im Sinne der "`Benutzungsordnung f"ur das Rechenzentrum"'
    [Anhang \ref{RZordnung}] ist das allgemeine Datenkommunikationsnetz
    der TU Braunschweig. F"ur den Betrieb des Hochschulnetzes gelten die
    nachfolgend aufgef"uhrten Bestimmungen.

  \item Die Verantwortung f"ur den Betrieb des Hochschulnetzes
    als logisches System liegt beim Rechenzentrum, das die dem
    Netzbetrieb (Netzmanagement, Einstellung intelligenter
    Netzkomponenten) zugrundeliegenden Programme betreibt.

  \item Die Verantwortung f"ur die Leitungen und "Ubergabepunkte
    des Hochschulnetzes liegt bei der Abteilung Betriebstechnik
    der TU Braunschweig.

  \item Die verf"ugbaren Netzdienste und Anschlusstypen (Kopplungselemente)
    werden in Publikationen des Rechenzentrums bekanntgemacht.

  \item Der Zugang zum Hochschulnetz ist "uber Rechner des
    Rechenzentrums oder "uber Rechner bzw. Rechnernetze
    (kurz: "`Rechner"') der Nutzungsberechtigten m"oglich,
    f"ur die geeignete Anschl"usse bereitgestellt werden.

  \item Betreiber von nutzereigenen "`Rechnern"' am Hochschulnetz
    ("`Rechnerbetreiber"') k"onnen alle Nutzungsberechtigten
    gem"a"s \S{}3 der Benutzungsordnung [Anhang \ref{RZordnung}] sein.

  \item F"ur jeden an das Hochschulnetz angeschlossenen "`Rechner"'
    gibt es einen "Ubergabepunkt, in der Regel ein
    Kopplungselement, mit dem die betriebstechnische Verantwortung
    f"ur das Netz endet. Die Kommunikation auf der
    Nutzerseite, d.h.\  bis zum "Ubergabepunkt, liegt in der
    Verantwortung des "`Rechnerbetreibers"'.
\end{enumerate}

\subsection{Aufgaben und Pflichten des Rechenzentrums}

\begin{enumerate}
  \item Das Rechenzentrum ist stets um einen sicheren und
    ununterbrochenen Betrieb des Hochschulnetzes bem"uht,
    soweit dies beim Stand der Technik m"oglich ist.
    Ein st"orungsfreier Betrieb kann aber angesichts des
    Entwicklungsstandes und des offenen Charakters der
    derzeitigen Netze nicht garantiert werden.

  \item Das Rechenzentrum vergibt und verwaltet die Netzadressen
    und ber"at die "`Rechnerbetreiber"' in Fragen der Nutzung
    des Hochschulnetzes. Soweit h"ohere Netzdienste zu
    koordinieren sind, "ubernimmt das Rechenzentrum
    diese Aufgabe.

  \item Das Rechenzentrum "ubernimmt keine Verantwortung
    f"ur "uber das Hochschulnetz herangetragene
    Beeintr"achtigungen des Betriebs der angeschlossenen
    "`Rechner"'.

  \item Das Rechenzentrum hat das Recht, Teile des Hochschulnetzes
    zeitweilig stillzulegen, wenn diese Ma"snahme
    erforderlich ist, um bei St"orungen die
    Funktionsf"ahigkeit des "ubrigen Netzes zu erhalten.
\end{enumerate}

\subsection{Aufgaben und Pflichten der Benutzer des Hochschulnetzes}

\begin{enumerate}
  \item Die Benutzung des Hochschulnetzes besteht darin, "uber
    ein an das Hochschulnetz angeschlossenes Rechnersystem Daten
    in das Hochschulnetz zu senden oder daraus zu empfangen.
    Als solche unterliegt sie der "`Benutzungsordnung f"ur das
    Rechenzentrum der TU Braunschweig"'. [Anhang \ref{RZordnung}]
    Benutzerinnen und Benutzer des Hochschulnetzes sind somit neben den im
    Rechenzentrum registrierten Benutzern auch Personen, die als Mitglieder
    und Angeh"orige der TU Braunschweig unter der Verantwortlichkeit eines
    "`Rechnerbetreibers"' t"atig sind. (Siehe Benutzungsordnung \S{}5 [hier:
    Anhang \ref{RZordnung}])

  \item Der Benutzerin oder dem Benutzer ist es untersagt,
    "Anderungen an den "Ubergabepunkten vorzunehmen.
    Soweit sie geeignet sind, den Betrieb des Hochschulnetzes
    zu beeinflussen, sind alle "Anderungen an den
    angeschlossenen Ger"aten und den vorgesehenen
    Betriebsweisen nur in Absprache mit dem Rechenzentrum
    zul"assig.

  \item Der Benutzerin oder dem Benutzer ist es untersagt, ohne
    Absprache mit dem Rechenzentrum Netzadressen zu "andern
    oder neue Netzadressen einzuf"uhren.

  \item Der Datenverkehr einer Benutzerin oder eines Benutzers darf den
    Datenverkehr anderer Benutzer/-innen nicht unangemessen
    beeintr"achtigen, z.B. durch ungezielte und
    "uberm"a"sige Verbreitung von Informationen.

  \item Jedes unbefugte Mitlesen oder Auswerten von Nachrichteninhalten
    sowie die Weitergabe unbeabsichtigt erhaltener Informationen
    ist untersagt.

  \item Die Benutzerin oder der Benutzer ist verpflichtet, dem
    Rechenzentrum unverz"uglich einen erkannten Mi"sbrauch
    des Hochschulnetzes bzw. St"orungen am Netz anzuzeigen.

  \item Im Sinne des Datenschutzgesetzes schutzw"urdige Daten d"urfen nur in
    verschl"usselter Form auf das Hochschulnetz geleitet werden, da die
    Abh"orsicherheit des Netzes nicht gew"ahrleistet werden kann.

  \item Vor "Ubertragung von sehr gro"sen Datenmengen bzw.\  von
    schutzw"urdigen Daten hat der "`Rechnerbetreiber"' Absprachen
    mit dem Rechenzentrum zu treffen.
\end{enumerate}

\subsection{Aufgaben und Pflichten des Rechnerbetreibers}

\begin{enumerate}
  \item Der Anschluss eines "`Rechners"' an das Hochschulnetz sowie
    der Ausbau des Leitungsnetzes m"ussen vom "`Rechnerbetreiber"'
    beim Rechenzentrum beantragt werden. Die Entscheidung "uber
    den Antrag liegt beim Rechenzentrum.

  \item Der "`Rechnerbetreiber"' ist f"ur die ordnungsgem"a"se
    Nutzung des Hochschulnetzes verantwortlich, da nur er den
    Zugang zu seinen "`Rechnern"' und damit auch zum Netz kontrollieren
    kann. Insbesondere hat er darauf zu achten, da"s nur
    dazu berechtigte Nutzerinnen und Nutzer Zugang zum Netz
    bekommen und da"s "uber den "`Rechner"' kein
    unautorisierter Netzzugang erschlossen wird.

  \item Der "`Rechnerbetreiber"' verpflichtet die Nutzer/-innen seiner
    Netzanschl"usse auf Einhaltung der Benutzungsordnung. [Anhang
    \ref{RZordnung}]

  \item Das Hochschulnetz sowie die Datenverbindungen nach au"sen
    d"urfen nur f"ur Aufgaben in Lehre und Forschung
    genutzt werden.

  \item Das Rechenzentrum teilt den "`Rechnerbetreibern"' Adressenbereiche
    f"ur Netzadressen zu, die von diesen z.T. selbst ausgef"ullt werden k"onnen.

  \item Soweit erforderlich werden zwischen Rechenzentrum und
    "`Rechnerbetreiber"' besondere Vereinbarungen getroffen, wie z.B.:
    \begin{itemize}
      \item zum "Ubergabepunkt,
      \item zur Nutzung spezieller Netzdienste,
      \item zu Kosten (z.B. Einrichtungs-, Wartungskosten,
        Verkehrsgeb"uhren).
    \end{itemize}
\end{enumerate}

\clearpage

\section[WWW-Richtlinien an der TU Braunschweig]{Richtlinien zum Betrieb von
  www--Servern und zur Nutzung von www--Diensten an der Technischen
  Universit"at Braunschweig}
\label{www-Richtlinien}

{\small Dieser Text basiert auf einem Entwurf des Arbeitskreises "`Internet"'
  der Hochschule. Die Senatskommission f"ur die elektronische Datenverarbeitung
  (SEDV) hat den Entwurf in dieser Fassung verabschiedet und wird ihn zur
  endg"ultigen Beschlussfassung dem Senat vorlegen.

  Stand: 01.\,2.\,99}

\subsection{Grunds"atze des Betriebs von www--Servern}

\begin{enumerate}
  \item Die Technische Universit"at Braunschweig betreibt einen
        www--Server. Er steht allen Organisationseinheiten, kooperierenden
        Institutionen, Mitgliedern und Angeh"origen der
        Universit"at zur Verf"ugung. Die Organisationseinheiten
        der Universit"at k"onnen eigene www--Server betreiben.

  \item Die Abteilung 52 koordiniert im Auftrage der Hochschulleitung und
        in Absprache mit der Presse- und "Offentlichkeitsarbeit
        das Informationsangebot des zentralen www--Servers. Das
        Rechenzentrum "ubernimmt als Betreiber des www--Servers die
        technische Realisierung und die Betreuung des Systems.
\end{enumerate}

\subsection{Inhalt und Gestaltung von www--Seiten}

\begin{enumerate}
  \item Der Inhalt von www--Seiten muss den Anforderungen der
        DFN--Benutzungsordnung [Anhang \ref{dfn}] und der DV--Nutzungsordnung
        der Hochschule [Anhang \ref{RZordnung}] gen"ugen.

  \item Der Inhalt von www--Seiten muss gesetzliche Bestimmungen einhalten,
        insbesondere das Informations- und Kommunikationsdienste--Gesetz.
        Zu beachten sind u.a.\  der Schutz von personenbezogenen Daten,
        Urheber- und Lizenzrechte, Pers"onlichkeitsrechte und
        Strafgesetze. Im Strafgesetzbuch ist u.a.\  geregelt, da"s
        die Propaganda f"ur verfassungswidrige Organisationen, die
        Verbreitung von rassistischem Gedankengut, Pornographie sowie
        Beleidigungen und Verleumdungen zu unterlassen sind.

  \item Daten "uber Zugriffe auf www--Seiten d"urfen nur gespeichert
        werden, um eine anonyme Zugriffsstatistik zu erstellen oder
        um eine "Uberpr"ufung der Zugriffsberechtigung seitens
        der Domain des zugreifenden Systems zu erm"oglichen.
        Daher ist eine Integration von Tools zur Protokollierung von
        Zugriffen in www--Seiten in der Regel unzul"assig und hat
        zu unterbleiben. --- Erhebung und Speicherung entsprechender Daten
        kann ausnahmsweise zul"assig sein, sofern im Sinne des
        Nieders"achsischen Datenschutzgesetzes
        die Einwilligung der Betroffenen gegeben ist und die
        Hochschulleitung zugestimmt hat.

  \item Die Hochschule ist bem"uht, ihr Informationsangebot so breit
        und attraktiv wie m"oglich zu gestalten und setzt auf die
        engagierte Mitarbeit aller ihrer Angeh"origen. Um diesem
        Anspruch gerecht zu werden, sind die Hochschule und ihre
        Organisationseinheiten berechtigt, ohne R"uckfrage bei
        ihren Mitgliedern und Angeh"origen die folgende Daten
        weltweit zur Verf"ugung zu stellen:
        \begin{itemize}
          \item die offizielle Bezeichnung der Einheit sowie
                \begin{itemize}
                  \item Adresse, Telefon-- und Fax--Nummer, E-Mail--Adresse
                  \item Gesch"aftsverteilungsplan
                  \item angebotene Lehrveranstaltungen (Titel, Nummer,
                        Ort und Zeit, Name des Dozenten/der Dozentin)
                  \item Listen von Gremien und deren Mitglieder
                  \item bereits in anderen Medien ver"offentlichte
                        Artikel und Beitr"age in ungek"urzter
                        oder in einer von der verantwortlichen
                        Organisationseinheit gek"urzten Fassung,
                        soweit kein Copyright verletzt wird
                \end{itemize}
          \item die folgenden Daten ihrer Mitglieder und Angeh"origen:
                \begin{itemize}
                  \item Name, Vorname, Geschlecht
                  \item Zugeh"origkeit zu Organisationseinheiten
                  \item dienstliche Telefon-- und Fax--Nummer, E-Mail--Adresse
                  \item (gestrichen: private Telefonnummer, soweit
                      die Person der \mbox{Residenzpflicht\hspace{-11cm}
                      \rule[1mm]{10.85cm}{0.4pt}} unterliegt\hspace{-1.85cm}
                    \rule[1mm]{1.7cm}{0.4pt})
                  \item dienstliche Funktionen und Aufgaben
                  \item angebotene Lehrveranstaltungen
                \end{itemize}
        \end{itemize}

  \item www--Datenbereiche d"urfen in der Regel nicht Dritten (Personen
        bzw. Organisationen) zur Nutzung "uberlassen werden. In
        Ausnahmef"allen kann mit Genehmigung der Hochschulleitung
        nicht gewinnorientierten "offentlichen Einrichtungen im
        Rahmen der Amtshilfe eine Mitnutzung des www--Servers gew"ahrt
        werden.

  \item www--Seiten d"urfen in der Regel nicht kommerziell genutzt
        werden. Erlaubt ist dagegen den Organisationseinheiten der
        Hochschule die Nennung von F"orderern und Sponsoren samt
        Firmen-- und Produkt--Logos auf je einer eigens daf"ur
        eingerichteten Seite. Derartige Seiten sind hochschulweit
        einheitlich zu gestalten und entsprechend zu kennzeichnen.
        Entsprechende Vereinbarungen mit F"orderern sind der
        Hochschulleitung bekanntzugeben.

  \item Die Gestaltung der www--Seiten sollte sich an den allgemein
        anerkannten Gestaltungsregeln orientieren. Zu diesen
        z"ahlen insbesondere die diesbez"uglichen Gestaltungsempfehlungen
        der interuniversit"aren Arbeitsgruppe. Es ist darauf zu
        achten, da"s das Erscheinungsbild der Universit"at
        im Internet dem in anderen Medien vergleichbar ist. F"ur
        Logos und Signets der Hochschule sind nur die offiziellen
        Versionen (\url{http://www.tu-bs.de/pressestelle/icons/index.html})
        zu verwenden. Abteilung 52 empfiehlt Gestaltungselemente,
        die in den www--Seiten der Hochschulverwaltung verwendet werden.
\end{enumerate}

\subsection{Verantwortlichkeiten des Informationsanbietenden}

\begin{enumerate}
  \item Die f"ur das jeweilige Informationsangebot verantwortliche
        Institution bzw. Person (Universit"atsverwaltung,
        Rechenzentrum, Fachbereich, Institut, \dots\  oder Einzelperson) ist
        f"ur den Inhalt der von ihr bereitgestellten www--Seite,
        ihre Pflege und die Herstellung von Verweisen auf andere
        www--Seiten verantwortlich. Die Verantwortlichkeit erstreckt
        sich auch auf die Einhaltung gesetzlicher Vorschriften.

  \item Die Verantwortung f"ur den Inhalt einer www--Seite
        umfa"st in eingeschr"ankter Weise auch
        Hypertext--Referenzen auf andere Dokumente. Letztere sind
        gelegentlich zu "uberpr"ufen, ob sie ihrerseits
        den gesetzlichen Anforderungen gen"ugen. Ist das erkennbar
        nicht der Fall, muss eine betreffende Referenz entfernt
        oder auf die rechtliche Fragw"urdigkeit des betreffenden
        Dokumentes hingewiesen werden.

  \item Erg"anzend zum www--Angebot der Universit"at bzw.\  ihrer
        Organisationseinheiten k"onnen Mitglieder und Angeh"orige
        der Universit"at, die "uber eine Nutzungsberechtigung
        des RZ oder einer anderen Einrichtung der Universit"at
        verf"ugen, im Rahmen der disponiblen Ressourcen
        pers"onliche www--Seiten
        anbieten, auf denen auch private Themen behandelt werden
        k"onnen. --- Der "Ubergang zu den pers"onlichen
        www--Seiten ist deutlich zu kennzeichnen.

  \item Auf jeder www--Seite ist die f"ur die Bereitstellung der
        Information verantwortliche Organisationseinheit
        einschlie"slich Bearbeiter bzw. Einzelperson sowie das
        Datum der Erstellung bzw. Modifikation zu nennen. Es soll ein
        Link auf eine E-Mail--Adresse zur Verf"ugung gestellt werden,
        "uber die weitere Ausk"unfte bzw. Informationen zur
        Seite eingeholt werden k"onnen. Bei hierarchisch
        nachgegliederten Seiten k"onnen diese Angaben entfallen,
        sofern ein eindeutiger Zusammenhang zwischen den Seiten besteht.
\end{enumerate}

\subsection{Verst"o"se gegen die Regelung des www--Angebotes}

\begin{enumerate}
  \item www--Seiten, deren Inhalte offensichtlich gegen diese Ordnung,
        gegen vorrangige Ordnungen und Regeln oder gegen geltendes
        Recht versto"sen, sind vom Betreiber des jeweiligen
        www--Servers unverz"uglich zu l"oschen.

  \item www--Seiten, aus denen nicht unmittelbar zu entnehmen ist,
        wer f"ur sie verantwortlich ist, k"onnen gel"oscht
        werden.

  \item Ist fraglich, ob der Inhalt einer www--Seite im Sinne des
        ersten Absatzes zu beanstanden ist, informiert der Betreiber
        des www--Servers die jeweilige Anbieterin bzw.\  den jeweiligen
        Anbieter "uber die Beanstandung und bittet um Abhilfe.
        Kommt sie/er diesem Wunsch nicht nach und kann sie/er auch
        nicht nachvollziehbar begr"unden, wieso die beanstandete
        Seite unverzichtbar ist (z. B. f"ur Zwecke von Forschung
        und Lehre), kann die Seite vom Server--Betreiber gesperrt oder
        gel"oscht werden. --- In Zweifelsf"allen hat die
        Hochschulleitung gem"a"s der Rechtslage zu
        entscheiden.

  \item Der Server--Betreiber ist nicht verpflichtet, eine
        Routinedurchsicht der www--Seiten auf seinem Server
        durchzuf"uhren. Erst bei positiver Kenntnis eines
        Versto"ses gegen diese Ordnung wird der
        Server--Betreiber t"atig und auf eine Abstellung
        hinwirken.
\end{enumerate}

\clearpage

\section[DFN-Benutzungsordnung]{Benutzungsordnung f"ur das Zusammenwirken der Anwender der DFN-Kommunikationsdienste}
\label{dfn}

{\small -- vom Vorstand beschlossen am 16.05.1994 und ge"andert am 09.08.2001 --

Stand: 30.\,05.\,2009}

Ziel der Benutzerordnung ist es, die Zusammenarbeit der Anwender
untereinander zu regeln. Um dieses Ziel zu erreichen, werden im folgenden
eine Reihe von unterst"utzenden organisatorischen Ma"snahmen durch die
nutzenden Einrichtungen gefordert und Verhaltensregeln f"ur einen sinnvollen
Umgang mit den Netzressourcen und zur Vermeidung mi"sbr"auchlicher Nutzung
aufgestellt.

Die Benutzerordnung richtet sich in erster Linie an Personen, die f"ur die
Bereitstellung von Kommunikationsdiensten in den Mitgliedseinrichtungen des
DFN--Vereins verantwortlich sind. Es wird erwartet, da"s jede Einrichtung ihre
Endnutzer von dieser Benutzerordnung in Kenntnis setzt. Dar"uber hinaus wird
empfohlen, f"ur die lokal angebotenen Kommunikationsdienste eine eigene
Benutzerordnung zu erstellen, die mit den in diesem Dokument aufgestellten
Richtlinien in Einklang steht oder auf sie verweist.

Das Einhalten dieser Ordnung liegt im gemeinsamen Interesse aller
Beteiligten, da die Verschwendung von Netzressourcen oder deren Mi"sbrauch zu
einer Erh"ohung der Nutzungsentgelte und zu Unregelm"a"sigkeiten bei der
Nutzung der Dienste f"uhren k"onnte.

\subsection{Geltungsbereich}

Die Benutzerordnung bezieht sich auf die DFN--Dienste, die auf der Grundlage
des Wissenschaftsnetzes (WiN) bereitgestellt werden und dazu dienen, den
nutzenden Einrichtungen eine leistungsf"ahige und st"orungsfreie
Kommunikationsinfrastruktur bereitzustellen.

Zum einen handelt es sich dabei um das WiN mit "Uberg"angen zu anderen Netzen,
die f"ur die Kommunikation zur Verf"ugung gestellt werden, zum anderen um die
Infrastruktur f"ur elektronische Post (z. B. Gateways und Relays) und um
Informationsdienste.

\subsection{Anforderungen an die nutzenden Einrichtungen}

Jede am DFN beteiligte Einrichtung tr"agt Sorge f"ur die Wahrnehmung der
Aufgaben des Netzadministrators, des Postmasters und der Verantwortlichen
f"ur Anwendungen sowie f"ur Beratung und Ausbildung. Die Aufgaben m"ussen nicht
notwendigerweise von verschiedenen Personen in der Einrichtung erbracht
werden. Je nach Gr"o"se der Einrichtung wird eine Person mehr als eine der
beschriebenen Aufgaben wahrnehmen. Es ist jedoch erforderlich, da"s jede
Einrichtung die f"ur die genannten Funktionen verantwortlichen Personen mit
den Aufgaben betraut.

Die mit der Wahrnehmung der Funktionen betrauten Personen sollen
verpflichtet werden, den im DFN--Verein abgesprochenen Routing--Strategien (z.
B. IP--Routing, Mail--Routing) zu folgen.

\subsubsection{Netzadministratorfunktion}

Der DFN--Verein empfiehlt, dem "ortlichen Netzadministrator folgende Aufgaben
zu "ubertragen: Der Netzadministrator sorgt f"ur

\begin{itemize}
  \item die Sicherung und Sicherheit des Netzzugangs,
  \item die Funktionsf"ahigkeit der Untervermittlung,
  \item die Netzverwaltung (Routerkonfiguration und --management,
    IP--Host--Adress-- vergabe),
  \item die Domainverwaltung (Betrieb des Nameservers, Verwaltung der
    Zonendaten und Domain--Namensvergabe),
  \item die Strukturierung der Datenfl"usse,
  \item die Fehlererkennung, Fehlermeldung und ggf. Fehlerbehebung,
  \item die Sicherstellung ununterbrochener Betriebsbereitschaft,
  \item den Kontakt zum DFN--Verein zur Sicherstellung des st"orungsfreien
    WiN--Zugangs.
\end{itemize}

\subsubsection{Postmasterfunktion}

Zum reibungslosen Ablauf des Maildienstes soll ein Postmaster benannt
werden, der folgende Aufgaben wahrnimmt:

\begin{itemize}
  \item Bereitstellen der Maildienste auf lokaler Ebene,
  \item Pflege der Adre"stabellen,
  \item Anlaufstelle bei Mailproblemen f"ur Endnutzer sowie f"ur die Betreiber
    von Gateway- und Relaydiensten.
\end{itemize}

\subsubsection{Funktion eines Verantwortlichen f"ur Anwendungen}

Ein Verantwortlicher f"ur Anwendungen soll benannt werden f"ur folgende
Aufgaben:

\begin{itemize}
  \item Pflege der angebotenen Services (Mailserver, Newsserver, FTP--Server),
  \item Pflege weiterer Kommunikationsdienste,
  \item Fehlermanagement.
\end{itemize}

\subsubsection{Beratungs- und Schulungsfunktion}

Die Aus"ubung dieser Funktion ist notwendig, um Fehlbedienungen durch die
Endnutzer zu vermeiden. Sie setzt sich aus folgenden Aufgaben zusammen:

\begin{itemize}
  \item Bereitstellen einer telefonischen Beratungsstelle w"ahrend der
    Arbeitszeit,
  \item Bereitstellen von Informations- und Schulungsmaterial,
  \item Aufkl"arung "uber Auswirkungen von Fehlverhalten bei den Endnutzern.
\end{itemize}

\subsection{Mi"sbrauch}

\subsubsection{Mi"sbr"auchliche Nutzung}

Mi"sbr"auchlich ist die Nutzung der DFN--Dienste, wenn das Verhalten der
Benutzer gegen einschl"agige Schutzvorschriften (u.a.\  Strafgesetz,
Jugendschutzgesetz, Datenschutzrecht) verst"o"st.

Aufgrund ihrer Fachkunde ist bei den Benutzern der Kommunikationsdienste die
jeweilige, insbesondere strafrechtliche Relevanz etwa der
Computer--Kriminalit"at, des Vertriebs pornographischer Bilder und Schriften
oder des Diebstahls, der Ver"anderung oder sonstige Manipulation von bzw.\  an
Daten und Programmen als bekannt vorauszusetzen. Diese Fachkenntnis bezieht
sich auch auf die Sensibilit"at der "Ubertragung von Daten, die geeignet sind,
das Pers"onlichkeitsrecht anderer und/oder deren Privatsph"are zu
beeintr"achtigen oder bestehende Urheberrechte bzw.\  auf diesen gr"undende
Lizenzen zu verletzen.

Als mi"sbr"auchlich ist auch eine Nutzung zu bezeichnen, die folgende, nicht
abschlie"send aufgef"uhrte Sachverhaltskonstellationen erf"ullt:

\begin{itemize}
  \item unberechtigter Zugriff zu Daten und Programmen, d.h.\  mangels
    Zustimmung unberechtigter Zugriff auf Informationen und Ressourcen
    anderer verf"ugungsbefugter Nutzer
  \item Vernichtung von Daten und Programmen, d.h.\  Verf"alschung und/oder
    Vernichtung von Informationen anderer Nutzer -- insbesondere auch durch
    die "`Infizierung"' mit Computerviren
  \item Netzbehinderung, d.h.\  Behinderungen und/oder St"orungen des
    Netzbetriebes oder anderer netzteilnehmender Nutzer, z. B. durch
    \begin{itemize}
      \item ungesichertes Experimentieren im Netz, etwa durch Versuche zum
        "`Knacken"' von Pa"sw"ortern,
      \item nichtangek"undigte und/oder unbegr"undete massive Belastung des
        Netzes zum Nachteil anderer Nutzer oder Dritter.
    \end{itemize}
\end{itemize}

\subsubsection{Empfehlungen an die nutzenden Einrichtungen zur Verhinderung des
Mi"sbrauchs}

Beim Mi"sbrauch der DFN--Dienste kann man grob unterscheiden zwischen
Mi"sbrauch aus Unkenntnis, fahrl"assigem und vors"atzlichem Mi"sbrauch. Je nach
Art des Mi"sbrauchs sind unterschiedliche Aktivit"aten zu seiner Verhinderung
gefragt. Sie reichen von der Aufkl"arung der Nutzer, "uber erh"ohte technische
Sicherheitsma"snahmen bis hin zur Androhung von Nutzungsausschluss und Haftung
f"ur schuldhaft verursachte Sch"aden.

Voraussetzung f"ur die Aufkl"arung von Mi"sbr"auchen ist, da"s die Personen,
denen Zugang zum DFN gew"ahrt wird, namentlich autorisiert sind. Die
Einrichtung, die Netzzugang gew"ahrt, darf daher nat"urlichen Personen den
Zugang nur erm"oglichen, wenn die Personen eine Berechtigung zur Nutzung
haben.

Durch die Wahrnehmung der geforderten Schulungs- und Beratungsfunktion und
durch Aufkl"arungsarbeit "uber Auswirkungen von falschem Nutzungsverhalten auf
andere Nutzer kann dem Mi"sbrauch aus Unkenntnis und dem fahrl"assigen
Mi"sbrauch entgegengewirkt werden. Dazu geh"ort insbesondere, die Endnutzer
zur vertraulichen Behandlung aller Pa"sworte, die f"ur den Zugang zu den
Kommunikationsdiensten ben"otigt werden, zu verpflichten und sie dazu
anzuhalten, ihre Pa"sworte so zu w"ahlen, da"s sie nicht durch einfache
Crackprogramme entschl"usselt werden k"onnen.

Dar"uber hinaus sollten die Betreiber von Kommunikationsdiensten in
zumutbarem Umfang Verfahren bereitstellen, die den pers"onlichen Charakter
und die Vertraulichkeit der auf elektronischem Wege ausgetauschten
Nachrichten oder sensitiven Daten wahren und sch"utzen. Je nach
Sicherheitsrelevanz der Daten wird folgendes empfohlen:

\begin{itemize}
  \item Einsetzen der vom Hersteller gelieferten Sicherheitsmechanismen (z. B.
    Pa"swortschutz),
  \item Anwendung topologischer Ma"snahmen (Abtrennen sicherheitsrelevanter
    Systeme durch Router und Bridges),
  \item Netz"uberwachung (z. B. Protokollierung von Zugriffen),
  \item Einhalten von Sicherheitsklassen (s. "`Kriterien f"ur die Bewertung der
    Sicherheit von Systemen der Informationstechnik"' (ITSEC), Luxemburg
    1991).
\end{itemize}

In den F"allen, wo Nutzern der uneingeschr"ankte Zugang zu bestimmten
Datenbest"anden gew"ahrt wird, ist durch geeignete Ma"snahmen daf"ur zu sorgen,
da"s die Nutzer "uber diesen Weg nicht den unautorisierten Zugang zu
weiteren, nicht--"offentlichen Datenbest"anden erhalten k"onnen. Die Betreiber
sind dar"uber hinaus gehalten, den DFN--Verein beim Aufsp"uren und Verhindern
unzul"assiger Nutzung in zumutbarem Umfang zu unterst"utzen.

Zus"atzlich soll der Endnutzer durch lokale Regelwerke auf den zul"assigen
Gebrauch der Kommunikationsdienste und die Auswirkungen von Fehlverhalten
hingewiesen bzw.\  vor Mi"sbrauch gewarnt werden.

\subsection{Konsequenzen bei Verst"o"sen}

Die das Deutsche Forschungsnetz nutzenden Einrichtungen sind verpflichtet,
ihre Endnutzer mit der Benutzungsordnung und den f"ur sie relevanten Inhalten
der Vertr"age mit dem DFN--Verein vertraut zu machen.

Bei Verst"o"sen gegen die Nutzungsregelungen sind die nutzenden Einrichtungen
gehalten, den Mi"sbrauch unverz"uglich abzustellen und sich untereinander zu
informieren.

Sollte es zur Wahrung der Interessen aller Einrichtungen, die die
Kommunikationsdienste des DFN--Vereins nutzen, erforderlich sein, ist der
DFN--Verein frei, aufgrund der unzul"assigen Nutzung einzelne Personen oder
Einrichtungen von der Nutzung der angebotenen Dienste oder Teilen davon
auszuschlie"sen. In besonders schwerwiegenden F"allen, bei denen die
unzul"assige Nutzung eine Verletzung von geltendem Recht darstellt, k"onnen
zivil- oder strafrechtliche Schritte eingeleitet werden.

\clearpage

\section{Glossar}

\begin{description}
  \item[Account] die Zugangsberechtigung zu einem Computer oder
    Computersystem. Zum Account geh"oren Nutzerkennzeichen - auch login genannt
    - und Pa"swort.
  \item[Administrator] kurz Admin: Jemand, der sich um die Technik k"ummert,
    Soft- und Hardware einrichtet, Fehler behebt
  \item[Archie] weltweites Datenbanksystem mit Dateiverzeichnissen von "uber
    1000 ftp--Servern
  \item[Authentifizierung] Vorgang des Nachweises einer bestimmten
    Identit"at. Mit Hilfe eines Geheimnisses (z.B. Pa"swort) oder typischen
    Merkmales (z.B. Stimmabdruck) "uberzeugt man ein System von der
    vorgegebenen Identit"at
  \item[Client] Programm oder Rechner, welcher den Dienst eines Servers in
    Anspruch nimmt
  \item[DNS] \textbf{D}omain \textbf{N}ame \textbf{S}ervice, Umsetzung von
    IP--Adressen auf Host- und Domainnamen
  \item[E-Mail] Versenden, Empfangen, Lagern und Katalogisieren von
    Nachrichten "uber Netze mittels \textbf{S}imple \textbf{M}ail
    \textbf{T}ransfer \textbf{P}rotocol unter TCP/IP 
  \item[Firewall] Ein Rechner der zwischen zwei Netzen steht und zwischen
    diesen die hin- und herlaufenden Daten filtert. Zweck ist, Hackern das
    Einbrechen in die Rechner "`auf der anderen Seite"' nicht zu einfach zu
    machen, vergleichbar mit einer Feuerschutzwand - daher auch der Name
  \item[FTP] Kopieren von (gro"sen) Dateien zwischen Rechnern mittels 
      \textbf{F}ile \textbf{T}ransfer \textbf{P}rotocol auf TCP/IP
  \item[Homepage] Startseite eines Informationsangebots im World Wide Web
  \item[Homeverzeichnis] Verzeichnis, in dem der Nutzer seine eigenen Daten
    ablegt. Auf UNIX mit \verb#~loginname# erreichbar
  \item[IRC] \textbf{I}nternet \textbf{R}elay \textbf{C}hat, weltweites
    Mehrbenutzer-Kommunikationssystem, Server verwalten viele
    Kommunikationskan"ale, Benutzer unterhalten sich auf den Kan"alen in
    Gruppen oder individuell
  \item[MAC--Adresse] Abk"urzung f"ur \textbf{M}edia \textbf{A}ccess
    \textbf{C}ontrol address, eine weltweit eindeutige Kennzeichnung aller
    Ger"ate innerhalb eines Netzwerkes.
  \item[NFS] \textbf{N}etwork \textbf{F}ile \textbf{S}ystem, das von Sun
    eingef"uhrte Protokoll zur gemeinsamen Nutzung von Dateisystemen im
    Netzwerk.
  \item[Protokoll] Vorschrift f"ur die Kommunikationsabfolge mehrerer
    Teilnehmer. Dient zum Datenaustausch oder zur Steuerungs"ubergabe.
    Protokolle sind oft genormt.
  \item[Proxy] Service, der Objektdaten verschiedener Netzdienste wie HTTP,
    FTP, GOPHER oder News zwischenspeichert und entsprechende Anfragen aus
    seinem lokalen Datenbestand schnellstens bedient (caching) oder
    weiterreicht (proxying). Dadurch werden die Zugriffszeiten auf schon im
    Cache befindliche Daten enorm verbessert und die "Ubertragungswege aus dem
    Internet entlastet.
  \item[RfC] \textbf{R}equest \textbf{f}or \textbf{C}omment sind die
    Dokumente, in denen die Standards des Internet definiert sind.
  \item[Server] Programm oder Rechner, welcher eine bestimmte Serviceleistung
    (Dienst) zur Verf"ugung stellt. (Compute-, File-, WWW-, Mail-Server, etc.)
  \item[SMB] \textbf{S}erver \textbf{M}essage \textbf{B}lock, das insbesondere
    von Windows verwendete Protokoll, um Dateisysteme, Drucker etc. gemeinsam
    im Netzwerk zu nutzen sowie um Listen der verf"ugbaren Ressourcen zu
    erstellen und auszutauschen.
  \item[Subnetz] Eine Menge an Rechnern mit gleichem Netzteil der IP-Adresse
  \item[telnet] Internet-Protokoll f"ur einen Dialog auf einem anderen
    (UNIX-)Rechner, Einloggen auf irgendeinem "offentlichen UNIX-Host
  \item[UNIX] sehr bew"ahrtes Betriebssystem, welches auf Gro"srechnern und
    Workstations eingesetzt wird. Entwickelt von AT\&T sowie gro"sen
    amerikanischen Universit"aten, 30 Jahre alt
  \item[Usenet News] weltweite "offentliche Diskussionen ("`schwarze Bretter"')
    von Teilnehmern zu beliebigen Themen mittels \textbf{N}etwork
    \textbf{N}ews \textbf{T}ransfer \textbf{P}rotocol unter TCP/IP, viele
    tausend Diskussionsgruppen (newsgroups) weltweit, entstanden 1979, lokale
    Administrationen f"ur spezifische Diskussionsgruppen mit regionaler
    Bedeutung
  \item[WWW] \textbf{W}orld \textbf{W}ide \textbf{W}eb ("`Surfen"' im
    Internet), das umfassende und f"uhrende Informationssystem, www ist
    multimedial, d.h.\  es integriert Schrift, Bild, Ton, bewegte Bilder,
    Video-Clip, Radio/TV Live, wenn entsprechende Multimedia-Tools installiert
    sind
\end{description}

\end{appendix}

\end{document}
