\section{Informationsdienste-Ordnung der
Technischen Universität Braunschweig}
\label{info-richtlinien}
\small{(gültig ab 15.07.2000)}
\subsection*{Regelungsinhalte:}
\renewcommand{\labelenumi}{(\theenumi)} 
%\renewcommand{\labelenumii}{(} 
\begin{itemize}
\item[§ 1]	Definitionen
\item[§ 2]	Grundsätze des Betriebs von Informations-Servern
\item[§ 3]	Inhalt und Gestaltung von Informationsangeboten
\item[§ 4]	Verantwortlichkeiten der Informationsanbietenden
\item[§ 5]	Verstöße gegen Vorschriften
\item[§ 6]	Haftung der oder des Informationsanbietenden
\item[§ 7]	Haftung der Universität
\item[§ 8]	In-Kraft-Treten
\end{itemize}

\subsection*{§ 1 Definitionen}
\begin{enumerate}
\item Diese Ordnung regelt die Bereitstellung von Informationsdiensten und öffentlichen Informationsangeboten durch Organisationseinheiten, Mitglieder und Angehörige der Technischen Universität Braunschweig. 
\item Informationsdienste sind technische Einrichtungen, mit denen Informationsangebote in elektronischer Form angeboten und abgerufen werden können. Dokumente sind dabei Informationsangebote in jeglicher Form, bestehend aus Texten, Bildern, Grafiken, Programmen, akustischen Darstellungen, Videosequenzen oder anderen multimedialen Gestaltungen.
\item Zu den Informationsdiensten gehören z. B.: 
www (World Wide Web) 
ftp (Download von Software und Dokumenten) 
E-Mail (elektronische Post) 
usenet-News (elektronische Diskussionsforen) 
irc (Internet Relay Chat) 
Distributions-Channels (Vertrieb von Informationen in Push-Technik) 
Directory Services (erweiterte Adressenverwaltung) 
Streaming-Server (Archive für Video-Dokumente)
\end{enumerate}
\subsection*{§ 2 Grundsätze des Betriebs von Informations-Servern}
\begin{enumerate}
\item 
 Die Technische Universität Braunschweig betreibt zentrale Informationsdienste-Server. Diese stehen allen Organisationseinheiten, Mitgliedern und Angehörigen der Universität - kooperierenden Institutionen, soweit dies vertraglich geregelt ist - zur Verfügung.
\item Die Hochschulleitung koordiniert das Informationsangebot der zentralen Informationsdienste-Server. Das Rechenzentrum übernimmt als Betreiber deren technische Realisierung und die Betreuung der Systeme. 
\item Der Betrieb weiterer, dezentraler Informations-Server bedarf der Zustimmung der Universität. Die Betreiber der dezentralen Informations-Server übernehmen die technische Realisierung, die Betreuung ihrer zugelassenen Systeme sowie die Verantwortung für den Inhalt, soweit Server-Betreibern diese nach den einschlägigen gesetzlichen Bestimmungen obliegt. 
\item Datenbereiche auf Informations-Servern dürfen in der Regel nicht Dritten (Personen bzw. Organisationen außerhalb der Universität) zur Nutzung überlassen werden. In Ausnahmefällen kann mit Genehmigung der Hochschulleitung nicht gewinnorientierten öffentlichen Einrichtungen im Rahmen der Amtshilfe eine Mitnutzung der Informations-Server gewährt werden.
\item Daten über Zugriffe auf Dokumente dürfen nur entsprechend den datenschutzrechtlichen Bestimmungen gespeichert werden, um eine anonyme Zugriffsstatistik zu erstellen.
Besteht der Verdacht, dass bei der Nutzung von Informationsdiensten der Hochschule Straftaten begangen wurden oder werden, so ist eine über Satz 1 hinausgehende Aufzeichnung und Speicherung von Daten (und Dateien) zur Beweissicherung zulässig.
\end{enumerate}
\subsection*{§ 3 Inhalt und Gestaltung von Informationsangeboten}
%\section*{}
\begin{enumerate}
\item  Der Inhalt von Informationsangeboten muss den Anforderungen der
  DFN-Benutzungsordnung und der DV-Nutzungsordnung der Universität
  genügen. Die gesetzlichen Bestimmungen, insbesondere das
  Informations- und Kommunikationsdienstegesetz, der Schutz von
  personenbezogenen Daten sowie die Urheber-, Lizenz- und
  Persönlichkeitsrechte sind zu beachten. (Propaganda für
  verfassungswidrige Organisationen, die Verbreitung von rassistischem
  Gedankengut, Pornographie sowie Beleidigungen, Verleumdungen, das
  Ausspähen von Daten, Datenveränderungen, Computersabotage und
  Computerbetrug stellen in der Regel Straftatbestände dar.)
\item Die Universität gestaltet ihr Informationsangebot so breit und attraktiv wie möglich. Sie und ihre Organisationseinheiten sind berechtigt, folgende Daten ihrer Bediensteten zu veröffentlichen (siehe auch Runderlass d. StK, d. MI u.d. übr. Min. v. 28.05.2001 -44.22-30800/5- veröffentlicht in: Nds. Ministerialblatt,15. Jahrgang, Nummer 25, Seite 571-572): 
Forschungsergebnisse unter Nennung der Autorinnen und Autoren sowie der Forschungseinrichtung (§ 27 NHG). 
Ankündigungen und Berichte von Tagungen mit Namen der Referentinnen und Referenten und Kontaktadressen. 
Namen, Kontaktadressen (einschließlich E-Mail-Adresse, Telefon- und Fax-Nummer) und Forschungsgebiet der unmittelbar in Forschung und Lehre tätigen Bediensteten. 
Sprechzeiten der lehrenden Bediensteten sowie Bezeichnung, Ort und Zeit der Lehrveranstaltungen. 
Private Kontaktadressen nur, wenn die vorgenannten Bediensteten sonst dienstlich (z. B. über das Sekretariat) nicht erreichbar sind. 
Name, Vorname, Telefonnummer, Fax-Nummer, E-Mail-Adresse, Einrichtung / Abteilung von Hochschulmitgliedern.
Der Zugriff auf diese Daten ist jedoch beschränkt auf die Domäne tu-bs.de. 


\item Weitere Angaben dürfen nur mit schriftlich erklärter Einwilligung der Betroffenen veröffentlicht werden. Die betroffenen Bediensteten sind von der Veröffentlichung rechtzeitig in Kenntnis zu setzen. Wenn die Betroffenen wegen überwiegender schutzwürdiger Belange der Veröffentlichung widersprechen, hat sie zu unterbleiben.
\item Sofern Daten von Angehörigen (z. B. Lehrbeauftragten, Privatdozentinnen und Privatdozenten, außerplanmäßigen Professorinnen und Professoren) veröffentlicht werden sollen, ist dies besonders zu vereinbaren.
\item Name, Kontakt- und E-Mail-Adresse von Studierenden und von Bediensteten, die nicht unter Abs. 2 Ziffer 3 fallen, werden nur nach deren vorheriger Zustimmung veröffentlicht.
\item Die Gestaltung der Informationsangebote sollte sich an Gestaltungsrichtlinien orientieren, die die Universitätsleitung zur Verfügung stellt. Als Logo und Signet der Universität ist nur die offizielle Version zu verwenden. Für bestimmte Bereiche (allgemeine TU-Seiten, Fachbereiche) werden Gestaltungselemente angeboten.
\item Jedes Informationsangebot soll Angaben über dessen Urheber enthalten (verantwortliche Organisationseinheit, Bearbeiter bzw. Einzelperson, Datum der Erstellung bzw. Modifikation).
\item Für die Veröffentlichung von Forschungsarbeiten gelten die
  gleichen Sorgfaltspflichten, wie für die Veröffentlichung in
  gedruckter Form.
\item Informationsdienste dürfen in der Regel nicht kommerziell genutzt werden. Den Organisationseinheiten der Universität ist die Nennung des Namens von Förderern und Sponsoren samt Firmenlogos gestattet. Entsprechende Vereinbarungen mit Förderern sind der Hochschulleitung bekannt zu geben.
\end{enumerate}
\subsection*{§ 4 Verantwortlichkeiten der Informationsanbietenden}
\begin{enumerate}
\item Die für das jeweilige Informationsangebot Zuständigen - im Folgenden Informationsanbietende genannt - sind unter Beachtung von § 2 Absatz 3 für den Inhalt der von ihnen bereitgestellten Informationsangebote, ihre Pflege und die Herstellung von Verweisen verantwortlich.
\item Sofern ergänzend zu den Informationsangeboten der Universität bzw. ihrer Organisationseinheiten, Mitarbeiter und Angehörige der Universität persönliche Dokumente ins Netz stellen, ist der Übergang von den offiziellen Informationsangeboten zu den persönlichen Dokumenten deutlich zu kennzeichnen. Für persönliche Dokumente ist deren Anbieter selbst verantwortlich.
\item Die Verantwortlichkeit für den Inhalt eines Informationsangebotes umfasst in eingeschränkter Weise auch Links auf andere Dokumente. Letztere sind gelegentlich zu überprüfen, ob sie ihrerseits den gesetzlichen Anforderungen genügen. Ist das erkennbar nicht der Fall, muss ein betreffender Link entfernt werden.
In jedem Fall empfiehlt es sich, bei allen Hyperlinks auf externe Informationsangebote darauf hinzuweisen, dass es sich bei den verlinkten Informationen um fremde Angebote handelt, die außerhalb des Einflussbereichs der Universität liegen.
\end{enumerate}
\subsection*{§ 5 Verstöße gegen Vorschriften}

\begin{enumerate}
\item Informationsangebote, deren Inhalte offensichtlich gegen diese Ordnung, gegen vorrangige Ordnungen und Regeln oder sonstige Rechtsvorschriften verstoßen, sind vom Betreiber des jeweiligen Informations-Servers unverzüglich zu löschen. Die für die beanstandeten Dokumente zuständigen Verantwortlichen sind entsprechend zu informieren.
\item Erscheint ein Verstoß nach Absatz 1 Satz 1 zwar nicht offensichtlich, aber möglich, informiert der Betreiber die verantwortliche Person hierüber mit der Bitte, die Rechtmäßigkeit des fraglichen Dokuments zu begründen bzw. das Dokument zu löschen. In Zweifelsfällen ist die Hochschulleitung zu informieren.
\item Informationsangebote, aus denen nicht unmittelbar zu entnehmen ist, wer für sie verantwortlich ist, können vom jeweiligen Betreiber gelöscht werden.
\item Informationsanbietende können vorübergehend oder dauerhaft in der Benutzung der DV-Ressourcen beschränkt oder hiervon ausgeschlossen werden, wenn 
sie schuldhaft gegen diese Ordnung, insbesondere gegen die in §§ 2 bis 4 aufgeführten Pflichten verstoßen oder 
sie die DV-Ressourcen der Universität für strafbare Handlungen missbrauchen oder 
der Universität durch sonstiges rechtswidriges Nutzerverhalten Nachteile entstehen. 

Diese Maßnahmen werden erst nach vorheriger erfolgloser Abmahnung
durch den Server-Betreiber getroffen.
\item Über vorübergehende Nutzungseinschränkungen entscheidet der jeweilige Betreiber nach Anhörung der oder des Informationsanbietenden. Die Informationsanbietenden sind über den Zeitpunkt der Einschränkung zu informieren, ihnen ist Gelegenheit zu geben, die vorhandenen Daten zu sichern. Eine vorübergehende Nutzungseinschränkung ist aufzuheben, sobald eine ordnungsgemäße Nutzung wieder gewährleistet erscheint.
\item Die Entscheidung über eine dauerhafte Nutzungseinschränkung oder einen vollständigen Ausschluss einer oder eines Informationsanbietenden trifft die Hochschulleitung auf Antrag des Betreibers. Diese Maßnahme kommt nur bei schwerwiegenden oder wiederholten Verstößen nach Absatz 4 in Betracht und setzt voraus, dass auch künftig ein ordnungsgemäßes Verhalten nicht mehr zu erwarten ist. Absatz 5 gilt entsprechend. Mögliche Ansprüche des jeweiligen Betreibers aus dem Nutzungsverhältnis bleiben hiervon unberührt.
\item Der Server-Betreiber ist nicht verpflichtet, eine Routinedurchsicht der Dokumente auf seinem Server durchzuführen. Bei positiver Kenntnis eines Verstoßes gegen diese Ordnung hat der Serverbetreiber gemäß Absatz 1 bis Absatz 6 tätig zu werden.
\end{enumerate}
\subsection*{
§ 6
Haftung der oder des Informationsanbietenden}
\begin{enumerate}
\item Die oder der Informationsanbietende (siehe § 4 Abs. 2) haftet für alle Schäden, die der Universität durch schuldhafte Missachtung dieser Ordnung entstehen. Diese Haftung umfasst auch Schäden, die der Universität durch Nutzung ihrer Ressourcen durch unberechtigte Dritte entstanden sind, wenn die oder der Informationsanbietende diese Drittnutzung zu vertreten hat, insbesondere im Falle einer Weitergabe ihrer oder seiner Benutzerkennung an Dritte. In diesem Falle kann die Universität von der oder dem Informationsanbietenden nach Maßgabe der Entgeltordnung ein Nutzungsentgelt für die Drittnutzung verlangen.
\item Die oder der Informationsanbietende hat die Universität von allen Ansprüchen freizustellen, wenn Dritte die Universität wegen ihrer oder seiner rechtswidrigen Informationsangebote auf Schadensersatz, Unterlassung oder in sonstiger Weise in Anspruch nehmen. Die Universität wird gegen die oder den Informationsanbietenden ggf. rechtliche Schritte einleiten, sofern Dritte gegen sie gerichtlich vorgehen.
\item Haftungsregelung unberührt. Im Übrigen gelten ergänzend die Bestimmungen der allgemeinen DV-Nutzungsordnung.
\end{enumerate}
\subsection*{§ 7 Haftung der Universität}
\begin{enumerate}
\item Die Universität übernimmt keine Garantie dafür, dass ihre Informations-Server fehlerfrei und jederzeit ohne Unterbrechung laufen. Eventuelle Datenverluste infolge technischer Störungen sowie die Kenntnisnahme vertraulicher Daten durch unberechtigte Zugriffe Dritter sind nicht ausschließbar.
\item Die Universität übernimmt keine Verantwortung für die zur
  Verfügung gestellten Server-Programme.
Die Universität haftet auch nicht für den Inhalt, insbesondere
insbesondere für die Richtigkeit, Vollständigkeit und Aktualität der
Informationen, zu denen sie lediglich den Zugang vermittelt.
\item  Die Universität haftet nur bei Vorsatz und grober
  Fahrlässigkeit ihrer Mitarbeiter, es sei denn, dass eine schuldhafte
  Verletzung wesentlicher Vertragspflichten oder Kardinalpflichten
  vorliegt. In diesem Fall ist die Haftung der Universität auf
  typische, bei Begründung des Nutzungsverhältnisses vorhersehbare
  Schäden beschränkt. Die Haftungssumme wird - außer bei vorsätzlich
  herbeigeführten Schäden - auf 1.000 DM begrenzt.
\item Mögliche Amtshaftungsansprüche gegen die Universität bleiben von den vorstehenden Regelungen unberührt.
\end{enumerate}
\subsection*{§ 8
In-Kraft-Treten}

Diese Ordnung tritt am Tage nach ihrer hochschulöffentlichen Bekanntmachung in Kraft.

%%% Local Variables: 
%%% mode: latex
%%% TeX-master: "../Netzeinfuehrung"
%%% End: 
