
\section[DFN-Benutzungsordnung]{Benutzungsordnung f"ur das Zusammenwirken der Anwender der DFN-Kommunikationsdienste}
\label{dfn}

{\small -- vom Vorstand beschlossen am 16.05.1994 und ge"andert am 09.08.2001 --

Stand: 30.\,05.\,2009}

Ziel der Benutzerordnung ist es, die Zusammenarbeit der Anwender
untereinander zu regeln. Um dieses Ziel zu erreichen, werden im folgenden
eine Reihe von unterst"utzenden organisatorischen Ma"snahmen durch die
nutzenden Einrichtungen gefordert und Verhaltensregeln f"ur einen sinnvollen
Umgang mit den Netzressourcen und zur Vermeidung mi"sbr"auchlicher Nutzung
aufgestellt.

Die Benutzerordnung richtet sich in erster Linie an Personen, die f"ur die
Bereitstellung von Kommunikationsdiensten in den Mitgliedseinrichtungen des
DFN--Vereins verantwortlich sind. Es wird erwartet, da"s jede Einrichtung ihre
Endnutzer von dieser Benutzerordnung in Kenntnis setzt. Dar"uber hinaus wird
empfohlen, f"ur die lokal angebotenen Kommunikationsdienste eine eigene
Benutzerordnung zu erstellen, die mit den in diesem Dokument aufgestellten
Richtlinien in Einklang steht oder auf sie verweist.

Das Einhalten dieser Ordnung liegt im gemeinsamen Interesse aller
Beteiligten, da die Verschwendung von Netzressourcen oder deren Mi"sbrauch zu
einer Erh"ohung der Nutzungsentgelte und zu Unregelm"a"sigkeiten bei der
Nutzung der Dienste f"uhren k"onnte.

\subsection{Geltungsbereich}

Die Benutzerordnung bezieht sich auf die DFN--Dienste, die auf der Grundlage
des Wissenschaftsnetzes (WiN) bereitgestellt werden und dazu dienen, den
nutzenden Einrichtungen eine leistungsf"ahige und st"orungsfreie
Kommunikationsinfrastruktur bereitzustellen.

Zum einen handelt es sich dabei um das WiN mit "Uberg"angen zu anderen Netzen,
die f"ur die Kommunikation zur Verf"ugung gestellt werden, zum anderen um die
Infrastruktur f"ur elektronische Post (z. B. Gateways und Relays) und um
Informationsdienste.

\subsection{Anforderungen an die nutzenden Einrichtungen}

Jede am DFN beteiligte Einrichtung tr"agt Sorge f"ur die Wahrnehmung der
Aufgaben des Netzadministrators, des Postmasters und der Verantwortlichen
f"ur Anwendungen sowie f"ur Beratung und Ausbildung. Die Aufgaben m"ussen nicht
notwendigerweise von verschiedenen Personen in der Einrichtung erbracht
werden. Je nach Gr"o"se der Einrichtung wird eine Person mehr als eine der
beschriebenen Aufgaben wahrnehmen. Es ist jedoch erforderlich, da"s jede
Einrichtung die f"ur die genannten Funktionen verantwortlichen Personen mit
den Aufgaben betraut.

Die mit der Wahrnehmung der Funktionen betrauten Personen sollen
verpflichtet werden, den im DFN--Verein abgesprochenen Routing--Strategien (z.
B. IP--Routing, Mail--Routing) zu folgen.

\subsubsection{Netzadministratorfunktion}

Der DFN--Verein empfiehlt, dem "ortlichen Netzadministrator folgende Aufgaben
zu "ubertragen: Der Netzadministrator sorgt f"ur

\begin{itemize}
  \item die Sicherung und Sicherheit des Netzzugangs,
  \item die Funktionsf"ahigkeit der Untervermittlung,
  \item die Netzverwaltung (Routerkonfiguration und --management,
    IP--Host--Adress-- vergabe),
  \item die Domainverwaltung (Betrieb des Nameservers, Verwaltung der
    Zonendaten und Domain--Namensvergabe),
  \item die Strukturierung der Datenfl"usse,
  \item die Fehlererkennung, Fehlermeldung und ggf. Fehlerbehebung,
  \item die Sicherstellung ununterbrochener Betriebsbereitschaft,
  \item den Kontakt zum DFN--Verein zur Sicherstellung des st"orungsfreien
    WiN--Zugangs.
\end{itemize}

\subsubsection{Postmasterfunktion}

Zum reibungslosen Ablauf des Maildienstes soll ein Postmaster benannt
werden, der folgende Aufgaben wahrnimmt:

\begin{itemize}
  \item Bereitstellen der Maildienste auf lokaler Ebene,
  \item Pflege der Adre"stabellen,
  \item Anlaufstelle bei Mailproblemen f"ur Endnutzer sowie f"ur die Betreiber
    von Gateway- und Relaydiensten.
\end{itemize}

\subsubsection{Funktion eines Verantwortlichen f"ur Anwendungen}

Ein Verantwortlicher f"ur Anwendungen soll benannt werden f"ur folgende
Aufgaben:

\begin{itemize}
  \item Pflege der angebotenen Services (Mailserver, Newsserver, FTP--Server),
  \item Pflege weiterer Kommunikationsdienste,
  \item Fehlermanagement.
\end{itemize}

\subsubsection{Beratungs- und Schulungsfunktion}

Die Aus"ubung dieser Funktion ist notwendig, um Fehlbedienungen durch die
Endnutzer zu vermeiden. Sie setzt sich aus folgenden Aufgaben zusammen:

\begin{itemize}
  \item Bereitstellen einer telefonischen Beratungsstelle w"ahrend der
    Arbeitszeit,
  \item Bereitstellen von Informations- und Schulungsmaterial,
  \item Aufkl"arung "uber Auswirkungen von Fehlverhalten bei den Endnutzern.
\end{itemize}

\subsection{Mi"sbrauch}

\subsubsection{Mi"sbr"auchliche Nutzung}

Mi"sbr"auchlich ist die Nutzung der DFN--Dienste, wenn das Verhalten der
Benutzer gegen einschl"agige Schutzvorschriften (u.a.\  Strafgesetz,
Jugendschutzgesetz, Datenschutzrecht) verst"o"st.

Aufgrund ihrer Fachkunde ist bei den Benutzern der Kommunikationsdienste die
jeweilige, insbesondere strafrechtliche Relevanz etwa der
Computer--Kriminalit"at, des Vertriebs pornographischer Bilder und Schriften
oder des Diebstahls, der Ver"anderung oder sonstige Manipulation von bzw.\  an
Daten und Programmen als bekannt vorauszusetzen. Diese Fachkenntnis bezieht
sich auch auf die Sensibilit"at der "Ubertragung von Daten, die geeignet sind,
das Pers"onlichkeitsrecht anderer und/oder deren Privatsph"are zu
beeintr"achtigen oder bestehende Urheberrechte bzw.\  auf diesen gr"undende
Lizenzen zu verletzen.

Als mi"sbr"auchlich ist auch eine Nutzung zu bezeichnen, die folgende, nicht
abschlie"send aufgef"uhrte Sachverhaltskonstellationen erf"ullt:

\begin{itemize}
  \item unberechtigter Zugriff zu Daten und Programmen, d.h.\  mangels
    Zustimmung unberechtigter Zugriff auf Informationen und Ressourcen
    anderer verf"ugungsbefugter Nutzer
  \item Vernichtung von Daten und Programmen, d.h.\  Verf"alschung und/oder
    Vernichtung von Informationen anderer Nutzer -- insbesondere auch durch
    die "`Infizierung"' mit Computerviren
  \item Netzbehinderung, d.h.\  Behinderungen und/oder St"orungen des
    Netzbetriebes oder anderer netzteilnehmender Nutzer, z. B. durch
    \begin{itemize}
      \item ungesichertes Experimentieren im Netz, etwa durch Versuche zum
        "`Knacken"' von Pa"sw"ortern,
      \item nichtangek"undigte und/oder unbegr"undete massive Belastung des
        Netzes zum Nachteil anderer Nutzer oder Dritter.
    \end{itemize}
\end{itemize}

\subsubsection{Empfehlungen an die nutzenden Einrichtungen zur Verhinderung des
Mi"sbrauchs}

Beim Mi"sbrauch der DFN--Dienste kann man grob unterscheiden zwischen
Mi"sbrauch aus Unkenntnis, fahrl"assigem und vors"atzlichem Mi"sbrauch. Je nach
Art des Mi"sbrauchs sind unterschiedliche Aktivit"aten zu seiner Verhinderung
gefragt. Sie reichen von der Aufkl"arung der Nutzer, "uber erh"ohte technische
Sicherheitsma"snahmen bis hin zur Androhung von Nutzungsausschluss und Haftung
f"ur schuldhaft verursachte Sch"aden.

Voraussetzung f"ur die Aufkl"arung von Mi"sbr"auchen ist, da"s die Personen,
denen Zugang zum DFN gew"ahrt wird, namentlich autorisiert sind. Die
Einrichtung, die Netzzugang gew"ahrt, darf daher nat"urlichen Personen den
Zugang nur erm"oglichen, wenn die Personen eine Berechtigung zur Nutzung
haben.

Durch die Wahrnehmung der geforderten Schulungs- und Beratungsfunktion und
durch Aufkl"arungsarbeit "uber Auswirkungen von falschem Nutzungsverhalten auf
andere Nutzer kann dem Mi"sbrauch aus Unkenntnis und dem fahrl"assigen
Mi"sbrauch entgegengewirkt werden. Dazu geh"ort insbesondere, die Endnutzer
zur vertraulichen Behandlung aller Pa"sworte, die f"ur den Zugang zu den
Kommunikationsdiensten ben"otigt werden, zu verpflichten und sie dazu
anzuhalten, ihre Pa"sworte so zu w"ahlen, da"s sie nicht durch einfache
Crackprogramme entschl"usselt werden k"onnen.

Dar"uber hinaus sollten die Betreiber von Kommunikationsdiensten in
zumutbarem Umfang Verfahren bereitstellen, die den pers"onlichen Charakter
und die Vertraulichkeit der auf elektronischem Wege ausgetauschten
Nachrichten oder sensitiven Daten wahren und sch"utzen. Je nach
Sicherheitsrelevanz der Daten wird folgendes empfohlen:

\begin{itemize}
  \item Einsetzen der vom Hersteller gelieferten Sicherheitsmechanismen (z. B.
    Pa"swortschutz),
  \item Anwendung topologischer Ma"snahmen (Abtrennen sicherheitsrelevanter
    Systeme durch Router und Bridges),
  \item Netz"uberwachung (z. B. Protokollierung von Zugriffen),
  \item Einhalten von Sicherheitsklassen (s. "`Kriterien f"ur die Bewertung der
    Sicherheit von Systemen der Informationstechnik"' (ITSEC), Luxemburg
    1991).
\end{itemize}

In den F"allen, wo Nutzern der uneingeschr"ankte Zugang zu bestimmten
Datenbest"anden gew"ahrt wird, ist durch geeignete Ma"snahmen daf"ur zu sorgen,
da"s die Nutzer "uber diesen Weg nicht den unautorisierten Zugang zu
weiteren, nicht--"offentlichen Datenbest"anden erhalten k"onnen. Die Betreiber
sind dar"uber hinaus gehalten, den DFN--Verein beim Aufsp"uren und Verhindern
unzul"assiger Nutzung in zumutbarem Umfang zu unterst"utzen.

Zus"atzlich soll der Endnutzer durch lokale Regelwerke auf den zul"assigen
Gebrauch der Kommunikationsdienste und die Auswirkungen von Fehlverhalten
hingewiesen bzw.\  vor Mi"sbrauch gewarnt werden.

\subsection{Konsequenzen bei Verst"o"sen}

Die das Deutsche Forschungsnetz nutzenden Einrichtungen sind verpflichtet,
ihre Endnutzer mit der Benutzungsordnung und den f"ur sie relevanten Inhalten
der Vertr"age mit dem DFN--Verein vertraut zu machen.

Bei Verst"o"sen gegen die Nutzungsregelungen sind die nutzenden Einrichtungen
gehalten, den Mi"sbrauch unverz"uglich abzustellen und sich untereinander zu
informieren.

Sollte es zur Wahrung der Interessen aller Einrichtungen, die die
Kommunikationsdienste des DFN--Vereins nutzen, erforderlich sein, ist der
DFN--Verein frei, aufgrund der unzul"assigen Nutzung einzelne Personen oder
Einrichtungen von der Nutzung der angebotenen Dienste oder Teilen davon
auszuschlie"sen. In besonders schwerwiegenden F"allen, bei denen die
unzul"assige Nutzung eine Verletzung von geltendem Recht darstellt, k"onnen
zivil- oder strafrechtliche Schritte eingeleitet werden.
