
\section[Richtlinien f"ur das TUBS-Net]{Richtlinien f"ur das Hochschulnetz der
  Technischen Universit"at Braunschweig}
\label{TUBSnet-Richtlinien}

{\small Stand: Dezember 1994}

\subsection{Allgemeines}

\begin{enumerate}
  \item Hochschulnetz im Sinne der "`Benutzungsordnung f"ur das Rechenzentrum"'
    [Anhang \ref{RZordnung}] ist das allgemeine Datenkommunikationsnetz
    der TU Braunschweig. F"ur den Betrieb des Hochschulnetzes gelten die
    nachfolgend aufgef"uhrten Bestimmungen.

  \item Die Verantwortung f"ur den Betrieb des Hochschulnetzes
    als logisches System liegt beim Rechenzentrum, das die dem
    Netzbetrieb (Netzmanagement, Einstellung intelligenter
    Netzkomponenten) zugrundeliegenden Programme betreibt.

  \item Die Verantwortung f"ur die Leitungen und "Ubergabepunkte
    des Hochschulnetzes liegt bei der Abteilung Betriebstechnik
    der TU Braunschweig.

  \item Die verf"ugbaren Netzdienste und Anschlusstypen (Kopplungselemente)
    werden in Publikationen des Rechenzentrums bekanntgemacht.

  \item Der Zugang zum Hochschulnetz ist "uber Rechner des
    Rechenzentrums oder "uber Rechner bzw. Rechnernetze
    (kurz: "`Rechner"') der Nutzungsberechtigten m"oglich,
    f"ur die geeignete Anschl"usse bereitgestellt werden.

  \item Betreiber von nutzereigenen "`Rechnern"' am Hochschulnetz
    ("`Rechnerbetreiber"') k"onnen alle Nutzungsberechtigten
    gem"a"s \S{}3 der Benutzungsordnung [Anhang \ref{RZordnung}] sein.

  \item F"ur jeden an das Hochschulnetz angeschlossenen "`Rechner"'
    gibt es einen "Ubergabepunkt, in der Regel ein
    Kopplungselement, mit dem die betriebstechnische Verantwortung
    f"ur das Netz endet. Die Kommunikation auf der
    Nutzerseite, d.h.\  bis zum "Ubergabepunkt, liegt in der
    Verantwortung des "`Rechnerbetreibers"'.
\end{enumerate}

\subsection{Aufgaben und Pflichten des Rechenzentrums}

\begin{enumerate}
  \item Das Rechenzentrum ist stets um einen sicheren und
    ununterbrochenen Betrieb des Hochschulnetzes bem"uht,
    soweit dies beim Stand der Technik m"oglich ist.
    Ein st"orungsfreier Betrieb kann aber angesichts des
    Entwicklungsstandes und des offenen Charakters der
    derzeitigen Netze nicht garantiert werden.

  \item Das Rechenzentrum vergibt und verwaltet die Netzadressen
    und ber"at die "`Rechnerbetreiber"' in Fragen der Nutzung
    des Hochschulnetzes. Soweit h"ohere Netzdienste zu
    koordinieren sind, "ubernimmt das Rechenzentrum
    diese Aufgabe.

  \item Das Rechenzentrum "ubernimmt keine Verantwortung
    f"ur "uber das Hochschulnetz herangetragene
    Beeintr"achtigungen des Betriebs der angeschlossenen
    "`Rechner"'.

  \item Das Rechenzentrum hat das Recht, Teile des Hochschulnetzes
    zeitweilig stillzulegen, wenn diese Ma"snahme
    erforderlich ist, um bei St"orungen die
    Funktionsf"ahigkeit des "ubrigen Netzes zu erhalten.
\end{enumerate}

\subsection{Aufgaben und Pflichten der Benutzer des Hochschulnetzes}

\begin{enumerate}
  \item Die Benutzung des Hochschulnetzes besteht darin, "uber
    ein an das Hochschulnetz angeschlossenes Rechnersystem Daten
    in das Hochschulnetz zu senden oder daraus zu empfangen.
    Als solche unterliegt sie der "`Benutzungsordnung f"ur das
    Rechenzentrum der TU Braunschweig"'. [Anhang \ref{RZordnung}]
    Benutzerinnen und Benutzer des Hochschulnetzes sind somit neben den im
    Rechenzentrum registrierten Benutzern auch Personen, die als Mitglieder
    und Angeh"orige der TU Braunschweig unter der Verantwortlichkeit eines
    "`Rechnerbetreibers"' t"atig sind. (Siehe Benutzungsordnung \S{}5 [hier:
    Anhang \ref{RZordnung}])

  \item Der Benutzerin oder dem Benutzer ist es untersagt,
    "Anderungen an den "Ubergabepunkten vorzunehmen.
    Soweit sie geeignet sind, den Betrieb des Hochschulnetzes
    zu beeinflussen, sind alle "Anderungen an den
    angeschlossenen Ger"aten und den vorgesehenen
    Betriebsweisen nur in Absprache mit dem Rechenzentrum
    zul"assig.

  \item Der Benutzerin oder dem Benutzer ist es untersagt, ohne
    Absprache mit dem Rechenzentrum Netzadressen zu "andern
    oder neue Netzadressen einzuf"uhren.

  \item Der Datenverkehr einer Benutzerin oder eines Benutzers darf den
    Datenverkehr anderer Benutzer/-innen nicht unangemessen
    beeintr"achtigen, z.B. durch ungezielte und
    "uberm"a"sige Verbreitung von Informationen.

  \item Jedes unbefugte Mitlesen oder Auswerten von Nachrichteninhalten
    sowie die Weitergabe unbeabsichtigt erhaltener Informationen
    ist untersagt.

  \item Die Benutzerin oder der Benutzer ist verpflichtet, dem
    Rechenzentrum unverz"uglich einen erkannten Mi"sbrauch
    des Hochschulnetzes bzw. St"orungen am Netz anzuzeigen.

  \item Im Sinne des Datenschutzgesetzes schutzw"urdige Daten d"urfen nur in
    verschl"usselter Form auf das Hochschulnetz geleitet werden, da die
    Abh"orsicherheit des Netzes nicht gew"ahrleistet werden kann.

  \item Vor "Ubertragung von sehr gro"sen Datenmengen bzw.\  von
    schutzw"urdigen Daten hat der "`Rechnerbetreiber"' Absprachen
    mit dem Rechenzentrum zu treffen.
\end{enumerate}

\subsection{Aufgaben und Pflichten des Rechnerbetreibers}

\begin{enumerate}
  \item Der Anschluss eines "`Rechners"' an das Hochschulnetz sowie
    der Ausbau des Leitungsnetzes m"ussen vom "`Rechnerbetreiber"'
    beim Rechenzentrum beantragt werden. Die Entscheidung "uber
    den Antrag liegt beim Rechenzentrum.

  \item Der "`Rechnerbetreiber"' ist f"ur die ordnungsgem"a"se
    Nutzung des Hochschulnetzes verantwortlich, da nur er den
    Zugang zu seinen "`Rechnern"' und damit auch zum Netz kontrollieren
    kann. Insbesondere hat er darauf zu achten, da"s nur
    dazu berechtigte Nutzerinnen und Nutzer Zugang zum Netz
    bekommen und da"s "uber den "`Rechner"' kein
    unautorisierter Netzzugang erschlossen wird.

  \item Der "`Rechnerbetreiber"' verpflichtet die Nutzer/-innen seiner
    Netzanschl"usse auf Einhaltung der Benutzungsordnung. [Anhang
    \ref{RZordnung}]

  \item Das Hochschulnetz sowie die Datenverbindungen nach au"sen
    d"urfen nur f"ur Aufgaben in Lehre und Forschung
    genutzt werden.

  \item Das Rechenzentrum teilt den "`Rechnerbetreibern"' Adressenbereiche
    f"ur Netzadressen zu, die von diesen z.T. selbst ausgef"ullt werden k"onnen.

  \item Soweit erforderlich werden zwischen Rechenzentrum und
    "`Rechnerbetreiber"' besondere Vereinbarungen getroffen, wie z.B.:
    \begin{itemize}
      \item zum "Ubergabepunkt,
      \item zur Nutzung spezieller Netzdienste,
      \item zu Kosten (z.B. Einrichtungs-, Wartungskosten,
        Verkehrsgeb"uhren).
    \end{itemize}
\end{enumerate}
