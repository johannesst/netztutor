
\section{Satzung des Vereins SchunterNet e.V.}
\label{satzung}

{\small Stand: 05. August 1997}

\Paragraph{Name und Sitz}
\begin{enumerate}
\item Der Verein f"uhrt den Namen "`\snev"'
\item Der Verein hat seinen Sitz in Braunschweig.
\item Der Verein ist eingetragener Verein im Vereinsregister beim Amtsgericht 
Braunschweig.
\end{enumerate}

\Paragraph{Vereinszweck}
\begin{enumerate}
\item Zweck des Vereins \snev ist die Planung, der Aufbau
    und der Betrieb eines Computernetzwerkes im "`Studentenwohnheim An der
    Schunter"' einschlie"slich dessen Anbindung an das Netz der Technischen
    Universit"at Braunschweig, sowie die F"orderung der Kommunikation von
    Studenten im nationalen und internationalen Rahmen.
\item Der Verein verfolgt ausschlie"slich gemeinn"utzige Zwecke und ist nicht
    eigenwirtschaftlich t"atig.
\end{enumerate}

\Paragraph{Mitgliedschaft}\label{mitglied}
\begin{enumerate}
\item Mitglied kann jeder Bewohner der Wohnheimanlage "`Studentenwohnheim An
  der Schunter"' werden.
\item Der Verein besteht aus aktiven und passiven Mitgliedern.
\item Passives Mitglied kann werden, wer einen Antrag auf einen
  SchunterNet--Anschluss gestellt hat und die Satzung sowie deren
  Erg"anzungsordnungen anerkennt.
\item Aktives Mitglied kann werden, wer einen Antrag auf einen
  SchunterNet--Anschluss gestellt hat, die Satzung und deren
  Erg"anzungsordnungen anerkennt und aktiv am Aufbau und Betrieb des Netzes
  mitarbeitet.
\item \label{aufnahme} "Uber die Aufnahme aktiver Mitglieder entscheidet die
  Mitgliederversammlung mit einfacher Mehrheit. Die Mitgliedschaft beginnt mit
  der Aufnahme durch die Mitgliederversammlung.
\item Ein passives Mitglied kann in jeder Mitgliederversammlung einen Antrag
  auf aktive Mitgliedschaft stellen. Die Aufnahme erfolgt entsprechend
  Ziffer \ref{aufnahme}.
\item Die Mitgliedschaft endet durch Verlust der Gesch"aftsf"ahigkeit, Auszug
  aus dem "`Studentenwohnheim An der Schunter"' oder Austrittserkl"arung.
\item Der Austritt ist schriftlich gegen"uber dem Vorstand zu erkl"aren 
  und wird einen Monat nach Eingang der schriftlichen Austrittserkl"arung bei 
  dem Vorstand wirksam.
\item \label{ausschluss}Mitglieder, die dem Zweck und Ansehen des Vereins
  zuwider handeln oder gegen Bestimmungen der g"ultigen Satzung oder der
  Erg"anzungsordnungen versto"sen, k"onnen durch Beschluss des Vorstandes aus dem
  Verein ausgeschlossen werden. Widerspricht der Betroffene innerhalb eines
  Monats, entscheidet die Mitgliederversammlung "uber den Ausschluss mit 2/3
  Mehrheit. Bis zur Entscheidung der Mitgliederversammlung ruhen die
  Mitgliedschaftsrechte des betroffenen Mitglieds.
\end{enumerate}

\Paragraph{Beitr"age}
\begin{enumerate}
\item[] Von den Mitgliedern werden keine Beitr"age erhoben.
\end{enumerate}

\Paragraph{Organe}
\begin{enumerate}
\item[] Die Organe des Vereins sind:
  \begin{enumerate}
  \item die Mitgliederversammlung
  \item der Vorstand
  \end{enumerate}
\end{enumerate}

\Paragraph{Mitgliederversammlung}
\begin{enumerate}
\item[]
  \begin{sloppypar}
    Die Mitgliederversammlung bestimmt auf der Grundlage des Vereinszwecks 
    die Richtlinien f"ur die T"atigkeit des Vereins.
  \end{sloppypar}

  Sie ist im "ubrigen insbesondere zust"andig f"ur:
  \begin{enumerate}
  \item Die Entgegennahme des Jahresberichtes des Vorstandes
  \item Die Erteilung von Entlastungen
  \item Die Wahl des Vorstandes
  \item Die Wahl der Systemverwaltung
  \item Satzungs"anderungen und Aufl"osung des Vereins
  \item Aufnahme von Mitgliedern und, im Falle des Widerspruchs gegen den den
    Ausschluss aussprechenden Vorstandsbeschluss, f"ur den Ausschluss von
    Mitgliedern.
\end{enumerate}
\end{enumerate}

\Paragraph{Einberufung der Mitgliederversammlung}
\begin{enumerate}
\item Die ordentliche Mitgliederversammlung muss mindestens einmal im Semester
  stattfinden. Der Termin der Mitgliederversammlung ist mindestens eine Woche
  vorher bekannt zu geben. Die Bekanntgabe des Termins erfolgt durch Aushang
  am Anschlagbrett der Heimselbstverwaltung oder E-Mail.
\item Au"serordentliche Mitgliederversammlungen finden statt:
  \begin{enumerate}
  \item auf Beschluss des Vorstandes oder
  \item wenn dies 10\% der Mitglieder unter Angabe des Zwecks verlangen.
  \end{enumerate}
  Die Versammlung wird vom Vorstand durch Aushang am schwarzen Brett oder
  E-Mail mit einer Ladungsfrist von einer Woche unter Mitteilung der
  Tagesordnung einberufen. 
\item Die Mitgliederversammlung beschlie"st mit einfacher Mehrheit der 
  Stimmen. Stimmberechtigt ist jedes aktive Mitglied.

  Satzungs"anderungen, die vorzeitige Abwahl des Vorstandes und die 
  Entscheidung "uber den Ausschluss von Mitgliedern nach \S{}3
  Ziffer \ref{ausschluss} dieser Satzung erfordern eine Mehrheit von 2/3 der
  abgegebenen Stimmen.
\item Jedes aktive und passive Mitglied hat in der Mitgliederversammlung
  Rederecht und darf Antr"age stellen. Werden gegen einen Beschluss die
  Unterschriften von mehr als der H"alfte aller aktiven und passiven
  Mitglieder vorgelegt, so gilt dieser als nicht gefa"st.
\end{enumerate}

\Paragraph{Beschlussf"ahigkeit der Mitgliederversammlung}
\begin{enumerate}
\item Jede ordnungsgem"a"s einberufene Mitgliederversammlung ist
  beschlussf"ahig, wenn mindestens 2/3 der aktiven Mitglieder anwesend 
  sind.
\item Im Falle der Beschlussunf"ahigkeit ist die Mitgliederversammlung
  innerhalb eines Monats erneut einzuberufen. Diese ist dann ohne R"ucksicht
  auf die Anzahl der erschienenen Mitglieder beschlussf"ahig.
\end{enumerate}

\Paragraph{Vorstand}
\begin{enumerate}
\item Die Zahl der Vorstandsmitglieder bestimmt die Mitgliederversammlung.\\
  Der Vorstand besteht jedoch mindestens aus drei Mitgliedern, n"amlich dem 
  Vorsitzenden, dem stellvertretenden Vorsitzenden und dem Kassenwart.
\item Je zwei Vorstandsmitglieder vertreten den Verein gemeinsam.
\item Die Vorstandsmitglieder werden von der Mitgliederversammlung einzeln und
  auf die Dauer von zwei Studiensemestern gew"ahlt.
  
  Jedes Vorstandsmitglied bleibt im Amt, bis die Amtszeit des neugew"ahlten
  Nachfolgers beginnt oder die Mitgliederversammlung beschlossen hat, sein Amt
  nicht wieder zu besetzen. Eine Wiederwahl ist m"oglich.

  Die vorzeitige Abwahl eines Vorstandsmitgliedes kann nur mit 2/3 Mehrheit der
  ordnungsgem"a"s einberufenen Mitgliederversammlung erfolgen. Die
  Abwahl eines Vorstandsmitgliedes wird erst wirksam, wenn sich die 
  Mitgliederversammlung zugleich auf einen Nachfolger geeinigt oder beschlossen
  hat, sein Amt nicht wieder zu besetzen.
\end{enumerate}

\Paragraph{Aufgaben des Vorstandes}
\begin{enumerate}
\item[] Der Vorstand sorgt f"ur die Durchf"uhrung der Beschl"usse der
  Mitgliederversammlung und f"ur die Information der Mitglieder.
\end{enumerate}

\Paragraph{Beurkundung von Beschl"ussen}
\begin{enumerate}
\item[] Der Schriftf"uhrer fertigt "uber Beschl"usse der Mitgliederversammlung
  Protokolle an, die vom Versammlungsleiter und ihm unterschrieben werden.
\end{enumerate}

\Paragraph{Schlussbestimmungen}
\begin{enumerate}
\item Im Falle einer Auf\/l"osung des Vereins f"allt das Vereinsverm"ogen der
  Heimkasse des Wohnheims "`An der Schunter"' zu.
\item Diese Satzung tritt mit dem Beschluss der Gr"undungsversammlung vom
  23. April 1997 in Kraft.
\end{enumerate}