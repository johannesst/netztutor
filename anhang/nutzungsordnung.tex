
\setcounter{para_nr}{0}

\section{Benutzerordnung des SchunterNet e.V.}
\label{nutzerordnung}

{\small Stand: 16. Februar 2000}

\mbox{ }

{\large\sf\textbf{Pr"aambel}}

Das im Studentenwohnheim \emph{An der Schunter} durch den
\snev betriebene Netzwerk soll allen Mietern die
M"oglichkeit bieten, mit ihren Heimrechnern (Mac, PC, etc.) einfach und
kosteng"unstig an moderner Datenkommunikation zu partizipieren. Die
gemeinsame Nutzung von Ressourcen steht dabei im lokalen Netz im
Mittelpunkt. Dazu geh"oren der Datenaustausch "uber das Netz, die
Nutzung zentraler oder privater Peripherieger"ate (Drucker, Scanner,
Streamer, etc.) sowie die Bereitstellung nichtkommerzieller Software
auf File--Servern.

Mit der Anbindung an das Hochschulnetz erh"alt der Teilnehmer die
M"oglichkeit, seinen PC als Terminal f"ur eine UNIX--Workstation
einzusetzen. Durch die Verbindung mit dem weltweiten Internet
bieten sich den Studenten letztendlich ganz neue Wege der Recherche
und Informationsbeschaffung vom eigenen Schreibtisch aus. Zentraler
Punkt ist hierbei das World Wide Web, welches in den letzten Jahren
die Popularit"at des Internet erst begr"undet hat.

\mbox{ }

\Paragraph{G"ultigkeit}

\begin{enumerate}
  \item Die folgenden Regelungen gelten f"ur alle Benutzer des Netzes
      des \snev im Studentenwohnheim \emph{An der Schunter}, Braunschweig. Sie
      erg"anzen die Benutzungsordnung f"ur das Rechenzentrum der Technischen
      Universit"at Braunschweig. [Anhang \ref{RZordnung}]
  \item Diese Ordnung tritt mit ihrer Ver"offentlichung in Kraft. Sie
      verliert ihre G"ultigkeit bei Inkrafttreten einer neuen Benutzerordnung.
\end{enumerate}

\Paragraph{Allgemeine Bestimmung}

\begin{enumerate}
  \item Die Teilnahme an Datennetzen verlangt von jedem einzelnen einen
      verantwortungsvollen Umgang mit diesem Medium. Die Benutzerordnung wurde
      geschaffen, um die Funktionsf"ahigkeit des Netzwerkes und ein geregeltes
      Miteinander der Teilnehmer zu gew"ahrleisten.
  \item Jeder Benutzer verpflichtet sich, diese Ordnung anzuerkennen.
  \item F"ur die Nutzung der Ressourcen des Hochschulnetzes (TUBS--Net
      und Zugang zum Internet) ist dar"uber hinaus  die
      Benutzungsordnung f"ur das Rechenzentrum der Technischen
      Universit"at Braunschweig [Anhang \ref{RZordnung}] verbindlich.
  \item Betriebs- und Hardwarekosten werden entsprechend der
      Geb"uhrenordnung [Anhang \ref{gebuehr}] des \snev auf die Nutzer
      umgelegt.
\end{enumerate}

\Paragraph{Zulassung der Benutzer}

\begin{enumerate}
  \item Grunds"atzlich ist jeder Bewohner des Studentenwohnheims
      \emph{An der Schunter} berechtigt, sich an das Wohnheimnetz
      anzuschlie"sen, sofern er sich mit den hier aufgef"uhrten
      Regelungen einverstanden erkl"art und dem \snev beitritt.
  \item Einschr"ankungen werden im Einzelfall durch den \snev ausgesprochen.
\end{enumerate}

\Paragraph{An- und Abmeldung}

\begin{enumerate}
  \item Zur Anmeldung ist der Antrag auf Netzanschluss sowie Mitgliedschaft
      im \snev auszuf"ullen und unterschrieben beim Vorstand des \snev
      einzureichen. Dieser stellt einen Nutzervertrag des Teilnehmers mit dem
      Betreiber dar.
  \item "Anderungen der Benutzerdaten sind dem Verein unverz"uglich
      mitzuteilen.
  \item Die Teilnahme kann durch Auszug aus dem Wohnheim, Abmeldung
      oder Ausschluss (s. \S{}8) beendet werden. Auszug oder Abmeldung sind dem
      Verein mindestens sechs Wochen im voraus anzuk"undigen.
\end{enumerate}

\Paragraph{Rechte des Benutzers}

\begin{enumerate}
  \item Jeder Benutzer hat das Recht, den ihm zur Verf"ugung gestellten
      Netzanschluss zu jeder Zeit im Rahmen dieser Benutzerordnung zu
      nutzen.
  \item Grunds"atzlich kann jeder Benutzer alle zur Verf"ugung
      gestellten Dienste des Netzes in Anspruch nehmen.
  \item Der Benutzer wird im Rahmen der M"oglichkeiten durch die
      Vertreter des \snev beraten und betreut. Dies
      ist zu den festgelegten \glossar Sprechstunden in den R"aumlichkeiten des
      Vereins m"oglich.
\end{enumerate} 

\Paragraph{Bereitgestellte Dienste des Netzes}

\begin{description}
  \item[Prim"arer Anschluss] Der prim"are Anschluss an
      der im Zimmer des Benutzers vorhandenen Netzwerkdose wird zur
      Verf"ugung gestellt und auf Antrag freigeschaltet. Der \snev
      ist stets um einen sicheren und unterbrechungsfreien Betrieb des
      Wohnheimnetzes bem"uht, soweit dies beim Stand der Technik und im
      zeitlichen Rahmen der Mitglieder m"oglich ist.
  \item[Sekund"arer Anschluss] Auf gesonderten Antrag
      kann bei ausreichenden Ressourcen der in der Dose vorhandene
      Zweitanschluss f"ur Netzwerknutzung zus"atzlich freigeschaltet
      werden.
  \item[Protokoll des Anschlusses]  Der Anschluss erfolgt "uber
      10 MBit/s Ethernet (10BaseT) und erm"oglicht je nach Netzwerkadapter des
      Benutzers Half-- bzw. Full--Duplex--Betrieb.
  \item[MAC Adressen der Netzwerkadapter] Jedem
      Anschluss werden aus Gr"unden der Sicherheit bis maximal vier vom
      Benutzer angegebene MAC--Adressen (Media Access Control,
      eindeutige Identifizierungsnummer eines Netzwerkadapters) fest
      zugeordnet.
  \item[Protokolle] Im internen Netz wird
      grunds"atzlich jedes Protokoll (TCP/IP, IPX/ODI\tm, NetBIOS\tm,
      appletalk\tm, etc.) durchgeschaltet. Die Verbindung zum
      Hochschulnetz erfolgt ausschlie"slich "uber das Internetprotokoll
      (IP).
  \item[IP--Adresse] Jedem Anschluss wird eine statische
      IP--Adresse zur Verf"u\-gung gestellt, die den Rechner im gesamten
      Internet eindeutig identifiziert. Hierdurch wird es grunds"atzlich
      m"oglich, die Dienste des Internet in Anspruch zu nehmen und von au"sen
      (z.B. vom Rechenzentrum aus) auf die Ressourcen des eigenen Rechners
      zuzugreifen. Einschr"ankungen oder Erweiterungen dieser
      Zugriffsm"oglichkeiten k"onnen nach den W"unschen des Benutzers vom
      Administrator eingestellt werden.
  \item[E-Mail] Jeder Benutzer erh"alt eine eigene
      Mailbox mit Adresse, auf die er "uber POP3 oder Imap zugreifen
      kann.
  \item[WWW-- und FTP--Zugang] Der Zugang zum
      World Wide Web und zu FTP--Servern wird "uber einen Proxyserver
      gew"ahrleistet.
  \item[Usenet News] Je nach vorhandenen Ressourcen
      werden Newsgruppen nach den W"un\-schen der Benutzer lokal zur
      Verf"ugung gestellt.
  \item[lokaler NTP Service] Es wird eine zeitgenaue
      Synchronisierung der Systemuhren der eigenen Rechner erm"oglicht.
  \item[Benutzerserver] Jedem Teilnehmer wird ein
      Arbeitsverzeichnis auf dem Benutzerserver (Linux--Workstation)
      zur Verf"ugung gestellt. Nach Wunsch k"onnen projektbezogene
      Benutzergruppen eingerichtet werden. Jeder Benutzer darf bis zu neun
      weitere Accounts mit eingeschr"ankten M"oglichkeiten beantragen.
  \item[private WWW--Homepages] Jeder Benutzer kann im
      Rahmen des zur Verf"u\-gung gestellten Speicherplatzes eigene \glossar
      Homepages erstellen und ver"offentlichen.
  \item[weitere Dienste] Der Verein kann dar"uber
      hinaus weitere Dienste zur Verf"u\-gung stellen.
\end{description}

\Paragraph{Pflichten des Benutzers}

Jeder Teilnehmer ist f"ur seinen Rechner und den Netzzugang "uber selbigen
voll verantwortlich. Das bedeutet:

\begin{enumerate}
  \item Der Teilnehmer hat die ihm zur Verf"ugung gestellten Betriebsmittel
      und Dienste sorgf"altig und ihren Bestimmungen entsprechend zu benutzen.
  \item Jeder Teilnehmer hat Ma"snahmen zum Schutz vor
      unbefugter Nutzung seines Anschlusses und der zur Verf"ugung gestellten
      Dienste durch Dritte zu ergreifen.
  \item Jeder Verdacht auf Mi"sbrauch von Ressourcen ist der Netzverwaltung
      unverz"uglich zu melden.
  \item Bauliche Ver"anderungen an der Netzwerkinstallation d"urfen nur mit
      schriftlicher Genehmigung des \snev vorgenommen werden.
  \item Die St"orung oder Beeintr"achtigung des Netzbetriebs durch unsachgem"a"sen
      Einsatz von Hard- und Software ist zu vermeiden. St"orungen jeder Art
      sind unverz"uglich dem \snev zu melden.
  \item Es ist dem Teilnehmer verboten,  eine andere als die ihm zugewiesene
      IP--Adresse im Netz zu benutzen (Address--Spoofing) oder Masquerading zu
      betreiben.
  \item Der am Netz angeschlossene Rechner darf grunds"atzlich nicht f"ur
      Routingzwecke verwendet werden. Ausnahmen bed"urfen der schriftlichen
      Genehmigung des \snev
  \item Jede Art des Mith"orens von Daten"ubertragungen, des unberechtigten
      Zugriffs auf fremde Daten oder des unberechtigten Zugangs zu fremden
      Rechnern ist zu unterlassen. Schon der Versuch ist strafbar.
  \item Die Bereitstellung und Nutzung von Software und Dokumentationen ist
      nur im Rahmen der ma"sgeblichen Lizenzbestimmungen zul"assig.
  \item Das Beziehen oder Verbreiten strafrechtlich relevanter Daten ist zu
      unterlassen.
  \item Der Teilnehmer ist dazu verpflichtet regelm"a\ss ig, nach M"oglichkeit
      t"aglich,  seine SchunterNet-EMails zu lesen. Er hat weiterhin daf"ur zu
      sorgen,  dass  er die EMails auch korrekt empfangen kann. Dabei ist es
      unerheblich, ob die EMails direkt vom SchunterNet-Mailserver oder "uber
      eine Weiterleitung bezogen werden.
\end{enumerate}

\Paragraph{Verfahren bei Verst"o"sen gegen die Benutzerordnung}

\begin{enumerate}
  \item Benutzer, die gegen die Benutzerordnung versto"sen, werden von den
      Vertretern des \snev auf den Versto"s hingewiesen.
  \item Bei schweren oder wiederholten Verst"o"sen gegen die Bestimmungen der
      Benutzerordnung wird der betreffende Teilnehmer von der weiteren Nutzung
      ausgeschlossen. Werden Belange des Zusammenlebens im Wohnheim ber"uhrt,
      kann zus"atzlich ein Heimratsverfahren angestrengt werden.
  \item Die Ger"ate und Anlagen werden in funktionsf"ahigem Zustand
      "ubergeben. Durch unsachgem"a"se Behandlung eingetretene Sch"aden hat
      der Nutzer in vollem Umfang zu tragen. Bei Beendigung der Nutzung,
      sp"atestens beim Auszug, wird von Vertretern des \snev der Zustand
      kontrolliert und ein Abnahmeprotokoll erstellt. Der Nutzer bleibt f"ur
      entstehende Sch"aden haftbar, solange er diese Abnahme nicht durchf"uhren
      lassen hat.
  \item Wird durch Verst"o"se zus"atzlicher administrativer Aufwand zur
      Wiederherstellung oder Bewahrung der Funktion und Sicherheit des Systems
      notwendig, so hat der Verursacher die entstehenden Kosten sowie die
      Arbeitsleistung entsprechend den in der Geb"uhrenordnung [Anhang
      \ref{gebuehr}] festgelegten Tarifen zu tragen.
  \item Wer "uber diese Bestimmungen hinaus gegen die Benutzungsordnung f"ur
      das Rechenzentrum der Technischen Universit"at Braunschweig [Anhang
      \ref{RZordnung}], Interessen dritter, nationales oder internationales
      Recht verst"o"st, hat mit Meldung an die zust"andigen Stellen bis hin zur
      Anzeige zu rechnen.
\end{enumerate}

\Paragraph{Haftungsausschluss}
\begin{enumerate}
  \item Ein Anspruch auf ununterbrochene Funktion des Netzes besteht
      nicht. Schadenersatzanspr"uche des Benutzers gegen"uber den Betreibern
      k"onnen nicht geltend gemacht werden.
  \item F"ur Sch"aden an Hardware, Software oder Daten des Benutzers, die
      durch die Teilnahme am Netzbetrieb entstehen, "ubernimmt der Betreiber
      keine Haftung.

\end{enumerate}
