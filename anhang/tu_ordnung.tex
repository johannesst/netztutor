
\section[RZ-Benutzungsordnung der TU Braunschweig]{Benutzungsordnung f"ur das
  Rechenzentrum der Technischen Universit"at Braunschweig}
\label{RZordnung}

{\small Stand: Dezember 1994}

\subsection{Allgemeines}

\Paragraph{Aufgaben des Rechenzentrums}

Das Rechenzentrum ist eine zentrale Einrichtung der Technischen
Universit"at Braunschweig, der alle Datenverarbeitungsanlagen und
Datenkommunikationsnetze innerhalb der Universit"at zugeordnet sind.
Dem Rechenzentrum obliegen folgende Aufgaben:

\begin{enumerate}
 \item Der Betrieb der Datenverarbeitungsanlagen und des
       Datenkommunikationsnetzes zur Erf"ullung von Aufgaben der
       Universit"at in Forschung, Lehre und Studium sowie zur
       Erledigung von Verwaltungsaufgaben.
 \item die Beratung und Unterst"utzung f"ur die Nutzung der
       Datenverarbeitungsanlagen, des Datenkommunikationsnetzes und der
       Rechnerprogramme,
 \item die Betreuung aller der Hochschule verf"ugbaren
       Datenverarbeitungskapazit"aten und Datenkommunikationsnetze
       sowie die betriebsfachliche Aufsicht "uber alle
       Datenverarbeitungsanlagen der Hochschule,
 \item die Koordination der Beschaffung und Erg"anzung von
       Datenverarbeitungsanlagen, Datenkommunikationsnetzen und
       Rechnerprogrammen.
\end{enumerate}

\Paragraph{Leistungen und Inanspruchnahme des Rechenzentrums}

\renewcommand{\labelenumi}{(\arabic{enumi})}
\renewcommand{\labelenumii}{\arabic{enumii}.}
\begin{enumerate}
  \item Das Leistungsangebot des Rechenzentrums umfa"st insbesondere:
    \begin{enumerate}
      \item Die Bereitstellung von DV--Ger"aten,
      \item die Bereitstellung einer allgemeinen Datenkommunikation,
        insbesondere "uber das Hochschulnetz,
      \item Zugang zu Informationsdiensten,
      \item Dienstleistungen im Zusammenhang mit der DV--Versorgung.
    \end{enumerate}

    Das Leistungsangebot des Rechenzentrums im einzelnen wird in
    Online--Do\-ku\-mentationen, Handb"uchern und Mitteilungen des
    Rechenzentrums gesondert bekanntgegeben.

  \item Die Inanspruchnahme des Rechenzentrums bedarf der Zulassung und erfolgt
    nach Ma"sgabe dieser Benutzungsordnung und ihrer Anlagen:
\end{enumerate}

\begin{itemize}
 \item Richtlinien f"ur das Hochschulnetz (Anlage 1 [Anhang
   \ref{TUBSnet-Richtlinien}]).
 \item Richtlinien zur Kontingentierung der DV--Kapazit"aten der
       zentralen Anlagen (Anlage 2).
 \item Richtlinien zum Betrieb von www--Servern und zur Nutzung von
       www--Diensten (Anlage 3 [Anhang \ref{www-Richtlinien}]).
\end{itemize}

\Paragraph{Nutzungsberechtigte}

Folgende Personen bzw. Institutionen k"onnen auf Antrag die Leistungen
des Rechenzentrums in Anspruch nehmen:

\begin{enumerate}
 \renewcommand{\labelenumi}{\arabic{enumi}.}
 \item \label{mitgl} Die Mitglieder und Angeh"origen der Universit"at, alle
       Fachbereiche, wissenschaftliche Einrichtungen, Betriebseinheiten
       usw., die im Ausstattungsplan, im Organisationsplan oder im
       Haushaltsplan einer nieders"achsischen Hochschule oder einer
       Hochschule au"serhalb des Landes gef"uhrt sind.
 \item \label{wiss} Andere wissenschaftliche Einrichtungen, die ganz oder
       "uberwiegend aus "offentlichen Mitteln finanziert
       werden.
 \item Sonstige Einrichtungen und Personen, die nicht unter
       Ziffer \ref{mitgl} oder Ziffer \ref{wiss} fallen.
\end{enumerate}

\Paragraph{DV--Benutzer und DV--Beauftragte}

Benutzerinnen bzw. Benutzer sind diejenigen Personen, die die Leistungen
des Rechenzentrums unmittelbar in Anspruch nehmen. Die von den
Einrichtungen der Hochschule mit der Abwicklung ihrer
Datenverarbeitungsvorhaben beauftragten Mitarbeiter/-innen hei"sen
DV--Beauftragte.


\subsection{Benutzungserlaubnis}

\Paragraph{Zulassungsverfahren}

\begin{enumerate}
  \item Wer die Leistungen des Rechenzentrums in Anspruch nehmen will, bedarf
    der Zulassung (Benutzungserlaubnis). Mit der Zulassung wird das
    Benutzungsverh"altnis begr"undet. Die Benutzer haben die
    Benutzungsordnung und deren Anlagen zu beachten.

  \item Die Benutzungserlaubnis ist --- einschlie"slich der erforderlichen
    Benutzeridentifikation --- schriftlich zu beantragen. Der Antrag soll auf
    den Vordrucken des Rechenzentrums gestellt werden und hat folgende
    Angaben zu enthalten:

    \begin{enumerate}
      \item Angaben zur beantragenden Person oder Einrichtung,
      \item Angaben, die eine Zuordnung zu Rangstufen der Bearbeitung
        gem"a"s \S{} 8 erm"oglichen,
      \item Angaben zu Art, Umfang und Zweck der beabsichtigten Nutzung,
      \item Abgabe der verlangten Erkl"arungen.
    \end{enumerate}

  \item Nutzungsberechtigte, die die Genehmigung erhalten, eigene Rechner
    bzw. Subnetze am Hochschulnetz anzuschlie"sen und zu betreiben,
    k"onnen Mitgliedern und Angeh"origen der TU Braunschweig unter
    Beachtung der "`Richtlinien f"ur das Hochschulnetz"' [Anhang
    \ref{TUBSnet-Richtlinien}] Zugang zum Hochschulnetz gew"ahren.
\end{enumerate}

\Paragraph{Erlaubniserteilung}

\begin{enumerate}
  \item Die Benutzungserlaubnis, mit der auch die Benutzeridentifikation
    vergeben wird, wird vom Leiter des Rechenzentrums schriftlich erteilt. Sie
    ist auf die beantragte und bewilligte Nutzungsart beschr"ankt. Mit der
    Erlaubniserteilung erfolgt die Einstufung in die jeweilige Rangstufe (\S{} 8)
    und Kostengruppe. Die Erlaubnis kann zeitlich befristet, eingeschr"ankt
    und unter Auflagen und Bedingungen erteilt werden. Die Benutzungserlaubnis
    ist nicht "ubertragbar.

  \item Die Berechtigung zur Nutzung bestimmter Leistungen kann vom
    Rechenzentrum insbesondere mit Bezug auf folgende Gesichtspunkte
    eingeschr"ankt oder versagt werden:

    \begin{itemize}
      \item Rangstufenfolge gem"a"s \S{} 8 (Nutzungspriorit"at),
      \item Vorrang von Arbeiten in Lehre und Forschung der
        Universit"atseinrichtungen,
      \item Zweckbestimmung der betreffenden Ger"ate bzw. Rechnersysteme,
      \item Lizenzbestimmungen,
      \item Wirtschaftlichkeit bzw. Verh"altnism"a"sigkeit der
        Verfahren,
      \item Leistungsverm"ogen und Auslastung der betreffenden
        Ger"ate bzw. Rechnersysteme.
    \end{itemize}

 \item Der/die Nutzungsberechtigte hat wesentliche "Anderungen seiner im
   Antrag gemachten Angaben unverz"uglich dem Rechenzentrum mitzuteilen;
   insbesondere ist er/sie verpflichtet, die Beendigung der Nutzung
   unverz"uglich bekanntzugeben.
\end{enumerate}

\Paragraph{Beendigung des Benutzungsverh"altnisses}

\begin{enumerate}
  \item Die Benutzungserlaubnis erlischt mit Beendigung des
    Benutzungsverh"altnisses:
    \begin{itemize}
      \item Nach Ablauf der erteilten Frist.
      \item Aufgrund einer entsprechenden Mitteilung des Nutzungsberechtigten
        oder der/des DV--Beauftragten.
      \item Sobald der Nutzungsberechtigte aus der TU Braunschweig bzw.\  aus
        derjenigen Einrichtung ausscheidet, die die Benutzungserlaubnis
        beantragt hat.
      \item Durch Ausschluss gem"a"s \S{} 12 (2).
    \end{itemize}

  \item Der Nutzungsberechtigte verpflichtet sich, bei Beendigung des
    Benutzungsverh"altnisses:
    \begin{itemize}
      \item alle ihn betreffenden bzw.\  von ihm genutzten Datenbereiche und
        Adressen freizugeben,
      \item die vom Rechenzentrum zur Verf"ugung gestellten Arbeitsmittel
        zur"uckzugeben,
      \item alle sonstigen Anspr"uche des Rechenzentrums, die aus dem
        Benutzungsverh"altnis entstanden sind, zu erf"ullen.
    \end{itemize}
\end{enumerate}

\Paragraph{Rangstufen}

Die Datenverarbeitungsvorhaben werden nach der Zugeh"origkeit der
sie durchf"uhrenden Nutzungsberechtigten (siehe \S{} 3) in Gruppen
gegliedert, denen Rangstufen zugeordnet sind. Die Rangstufenzuordnung
legt die jeweilige Nutzungspriorit"at fest und richtet sich nach
den vorl"aufigen Grunds"atzen f"ur die Errichtung und den
Betrieb von Hochschulrechenzentren in Niedersachsen --- bekanntgegeben
mit RdErl.\  d. MWK vom 19.9.1978 -- 1053 -- 02 804 -- G"ultL MWK 60/55 ---.

\subsection{Benutzungsregeln}

\Paragraph{Rechte und Pflichten der Benutzer}

\begin{enumerate}
  \item Benutzerinnen und Benutzer sind verpflichtet, diese Benutzungsregeln
    einzuhalten. F"ur die Nutzung des Hochschulnetzes gelten
    zus"atzlich die speziellen, in den "`Richtlinien f"ur das
    Hochschulnetz"' (Anlage 1 [Anhang \ref{TUBSnet-Richtlinien}])
    aufgef"uhrten Regelungen.

  \item Die Nutzungsberechtigten k"onnen diejenigen Leistungen des
    Rechenzentrums in Anspruch nehmen, f"ur die sie eine
    Benutzungserlaubnis haben. Die bereitgestellten Ressourcen, die durch
    Kontingente (gem"a"s \S{} 14) begrenzt sein k"onnen, sind in
    wirtschaftlicher und dem Nutzungszweck angemessener Weise zu nutzen.
    Im "ubrigen haben die Benutzer darauf zu achten, da"s sie
    die Nutzungsm"oglichkeiten anderer nicht in unangemessener Weise
    beeintr"achtigen.

  \item Die Nutzung der DV--Einrichtungen f"ur kommerzielle oder private
    Zwecke ist nur mit schriftlicher Zustimmung des Rechenzentrums und
    nach Festlegung der Entgelte zul"assig.

  \item Der Nutzungsberechtigte hat daf"ur Sorge zu tragen, da"s die
    ihm zugeteilten Benutzeridentifikationen nicht an andere weitergegeben
    oder in sonstiger Weise mi"sbr"auchlich verwendet werden.

  \item Bei der Nutzung von R"aumen bzw. Ger"aten des Rechenzentrums
    sind die Bedienungsanleitungen, allgemeinen Sicherheitsvorschriften
    und die Vorschriften der Hausordnung zu beachten. Beim Umgang mit
    Einrichtungen und Ger"aten des Rechenzentrums ist die gebotene
    Sorgfalt aufzuwenden.

  \item Benutzerinnen und Benutzer d"urfen Software und Dokumentationen,
    die ihnen vom Rechenzentrum direkt oder indirekt zur Verf"ugung
    gestellt werden, nicht ohne Genehmigung des Rechenzentrums kopieren,
    an Dritte weitergeben oder Dritten zug"anglich machen oder an
    anderen Prozessoren als denen verwenden, f"ur die das Rechenzentrum
    die Software bestimmt hat. Im "ubrigen sind die f"ur die
    zur Verf"ugung gestellte Software ma"sgeblichen
    Lizenzbestimmungen einzuhalten.

  \item Es ist untersagt, Manipulationen an der Betriebssystem--Software
    und an Benutzerverzeichnissen vorzunehmen oder Zugriff auf
    Benutzerbereiche auszuf"uhren, f"ur die keine Berechtigung
    vorliegt.

  \item Nach Aufforderung durch das Rechenzentrum ist der Nutzungsberechtigte
    verpflichtet, einen Bericht "uber die Benutzung der Rechenanlagen
    und die dabei gewonnenen Erfahrungen abzugeben.
\end{enumerate}

\Paragraph{Sicherheit des Datenmaterials}

\begin{enumerate}
  \item Das Rechenzentrum sorgt im allgemein "ublichen Rahmen f"ur
    die Sicherung der Daten, die die Benutzer-/innen auf elektronischen
    Datentr"agern des Rechenzentrums speichern.

  \item Das Rechenzentrum bewahrt Medien, die mit Daten von Benutzern
    beschrieben sind, w"ahrend einer festgelegten Frist auf. Die
    innerhalb dieser Frist nicht abgeholten Medien k"onnen vom
    Rechenzentrum vernichtet werden.
\end{enumerate}

\Paragraph{Verarbeitung schutzw"urdiger Daten}

Die Verarbeitung und "Ubertragung von Daten, die schutzbed"urftig
im Sinne der Datenschutzbestimmungen sind, ist nur nach vorheriger R"ucksprache
mit dem Rechenzentrum und nur unter Einhaltung der vorgeschriebenen
Sicherheitsma"snahmen gestattet.

\Paragraph{Ordnungsma"snahmen}

\begin{enumerate}
  \item Verst"o"st ein Nutzungsberechtigter gegen diese
    Benutzungsordnung und deren Anlagen, insbesondere gegen die sich aus
    \S{} 9 ergebenden Pflichten, kann der Leiter des Rechenzentrums die
    Benutzungserlaubnis vor"ubergehend einschr"anken bzw.\  in
    wiederholten oder schwerwiegenden F"allen die Benutzung f"ur
    einen bestimmten Zeitraum untersagen. Der betreffende Benutzer
    muss davon unter Angabe der Gr"unde in Kenntnis gesetzt werden.

  \item In besonders schwerwiegenden F"allen kann dem betreffenden Benutzer
    nach vorheriger Anh"orung und mit Zustimmung der "`Senatskommission
    f"ur die elektronische Datenverarbeitung"' die Benutzungserlaubnis
    entzogen werden. Bei einem schwerwiegenden Versto"s wird der
    Leiter des Rechenzentrums pr"ufen, ob strafrechtliche oder
    zivilrechtliche Schritte einzuleiten sind.

  \item Der Leiter des Rechenzentrums "ubt das Hausrecht aus und ist
    weisungsberechtigt.
\end{enumerate}

\Paragraph{Haftung}

\begin{enumerate}
  \item Die Benutzer haften f"ur die von ihnen schuldhaft verursachten
    Sch"aden sowie f"ur Verluste und Ver"anderungen der Daten
    des Rechenzentrums oder Dritter. Sie stellen das Rechenzentrum (bzw.\ 
    die Universit"at) vor Anspr"uchen Dritter frei, sofern etwaige
    Sch"aden auf Verst"o"se gegen diese Benutzungsordnung,
    insbesondere gegen Lizenzbestimmungen Dritter zur"uckzuf"uhren
    sind.

  \item Das Rechenzentrum haftet f"ur die von seinen Mitarbeiterinnen und
    Mitarbeitern in Aus"ubung ihrer Dienstpflichten vors"atzlich
    oder grob fahrl"assig verursachten Sch"aden. Eine Haftung des
    Rechenzentrums f"ur fehlerhafte Rechenergebnisse, f"ur die
    Zerst"orung von Daten und die Besch"adigung von Datentr"agern
    sowie f"ur Termin"uberschreitungen bei Rechenarbeiten ist ---
    soweit rechtlich zul"assig --- ausgeschlossen.
\end{enumerate}

\subsection{Bewirtschaftung von Betriebsmitteln}

\Paragraph{Kontingentierung}

\begin{enumerate}
  \item Da Rechenzeiten und Betriebsmittel in beschr"anktem Umfang zur
    Verf"ugung stehen, werden sie gegebenenfalls in Form von
    Kontingenten an die Benutzer verteilt. Betriebsmittel im Sinne dieser
    Ordnung sind insbesondere Speicherbereiche, Drucker, "Ubertragungswege
    auf Datenleitungen und Rechner--Arbeitspl"atze.

  \item Die Zuteilung der Kontingente erfolgt gem"a"s einem
    Verteilungsschl"ussel, der die Zugeh"origkeit der
    Nutzungsberechtigten (gem"a"s \S{} 8) sowie den Aufgabenbezug,
    insbesondere den Vorrang von Lehre und Forschung, ber"ucksichtigt.
    Die jeweiligen Kontingente setzen sich aus einem festen Sockelbetrag
    und einem bedarfsbezogenen Anteil zusammen. Die Kontingente werden in
    der Regel jeweils zu Beginn eines Quartals neu berechnet.

  \item Die Regelungen zur Kontingentierung im einzelnen werden durch
    Senatsbeschluss festgelegt.

  \item Die Verteilung der Rechenzeiten der zentralen Rechenanlagen geschieht
    im einzelnen nach Ma"sgabe der "`Richtlinien zur Kontingentierung der
    DV--Kapazit"at der zentralen Anlagen der TU Braunschweig"' (Anlage 2).
\end{enumerate}

\Paragraph{Entgelte und Kostenermittlung}

\begin{enumerate}
  \item F"ur die Inanspruchnahme der Leistungen des Rechenzentrums sind
    je nach Rangstufe folgende Entgelte zu entrichten:

    \begin{tabular}{lll}
      Rangstufen 1, 2&Angeh"orige von Hochschulen des Landes&unentgeltlich\\
      Rangstufe 3&Angeh"orige von Hochschulen&Selbstkosten Land\\
      &au"serhalb des Landes&\\
      Rangstufe 4&Hochschulbedienstete bei Nebent"atigkeit&Marktpreise.\\
      &und sonstige Nutzer&\\
    \end{tabular}

    Die Nutzung von Rangstufe 1 und 2 erfolgt grunds"atzlich unentgeltlich;
    Aufwendungen im Sinne von \S{} 61 Abs. 1 Satz 2 LHO (Landeshaushaltsordnung)
    sind zu erstatten. In den Rangstufen 3 und 4 werden Entgelte erhoben.

  \item F"ur Leistungen, die den im Rechenzentrum "ublichen Rahmen
    "uberschreiten, k"onnen zus"atzliche Entgelte erhoben werden.
    Diese legt das Rechenzentrum fest und teilt sie auf Anfrage mit.

  \item Grundlage f"ur die Bemessung der in Anspruch genommenen Leistungen
    sind die Betriebsunterlagen des Rechenzentrums. An Hand dieser Unterlagen
    ermittelt das Rechenzentrum die zu zahlenden Entgelte. Den
    Nutzungsberechtigten werden regelm"a"sig Nachweise "uber
    die entstandenen Kosten zugestellt und, falls nach der
    Landeshaushaltsordnung Kosten zu erheben sind, diese in Rechnung
    gestellt.

  \item Benutzungsentgelte sind auch dann zu entrichten, wenn Programme
    ergebnislos oder fehlerhaft durchgef"uhrt werden, es sei denn,
    der Fehler ist nachweislich und aufgrund grober Fahrl"assigkeit
    vom Rechenzentrum zu vertreten und das Benutzungsentgelt ist erheblich.
    Bei Inanspruchnahme sonstiger Leistungen des Rechenzentrums gilt
    Satz 1 entsprechend.

  \item Das N"ahere zu den Entgelten einschlie"slich der Sonderleistungen
    und der Kostenermittlung ist in der "`Entgeltordnung"' geregelt. Es gelten
    die jeweils vom Pr"asidenten genehmigten S"atze f"ur die
    Kategorien "`Selbstkosten Land"' und "`Marktpreise"'.
\end{enumerate}

\Paragraph{Inkrafttreten}

Diese Benutzungsordnung tritt am Tage nach ihrer hochschul"offentlichen
Bekanntmachung in Kraft. Gleichzeitig treten fr"uhere
Benutzungsordnungen au"ser Kraft.
