
\section[WWW-Richtlinien an der TU Braunschweig]{Richtlinien zum Betrieb von
  www--Servern und zur Nutzung von www--Diensten an der Technischen
  Universit"at Braunschweig}
\label{www-Richtlinien}

{\small Dieser Text basiert auf einem Entwurf des Arbeitskreises "`Internet"'
  der Hochschule. Die Senatskommission f"ur die elektronische Datenverarbeitung
  (SEDV) hat den Entwurf in dieser Fassung verabschiedet und wird ihn zur
  endg"ultigen Beschlussfassung dem Senat vorlegen.

  Stand: 01.\,2.\,99}

\subsection{Grunds"atze des Betriebs von www--Servern}

\begin{enumerate}
  \item Die Technische Universit"at Braunschweig betreibt einen
        www--Server. Er steht allen Organisationseinheiten, kooperierenden
        Institutionen, Mitgliedern und Angeh"origen der
        Universit"at zur Verf"ugung. Die Organisationseinheiten
        der Universit"at k"onnen eigene www--Server betreiben.

  \item Die Abteilung 52 koordiniert im Auftrage der Hochschulleitung und
        in Absprache mit der Presse- und "Offentlichkeitsarbeit
        das Informationsangebot des zentralen www--Servers. Das
        Rechenzentrum "ubernimmt als Betreiber des www--Servers die
        technische Realisierung und die Betreuung des Systems.
\end{enumerate}

\subsection{Inhalt und Gestaltung von www--Seiten}

\begin{enumerate}
  \item Der Inhalt von www--Seiten muss den Anforderungen der
        DFN--Benutzungsordnung [Anhang \ref{dfn}] und der DV--Nutzungsordnung
        der Hochschule [Anhang \ref{RZordnung}] gen"ugen.

  \item Der Inhalt von www--Seiten muss gesetzliche Bestimmungen einhalten,
        insbesondere das Informations- und Kommunikationsdienste--Gesetz.
        Zu beachten sind u.a.\  der Schutz von personenbezogenen Daten,
        Urheber- und Lizenzrechte, Pers"onlichkeitsrechte und
        Strafgesetze. Im Strafgesetzbuch ist u.a.\  geregelt, da"s
        die Propaganda f"ur verfassungswidrige Organisationen, die
        Verbreitung von rassistischem Gedankengut, Pornographie sowie
        Beleidigungen und Verleumdungen zu unterlassen sind.

  \item Daten "uber Zugriffe auf www--Seiten d"urfen nur gespeichert
        werden, um eine anonyme Zugriffsstatistik zu erstellen oder
        um eine "Uberpr"ufung der Zugriffsberechtigung seitens
        der Domain des zugreifenden Systems zu erm"oglichen.
        Daher ist eine Integration von Tools zur Protokollierung von
        Zugriffen in www--Seiten in der Regel unzul"assig und hat
        zu unterbleiben. --- Erhebung und Speicherung entsprechender Daten
        kann ausnahmsweise zul"assig sein, sofern im Sinne des
        Nieders"achsischen Datenschutzgesetzes
        die Einwilligung der Betroffenen gegeben ist und die
        Hochschulleitung zugestimmt hat.

  \item Die Hochschule ist bem"uht, ihr Informationsangebot so breit
        und attraktiv wie m"oglich zu gestalten und setzt auf die
        engagierte Mitarbeit aller ihrer Angeh"origen. Um diesem
        Anspruch gerecht zu werden, sind die Hochschule und ihre
        Organisationseinheiten berechtigt, ohne R"uckfrage bei
        ihren Mitgliedern und Angeh"origen die folgende Daten
        weltweit zur Verf"ugung zu stellen:
        \begin{itemize}
          \item die offizielle Bezeichnung der Einheit sowie
                \begin{itemize}
                  \item Adresse, Telefon-- und Fax--Nummer, E-Mail--Adresse
                  \item Gesch"aftsverteilungsplan
                  \item angebotene Lehrveranstaltungen (Titel, Nummer,
                        Ort und Zeit, Name des Dozenten/der Dozentin)
                  \item Listen von Gremien und deren Mitglieder
                  \item bereits in anderen Medien ver"offentlichte
                        Artikel und Beitr"age in ungek"urzter
                        oder in einer von der verantwortlichen
                        Organisationseinheit gek"urzten Fassung,
                        soweit kein Copyright verletzt wird
                \end{itemize}
          \item die folgenden Daten ihrer Mitglieder und Angeh"origen:
                \begin{itemize}
                  \item Name, Vorname, Geschlecht
                  \item Zugeh"origkeit zu Organisationseinheiten
                  \item dienstliche Telefon-- und Fax--Nummer, E-Mail--Adresse
                  \item (gestrichen: private Telefonnummer, soweit
                      die Person der \mbox{Residenzpflicht\hspace{-11cm}
                      \rule[1mm]{10.85cm}{0.4pt}} unterliegt\hspace{-1.85cm}
                    \rule[1mm]{1.7cm}{0.4pt})
                  \item dienstliche Funktionen und Aufgaben
                  \item angebotene Lehrveranstaltungen
                \end{itemize}
        \end{itemize}

  \item www--Datenbereiche d"urfen in der Regel nicht Dritten (Personen
        bzw. Organisationen) zur Nutzung "uberlassen werden. In
        Ausnahmef"allen kann mit Genehmigung der Hochschulleitung
        nicht gewinnorientierten "offentlichen Einrichtungen im
        Rahmen der Amtshilfe eine Mitnutzung des www--Servers gew"ahrt
        werden.

  \item www--Seiten d"urfen in der Regel nicht kommerziell genutzt
        werden. Erlaubt ist dagegen den Organisationseinheiten der
        Hochschule die Nennung von F"orderern und Sponsoren samt
        Firmen-- und Produkt--Logos auf je einer eigens daf"ur
        eingerichteten Seite. Derartige Seiten sind hochschulweit
        einheitlich zu gestalten und entsprechend zu kennzeichnen.
        Entsprechende Vereinbarungen mit F"orderern sind der
        Hochschulleitung bekanntzugeben.

  \item Die Gestaltung der www--Seiten sollte sich an den allgemein
        anerkannten Gestaltungsregeln orientieren. Zu diesen
        z"ahlen insbesondere die diesbez"uglichen Gestaltungsempfehlungen
        der interuniversit"aren Arbeitsgruppe. Es ist darauf zu
        achten, da"s das Erscheinungsbild der Universit"at
        im Internet dem in anderen Medien vergleichbar ist. F"ur
        Logos und Signets der Hochschule sind nur die offiziellen
        Versionen (\url{http://www.tu-bs.de/pressestelle/icons/index.html})
        zu verwenden. Abteilung 52 empfiehlt Gestaltungselemente,
        die in den www--Seiten der Hochschulverwaltung verwendet werden.
\end{enumerate}

\subsection{Verantwortlichkeiten des Informationsanbietenden}

\begin{enumerate}
  \item Die f"ur das jeweilige Informationsangebot verantwortliche
        Institution bzw. Person (Universit"atsverwaltung,
        Rechenzentrum, Fachbereich, Institut, \dots\  oder Einzelperson) ist
        f"ur den Inhalt der von ihr bereitgestellten www--Seite,
        ihre Pflege und die Herstellung von Verweisen auf andere
        www--Seiten verantwortlich. Die Verantwortlichkeit erstreckt
        sich auch auf die Einhaltung gesetzlicher Vorschriften.

  \item Die Verantwortung f"ur den Inhalt einer www--Seite
        umfa"st in eingeschr"ankter Weise auch
        Hypertext--Referenzen auf andere Dokumente. Letztere sind
        gelegentlich zu "uberpr"ufen, ob sie ihrerseits
        den gesetzlichen Anforderungen gen"ugen. Ist das erkennbar
        nicht der Fall, muss eine betreffende Referenz entfernt
        oder auf die rechtliche Fragw"urdigkeit des betreffenden
        Dokumentes hingewiesen werden.

  \item Erg"anzend zum www--Angebot der Universit"at bzw.\  ihrer
        Organisationseinheiten k"onnen Mitglieder und Angeh"orige
        der Universit"at, die "uber eine Nutzungsberechtigung
        des RZ oder einer anderen Einrichtung der Universit"at
        verf"ugen, im Rahmen der disponiblen Ressourcen
        pers"onliche www--Seiten
        anbieten, auf denen auch private Themen behandelt werden
        k"onnen. --- Der "Ubergang zu den pers"onlichen
        www--Seiten ist deutlich zu kennzeichnen.

  \item Auf jeder www--Seite ist die f"ur die Bereitstellung der
        Information verantwortliche Organisationseinheit
        einschlie"slich Bearbeiter bzw. Einzelperson sowie das
        Datum der Erstellung bzw. Modifikation zu nennen. Es soll ein
        Link auf eine E-Mail--Adresse zur Verf"ugung gestellt werden,
        "uber die weitere Ausk"unfte bzw. Informationen zur
        Seite eingeholt werden k"onnen. Bei hierarchisch
        nachgegliederten Seiten k"onnen diese Angaben entfallen,
        sofern ein eindeutiger Zusammenhang zwischen den Seiten besteht.
\end{enumerate}

\subsection{Verst"o"se gegen die Regelung des www--Angebotes}

\begin{enumerate}
  \item www--Seiten, deren Inhalte offensichtlich gegen diese Ordnung,
        gegen vorrangige Ordnungen und Regeln oder gegen geltendes
        Recht versto"sen, sind vom Betreiber des jeweiligen
        www--Servers unverz"uglich zu l"oschen.

  \item www--Seiten, aus denen nicht unmittelbar zu entnehmen ist,
        wer f"ur sie verantwortlich ist, k"onnen gel"oscht
        werden.

  \item Ist fraglich, ob der Inhalt einer www--Seite im Sinne des
        ersten Absatzes zu beanstanden ist, informiert der Betreiber
        des www--Servers die jeweilige Anbieterin bzw.\  den jeweiligen
        Anbieter "uber die Beanstandung und bittet um Abhilfe.
        Kommt sie/er diesem Wunsch nicht nach und kann sie/er auch
        nicht nachvollziehbar begr"unden, wieso die beanstandete
        Seite unverzichtbar ist (z. B. f"ur Zwecke von Forschung
        und Lehre), kann die Seite vom Server--Betreiber gesperrt oder
        gel"oscht werden. --- In Zweifelsf"allen hat die
        Hochschulleitung gem"a"s der Rechtslage zu
        entscheiden.

  \item Der Server--Betreiber ist nicht verpflichtet, eine
        Routinedurchsicht der www--Seiten auf seinem Server
        durchzuf"uhren. Erst bei positiver Kenntnis eines
        Versto"ses gegen diese Ordnung wird der
        Server--Betreiber t"atig und auf eine Abstellung
        hinwirken.
\end{enumerate}
