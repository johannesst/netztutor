\section{Glossar}
\begin{description}
  \item[Account] die Zugangsberechtigung zu einem Computer oder
    Computersystem. Zum Account geh"oren Nutzerkennzeichen - auch login genannt
    - und Pa"swort.
  \item[Administrator] kurz Admin: Jemand, der sich um die Technik k"ummert,
    Soft- und Hardware einrichtet, Fehler behebt
  % Seit Jahren veraltet/nicht mehr vorhanden!
%  \item[Archie] weltweites Datenbanksystem mit Dateiverzeichnissen von "uber
 %   1000 ftp--Servern
  \item[Authentifizierung] Vorgang des Nachweises einer bestimmten
    Identit"at. Mit Hilfe eines Geheimnisses (z.B. Pa"swort) oder typischen
    Merkmales (z.B. Stimmabdruck) "uberzeugt man ein System von der
    vorgegebenen Identit"at
  \item[Client] Programm oder Rechner, welcher den Dienst eines Servers in
    Anspruch nimmt
  \item[DNS] \textbf{D}omain \textbf{N}ame \textbf{S}ervice, Umsetzung von
    IP--Adressen auf Host- und Domainnamen
  \item[E-Mail] Versenden, Empfangen, Lagern und Katalogisieren von
    Nachrichten "uber Netze mittels \textbf{S}imple \textbf{M}ail
    \textbf{T}ransfer \textbf{P}rotocol unter TCP/IP 
  \item[Firewall] Ein Rechner der zwischen zwei Netzen steht und zwischen
    diesen die hin- und herlaufenden Daten filtert. Zweck ist, Hackern das
    Einbrechen in die Rechner "`auf der anderen Seite"' nicht zu einfach zu
    machen, vergleichbar mit einer Feuerschutzwand - daher auch der Name
  \item[FTP] Kopieren von (gro"sen) Dateien zwischen Rechnern mittels 
      \textbf{F}ile \textbf{T}ransfer \textbf{P}rotocol auf TCP/IP,
      von SFTP abgelöst.
  \item{SFTP} Kopieren von großen Dateien zwischen Rechnern mittels
    \textbf{S}SH \textbf{F}ile  \textbf{T}ransfer \textbf{P}rotocol
    auf TCP/IP
  \item[Homepage] Startseite eines Informationsangebots im World Wide Web
  \item[Homeverzeichnis] Verzeichnis, in dem der Nutzer seine eigenen Daten
    ablegt. Auf UNIX mit \verb#~loginname# erreichbar
  \item[IRC] \textbf{I}nternet \textbf{R}elay \textbf{C}hat, weltweites
    Mehrbenutzer-Kommunikationssystem, Server verwalten viele
    Kommunikationskan"ale, Benutzer unterhalten sich auf den Kan"alen in
    Gruppen oder individuell
  \item[Linux] Eine frei verfügbare, weit verbreitete Version von \glossar Unix
  \item[MAC--Adresse] Abk"urzung f"ur \textbf{M}edia \textbf{A}ccess
    \textbf{C}ontrol address, eine weltweit eindeutige Kennzeichnung aller
    Ger"ate innerhalb eines Netzwerkes.
  \item[NFS] \textbf{N}etwork \textbf{F}ile \textbf{S}ystem, das von Sun
    eingef"uhrte Protokoll zur gemeinsamen Nutzung von Dateisystemen im
    Netzwerk.
  \item[Protokoll] Vorschrift f"ur die Kommunikationsabfolge mehrerer
    Teilnehmer. Dient zum Datenaustausch oder zur Steuerungs"ubergabe.
    Protokolle sind oft genormt.
  \item[Proxy] Service, der Objektdaten verschiedener Netzdienste wie HTTP,
    FTP, GOPHER oder News zwischenspeichert und entsprechende Anfragen aus
    seinem lokalen Datenbestand schnellstens bedient (caching) oder
    weiterreicht (proxying). Dadurch werden die Zugriffszeiten auf schon im
    Cache befindliche Daten enorm verbessert und die "Ubertragungswege aus dem
    Internet entlastet.
  \item[RfC] \textbf{R}equest \textbf{f}or \textbf{C}omment sind die
    Dokumente, in denen die Standards des Internet definiert sind.
  \item[Server] Programm oder Rechner, welcher eine bestimmte Serviceleistung
    (Dienst) zur Verf"ugung stellt. (Compute-, File-, WWW-, Mail-Server, etc.)
  \item[SMB] \textbf{S}erver \textbf{M}essage \textbf{B}lock, das insbesondere
    von Windows verwendete Protokoll, um Dateisysteme, Drucker etc. gemeinsam
    im Netzwerk zu nutzen sowie um Listen der verf"ugbaren Ressourcen zu
    erstellen und auszutauschen.
  \item[Sprechstunde] Jeden Montag von 19.00 Uhr bis 19.30 Uhr
   wird eine Sprechstunde im Clubhaus
    angeboten. Hier k"onnen Probleme beim Netzbetrieb besprochen oder
    der Netzantrag abgeholt und eingereicht werden.  
  \item[Subnetz] Eine Menge an Rechnern mit gleichem Netzteil der IP-Adresse
  \item[telnet] Internet-Protokoll f"ur einen Dialog auf einem anderen
    (UNIX-)Rechner, Einloggen auf irgendeinem "offentlichen UNIX-Host,
    mittlerweile durch ssh abgelöst
  \item[ssh] nternet-Protokoll f"ur einen Dialog auf einem anderen
    (UNIX-)Rechner, Einloggen auf irgendeinem "offentlichen UNIX-Host,
    anders als bei telnet  wird hierbei die Verbindung
    verschlüsselt. Basis für SFTP.
  \item[UNIX] sehr bew"ahrtes Betriebssystem, welches auf Gro"srechnern und
    Workstations eingesetzt wird. Entwickelt von AT\&T sowie gro"sen
    amerikanischen Universit"aten, 30 Jahre alt
  \item[Usenet News] weltweite "offentliche Diskussionen ("`schwarze Bretter"')
    von Teilnehmern zu beliebigen Themen mittels \textbf{N}etwork
    \textbf{N}ews \textbf{T}ransfer \textbf{P}rotocol unter TCP/IP, viele
    tausend Diskussionsgruppen (newsgroups) weltweit, entstanden 1979, lokale
    Administrationen f"ur spezifische Diskussionsgruppen mit regionaler
    Bedeutung
  \item[WWW] \textbf{W}orld \textbf{W}ide \textbf{W}eb ("`Surfen"' im
    Internet), das umfassende und f"uhrende Informationssystem, www ist
    multimedial, d.h.\  es integriert Schrift, Bild, Ton, bewegte Bilder,
    Video-Clip, Radio/TV Live, wenn entsprechende Multimedia-Tools installiert
    sind
\end{description}
