

\section{Geb"uhrenordnung des SchunterNet e.V.}
\label{gebuehr}

{\small Stand: 07. September 2009}

\begin{enumerate}
\item Eine Anschlussgeb"uhr wird nicht erhoben.

\item Die Nutzungsgebühr (Monatsgebühr) wird dynamisch erhoben. Sie wird
  abhängig vom Finanzbedarf des Vereins sowie der Anzahl der Nutzer
  angepasst und sollte einen Betrag von 13,-- \euro (in Worten: dreizehn
  Euro) nicht überschreiten. Derzeit ist die Gebühr auf 8,-- \euro (in Worten: acht Euro) festgelegt.
\item Zur Absicherung eventueller durch den Nutzer verursachter Forderungen Dritter an den Verein, ist eine Kaution in Höhe der aktuellen monatlichen Nutzungsgebühr zu hinterlegen

\item Alle Folgezahlungen sind bis 15. des Vormonats f"ur den
  entsprechenden Monat einzuzahlen (jeweils eine Monatsgeb"uhr).

\item Die Zahlung der Nutzungsgebühr erfolgt in erster Linie per Lastschrifteinzugsverfahren. Bei Barzahlungen wird eine zusätzliche Gebühr von 2,-- \euro je Zahlung erhoben. Ausnahme stellt die erste Barzahlung dar, bei der keine Barzahlungsgebühr erhoben wird.
\item Schlägt die Abbuchung aus Gründen, die der Nutzer zu vertreten hat, fehl, ist er verpflichtet die anfallenden Bankgebühren sowie die nichteingezogenen Gebühren in Bar zu begleichen. Dabei entfällt die Barzahlungsgebühr in Höhe von 2,-- \euro.
\item  Die Nutzungsgebühren werden in erster Linie zur Deckung der laufenden Kosten verwendet. Eventuelle Überschüsse werden für Reparaturen und weitere Investitionen genutzt.

\item Wird ein Zweitanschluss beantragt, so sind die Kosten der
  Umrüstung in Höhe von 15,-- \euro durch den Antragsteller zu tragen.

\item Durch Benutzer verursachter zusätzlicher Arbeitsaufwand im Sinne
  §8, Absatz 4, der Benutzerordnung [Anhang \ref{nutzerordnung}] wird zu einem Stundensatz von
  12,00 \euro in Rechnung gestellt, wobei ein Mindestbetrag von 6,-- \euro
  erhoben wird. 


\item Sch"aden an vereinseigener Hardware, die durch unsachgem"a"sen Umgang
  fahrl"assig oder vors"atzlich entstanden sind, gehen in vollem Umfang zu
  Lasten des Verursachers.

\item "Anderungen dieser Geb"uhrenordnung sind in den halbj"ahrlichen
  Mitgliederversammlungen m"oglich.
\end{enumerate}
