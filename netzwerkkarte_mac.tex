
\subsection{Netzwerkkarte und MAC--Adresse}
In jedem Zimmer befindet sich eine Anschlussdose. Sie befindet sich
meistens in Nähe der Steckdosen und sieht wie folgt aus:

%\bf{FIX ME: Bild einer typischen Netzwerkdose einfügen :D}

\rm Zum Anschluss ist ein Netzwerkadapter (Ethernet--Karte) mit RJ45--Buch\-se und
ein Anschlusskabel (Twisted Pair, RJ45, \textbf{nicht} crossed over) der
ben"otigten L"ange erforderlich. Ein entsprechender Adapter ist in den
gängigen Computern heutzutage fest verbaut. Das dazugehörige Kabel
kann, wenn nicht schon vorhanden, in der Sprechstunde des Vereins zum
Selbstkostenpreis erworben werden.
%Die
%Bandbreite der Netzkarte bzw.\  der Netzanbindung kann 10~MBit/s (10BaseT) oder 100~MBit/s (100BaseT) betragen. Macs haben seit 1999 10/100BaseT Ethernet eingebaut, das gilt auch
%f"ur iMacs/iBooks/PowerBooks.

Jede Ethernet--Karte bzw.\ --Chip hat eine weltweit eindeutige Hardware--(oder
\glossar MAC--)Adresse, die fest in der Karte eingestellt ist. Bei "alteren
Ethernet--Karten ist sie auf einen der Chips aufgeklebt, bei neueren
erf"ahrt man sie entweder aus der mitgelieferten Dokumentation, oder
durch ein Service--Programm.%, welches auf der mitgelieferten Diskette
%zu finden ist. Dazu wird der Rechner im MS--DOS--Mode gestartet (echtes DOS ist
%wichtig, also keine MS--DOS--Eingabeaufforderung oder DOS--Box unter Windows!)
%und das Programm \verb#A:\setup# aufgerufen. Unter "`View
%Configuration"' steht die Hardwareadresse (6 Byte, hexadezimal, jeweils durch
%Doppelpunkt getrennt).
Die Adresse kann z.B. so aussehen: \texttt{00:12:6b:9f:20:cc}
Je nach Betriebssystem gibt es mehrere Möglichkeiten, die MAC-Adresse
herauszufinden.

% %\subsubsection*{Windows 95/98/NT}
% \textbf{FIXME: Kann das raus? Oder gibts da noch welche?}\\
% Benutzer von Windows 95/98 haben  eine weitere recht einfache
% M"oglichkeit. Das Programm winipcfg.exe im Windows--Verzeichnis (meist
% \verb#C:\WINDOWS#) zeigt Informationen "uber den Netzwerkanschluss
% an, unter anderem auch die Hardwareadresse. Wichtig ist hierbei, da"s im
% entsprechenden Auswahlfeld die Ethernet-Netzwerkkarte angew"ahlt ist (die
% PPP-Adresse 44:45:53:54:00:00 ist \textbf{nicht} die gesuchte)
% Unter Windows NT findet sich das dazu n"otige Programm im Startmen"u unter
% Programme/Verwaltung (Allgemein)/Windows NT-Diagnose. Hier steht die
% MAC-Adresse auf der Seite "`Netzwerk"' und da "uber den Button "`Transporte"'.

\subsubsection*{Windows XP}
Unter Windows XP wählt man den Punkt Startmenü. Anschließend wählt man
den Menüpunkt ,,Ausführen'' und tippt dadrin \texttt{cmd} ein. In die sich
nun öffnende Eingabeaufforderung gibt man den Befehl \texttt{ipconfig /all}
ein.

 %Illustrierende Bilder
%\textbf{FIXME: Bilder sagen mehr als tausend worte}

\subsubsection*{Windows Vista/ 7}

\textbf{FIXME: Bilder sagen mehr als tausend worte}


\subsubsection*{Mac OS}

% In einer Sprechstunde sah ich, wie eine Benutzerin die MAC über den
% Apple System Profiler herausfand, würde das gerne mal anhand des
% Minis nachvollziehen und fotografisch aufbereiten :)
Unter MacOS X kann die MAC-Adresse mit den ,,Apple System Profiler''
ermittelt werden: %screenshot mini
Unter älteren Systemen (Mac OS 9.22 und dessen Vorgänger) kann 
die MAC--Adresse ermittelt werden, indem man das
Kontrollfeld "`TCP/IP"' aufruft und dessen INFO-Button anw"ahlt. Die hier
angegebene "`Hardware Adresse"' ist die MAC--Adresse. Alternativ kann die
MAC--Adresse auch mit dem "`Apple System Profiler"' im Apple--Men"u ermittelt
werden. 

\textbf{FIXME: Bilder sagen mehr als tausend worte}

\subsubsection*{Unix/Linux}
Zunächst öffnet man ein Terminal. Dadrin gibt man dann den Befehl 
. Unter
\glossar UNIX/Linux öffnet man zunächst ein Terminal. Dort gibt man
dann den Befehl \texttt{ifconfig} ein. Unter Umständen muss der
absolute Pfad \texttt{ifconfig} benutzt werden. Die Ausgabe könnte so
aussehen: %insert screenshot mit englischer und deutscher Form!

\textbf{FIXME: Bilder sagen mehr als tausend worte}
