
%\subsection{IRC (Internet Relay Chat)}

%Auskommentiert->Enventuell durch Verweis auf jabber-server ersetzen
% \subsection{ICQ (I Seek You)}

% \subsubsection*{Was ist ICQ?}

% ICQ ist ein kleines n"utzliches Internet Tool, welches dich dar"uber
% informiert, wer gerade online ist und es einem erm"oglicht, mit denjenigen in
% Kontakt zu treten. ICQ "ubernimmt f"ur dich die Internetsuche nach einem
% Bekannten und informiert dich, wenn der Betreffende gerade online ist. Man
% kann mit ICQ Chatten, Mitteilungen, Dateien und URL's schicken oder einfach
% nur mit Bekannten rumh"angen, w"ahrend man durchs WWW surft.

% \subsubsection*{Wie funktioniert es?}

% Wenn du ICQ installierst (nachdem du es von
% \url{http://www.icq.com/download/step-by-step.html} heruntergeladen geladen
% hast), fragt dich das Programm danach, dich auf einem Server zu registrieren,
% welcher mit dem weltumspannenden ICQ-Netzwerk verbunden ist. Zum Zeitpunkt der
% Registrierung erh"altst du eine eindeutige ICQ-Nummer (die UIN -- Universal
% Internet Number -- am besten irgendwo notieren!), weiterhin gibt dir ICQ die
% M"oglichkeit, pers"onlich Informationen unter deiner ICQ\# zu speichern. Dies
% erm"oglicht anderen ICQ-Nutzern, dich eindeutig zu erkennen, wenn du dich
% einloggst. 

% Wenn du dich einmal registriert hast, kannst du eine Liste "uber
% Freunde und Bekannte f"ur dich erstellen, welche ICQ dann dazu benutzt, diese
% Bekannten f"ur dich zu suchen. W"ahrenddessen l"auft ICQ im Hintergrund, ohne
% andere laufende Anwendungen zu unterbrechen. Sobald du dich im Internet
% einloggst, informiert ICQ alle Bekannten, da"s du anwesend bist, und
% informiert dich dar"uber, wenn Bekannte sich ein-- oder ausloggen. Sobald du
% wei"st, wer online ist, reicht ein Rechtsklick auf den Namen des Betreffenden
% um einen Chat zu starten, eine Mitteilung zu schicken, Dateien auszutauschen,
% oder andere peer-to-peer Anwendungen zu starten.

% Weitere Tips gibt es unter \url{http://www.icq.com/icqtour/} 

% \subsubsection*{Firewall/Router-Einstellungen}

% \begin{sloppypar}
% Auf den Button \fbox{ICQ/Preferences \& Security/Preferences} klicken. Danach
% das Feld \fbox{Connection} ausw"ahlen. Nun auf \fbox{I'm using a permanent
%   internet connection (LAN)}, weiterhin \fbox{I am behind a firewall or proxy}
% $\Rightarrow$ \fbox{Firewall Settings}. Nun \fbox{I don't use a SOCKS Proxy
%   server on my Firewall or I am using another Proxy server}. \fbox{Next}
% w"ahlen, nun \fbox{Use dynamically allocated port Numbers (Default)}
% $\Rightarrow$ \fbox{Next} $\Rightarrow$ \fbox{Check My Firewall / Proxy
%   Setting}. Wenn bei "`Status"' "`Success"' steht, l"auft alles. Herzlichen
% Gl"uckwunsch! Nur noch auf den Button unten rechts (das Bl"umchen!)
% klicken. Ist es gr"un, seid ihr online.
% \end{sloppypar}

% Falls nicht, wendet euch doch bitte an die Newsgroup \url{schunter.general}
% auf dem Server \url{news.schunter.etc.tu-bs.de}.

% \subauthor{EWU (ICQ\# 37759570)}

%\subsection{FTP}

%%Auf dem lokalen \glossar ftp-Server \url{ftp://ftp.schunter.etc.tu-bs.de/}
%kann frei nutzbare Software zur Verf"ugung gestellt werden.
