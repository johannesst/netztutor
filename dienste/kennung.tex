\subsection{Benutzerkennung und Passwort}
Bei der Anmeldung erhält man neben der Benutzerkennung auch ein
zufallbasiertes Passwort. Es wird benötigt, um auf die Serverdienste
des Netzes zuzugreifen. Allerdings sind unsere automatisch vergebenen
Passwörter nicht gerade einfach zu merken, daher folgt nun eine
Anleitung zum Ändern desselbigen. 
\subsubsection*{Wie "andere ich das Pa"swort?}

Um das Pa"swort zu "andern, muss zun"achst ein Terminalprogramm gestartet werden,
z.B. \glossar ssh oder \glossar putty. Als Zieladresse wird der Rechner jupiter.schunter.etc.tu-bs.de angegeben
(IP--Adresse ist 134.169.168.3, falls der \glossar DNS--Server nicht benutzt
werden kann). Man erh"alt dann folgende (oder "ahnliche) Begr"u"sungsmeldung: 

\begin{verbatim}
Login:
Password: 
\end{verbatim}
Es folgt die Eingabe der Benutzerkennung und des (alten) Pa"sworts. Bei der
Eingabe des Pa"sworts erfolgt kein Echo auf dem Bildschirm, auch nicht
symbolisch (z.B. durch *). Danach erscheint folgende (oder eine "ahnliche)
Meldung:
\begin{verbatim}
Linux jupiter.schunter.etc.tu-bs.de 2.6.32-jupiter #1 SMP Sat Dec 12 00:42:27 CET 2009 x86_64

Willkommen auf dem neuen Jupiter !

Wichtigste Neuerung : Die Mails werden fortan im Maildir-Format
abgelegt, siehe Maildir/ Ordner im Homeverzeichnis.

Last login: Thu Mar 20 15:30:06 on ttyp1 from erde.schunter.et
No mail.
loginname@jupiter:/home/users/loginname >
\end{verbatim}

Man befindet sich nun in einer \glossar UNIX--Shell--Umgebung. Zum "Andern des
Pa"sworts reicht es, das Kommando \verb#kpasswd# zu kennen. Nach dessen Aufruf
wird man zun"achst nach dem alten, und dann zweimal nach dem neuen Pa"swort
gefragt. Falls bei der Frage nach dem neuen Pa"swort zwei unterschiedliche
Zeichenketten angegeben wurden (Tippfehler), erscheint die Meldung

\verb#They don't match; try again.#

und man darf noch mal ein neues Pa"swort angeben (wiederum zweimal). Wichtig:
In der \glossar UNIX--Welt wird grunds"atzlich zwischen Gro"s- und
Kleinschreibung unterschieden, also auch beim Pa"swort. Wenn alles geklappt
hat, kann man die Umgebung mit dem Kommando \verb#exit# wieder verlassen. Auf
spezielle deutsche Sonderzeichen (Umlaute und Esszett) sollte beim Pa"swort
verzichtet werden. Auch wenn die \glossar UNIX--Anmeldung und das \glossar
E-Mail abholen noch klappt, so gibt es doch sp"atestens bei \glossar
SMB--Zugriffen (\glossar Homedirectory abbilden, Drucken "ubers Netz)
Schwierigkeiten.\\
Hinweise zur sicheren Wahl eines Passwortes findet man auf Seite
\pageref{passwort}.

%%% Local Variables: 
%%% mode: latex
%%% TeX-master: "../Netzeinfuehrung"
%%% End: 
