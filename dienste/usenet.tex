\subsection{Usenet News}

Zur Teilnahme am weltweiten Informationsaustausch "uber das \glossar Usenet
kann im \snev der \glossar Server\footnote{Tats"achlich l"auft im lokalen
  Netz gar kein Newsserver im eigentlichen Sinne (abgesehen von dem f"ur die
  lokalen Gruppen), sondern lediglich ein \glossar Proxy
  (\texttt{nntpcache}). Das bedeutet, dem \glossar Client wird ein
  vollwertiger Newsserver suggeriert, aber die Daten werden von verschiedenen
  \glossar Servern geladen und an den Client weitergereicht sowie lokal
  gecached. Vorteil dieser Methode ist, da"s weniger Festplattenplatz
  vorgehalten werden muss und dennoch eine gro"se Auswahl an Newsgruppen
  angeboten werden kann, wobei zweite und folgende Zugriffe auf dieselben
  Nachrichten beschleunigt erfolgen. Nachteilig sind fehlender Einfluss auf
  die Haltezeiten und langsamerer Erstzugriff auf die Informationen.}
\url{news.schunter.etc.tu-bs.de} genutzt werden (Standard--Port ist 119).
Dieser stellt neben die im DFN (\url{news.tu-bs.de}) aufliegenden
Newsgruppen% auch die Hierarchie % corel.*, intel.*, linux.*, microsoft.*,
%netscape.*, star\-office.* und 
%schunter.* 
zur Verf"ugung.

%Diese ist die lokale Hierarchie unseres Wohnheims mit den Gruppen 
%%\begin{tabbing}
%  schunter.general.testnetmm\=\kill
%  \url{schunter.general} \> Allgemeines im Wohnheim an der Schunter. \\
%  \url{schunter.net} \> Das Netzwerk. \\
%  \url{schunter.test} \> Tests nur hier!. 
%\end{tabbing}
%\newpage
Im Usenet herrschen andere Gepflogenheiten als in Chatrooms oder auf
WWW--Boards. Empfohlen wird, zun"achst 
\begin{tabbing}
  schunter.general.testnetmm\=\kill
  \url{de.newusers.infos} \> Infos und periodische Postings fuer neue User. (Moderated)
\end{tabbing}
aufzusuchen und sich die dort st"andig neu geposteten 15 Artikel f"ur
Einsteiger durchzulesen. Fragen dazu kann man in
\begin{tabbing}
  schunter.general.testnetmm\=\kill
  \url{de.newusers.questions} \> Neue Benutzer im Netz fragen, Experten antworten.
\end{tabbing}
stellen, dort wird einem von kompetenten Leuten geholfen. Diese Informationen
und die Einhaltung der dortigen Empfehlungen k"onnen eine Menge "Arger
ersparen und halten so zumindest den deutschsprachigen Teil des Usenets (de.*)
nutzbar.

Zwingend erforderlich ist die Verwendung einer g"ultigen E-Mail--Adresse nach
\glossar RfC 1036. Um sich vor unerw"unschter Werbemail (Bulk E-Mail, SPAM) zu
sch"utzen ist es eine sehr gute Idee, sich eine 
%\begin{enumerate}
 % \item[a)]
 eine kostenlose Extra--Mailadresse f"ur das Usenet einzurichten
    (z.B bei Googe Mail oder GMX) die man dann für nichts anderes
    nutzt.
% oder aber
 % \item[b)] sich eine kostenlose E-Mail--Adresse z.B bei \url{http://www.gmx.de} f"ur
  %  Mail \& News zu besorgen.
%\end{enumerate}
Auf gar keinen Fall darf mit einer ung"ultigen bzw. nicht erreichbaren
E-Mail--Adresse gearbeitet werden.

%Eine GMX-Adresse (mit automatischer Weiterleitung nach
%\url{schunter.etc.tu-bs.de}) hat die Vorteile, da"s
%\begin{enumerate}
%  \item Man nicht st"andig das "`Konto"' wechseln bzw. beim Mailen statt News
%    posten die Mail--Adresse nicht "andern muss.
%  \item sehr effektive automatische Filter bei GMX vor SPAM sch"utzen, und
%  \item beim Auszug aus dem Wohnheim bzw. K"undigung des Netzzugangs die Mails
%    an GMX auf den k"unftigen Server umgeleitet werden k"onnen oder aber bei
%    GMX verbleiben k"onnen, diese Adresse also stets g"ultig ist und nicht
%    st"andig umge"andert werden muss.
%\end{enumerate}

%%% Local Variables: 
%%% mode: latex
%%% TeX-master: "../Netzeinfuehrung"
%%% End: 
