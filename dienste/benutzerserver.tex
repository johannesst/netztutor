
\subsection{Benutzerserver (\texttt{jupiter.schunter.etc.tu-bs.de})}
\label{benutzerserver}
\subsubsection{Zugriff auf das Homedirectory}

Wie so oft gibt es auch hier mehrere M"oglichkeiten: 

\paragraph{\dots\ per ftp (File Transfer Protocol)} \ \\
Mit einem beliebigen \glossar ftp--Programm k"onnen Dateien in das
\glossar Home--Directory eingespielt werden. Auch die meisten anderen
Dateioperationen wie L"oschen, Verzeichnisse anlegen usw.\  k"onnen im
allgemeinen mit solchen Programmen durchgef"uhrt werden.

Beim Einlogproze"s m"ussen Benutzerkennung und Pa"swort angegeben werden, ein
Zugriff auf die \glossar Home--Directorys per anonymous--ftp ist nicht
m"oglich. Außerdem ist ftp nicht gerade das sicherste Verfahren, da
alle Passwörter in Klartext übertragen werden. Deswegen empfehlen wir
die Verwendung des deutlich sichereren scp/sftp-Protokolls.
% Aus diesem Grund ist der Netscape Navigator f"ur diese Zugriffsart
%nicht so gut geeignet.
\paragraph{\dots\ per sftp/scp} \ \\
Die Vorgehensweise ist hier ähnlich wie unter ftp, aber mit den
Vorteil, dass alle Verbindungen verschlüsselt werden. Unter
Unix-Systemen inklusive Mac OS X, sind in der Regel sftp und scp für
die Konsole schon vorhanden. Die gängigen Desktops (Gnome/Kde) haben
meistens schon einen graphischen Client direkt integriert, womit sich
der Zugang über \url{sftp:/login@jupiter.schunter.etc.tu-bs.de/}
herstellen lässt. Danach nur noch Passwort eintippen und man ist
verbunden. Je nach verwenden System kann die Url auch
\url{scp:/login@jupiter.schunter.etc.tu-bs.de/} oder
\url{ssh:/login@jupiter.schunter.etc.tu-bs.de/} lauten. 
Für Windows empfiehlt sich als Client das kostenlose Programm WinSCP
und für Mac OS X gibt es unter anderen Cyberduck. 

\paragraph{\dots\ durch Einloggen auf dem Server} \ \\ 
Mit ssh  kann man sich direkt auf dem \glossar Server (Name: "`jupiter.schunter.etc.tu-bs.de"')
einloggen und in einer \glossar UNIX--Shell--Umgebung (bash) arbeiten. Als
Texteditoren stehen z.B. vi (bzw.\  vim, elvis) und joe sowie emacs zur
Verf"ugung. Allerdings wollen wir aus Sicherheitsgründen diese
Funktionalität bald einschränken, dass neue Benutzer zunächst sich nur
via scp/sftp verbinden können. Auf Anfrage wird man allerdings einen
vollwertigen Zugriff erreichen können

%scheint derzeit nicht zu gehen, aber der samba server läuft noch starjoha
%\subsubsection{\dots\ per SMB--Mount} 

% F"ur Windows--Benutzer ist diese Variante vermutlich die komfortabelste
% Zugriffsm"oglichkeit. Dies funktioniert folgenderma"sen:

% \begin{sloppypar}
% Unter \fbox{Systemsteuerung} $\Rightarrow$ \fbox{Netzwerk} wird "uber
% \fbox{Hinzuf"ugen\dots} $\Rightarrow$ \fbox{Dienst} "`Datei und
% Druckerfreigabe f"ur Windows--Netzwerke"' installiert, falls dies nicht schon
% zuvor geschehen ist. Dann kann mit dem Windows--Explorer "uber \fbox{Extras}
% $\Rightarrow$ \fbox{Netzlaufwerk verbinden} mit der Angabe
% "`\url{\\jupiter\<username>}"' das \glossar Homedirectory auf einen neuen
% Laufwerksbuchstaben abgebildet werden. Es ist sinnvoll, die Option
% "`Verbindung beim Start wiederherstellen"' zu aktivieren. Das Pa"swort ist in
% diesem Fall das Server--Pa"swort. Der Datei- und Druckerfreigabedienst kann
% "ubrigens sp"ater wieder entfernt werden. Wenn er allerdings nie installiert
% wurde, klappt auch der \glossar SMB--Zugriff auf fremde Laufwerke nicht (eine
% Windows--Kuriosit"at).
% \end{sloppypar}

% Auch f"ur Linux--Benutzer besteht die M"oglichkeit, mit Hilfe von
% \path{smbmount} diese Zugriffsmethode zu benutzen. Ein Nachteil dabei ist, da"s
% normalerweise bei jedem Mount--Vorgang das Pa"swort eingegeben werden muss. Man
% kann sich zwar leicht ein Skript schreiben, das diese Aufgabe "ubernimmt, aber
% dann muss man auch das Pa"swort im Klartext in dem Skript unterbringen, was eine
% potentielle Sicherheitsl"ucke darstellt. 

% gibt net mehr starjoha
% \subsubsection{\dots\ per NFS--Mount}

% F"ur Linux und alle anderen \glossar UNIX--Varianten ist dies die richtige
% L"osung. Da sie im Wohnheim relativ selten genutzt wird, ist eine Freigabe des
% \glossar Home--Directorys per \glossar NFS nicht initial f"ur alle Benutzer
% eingerichtet. Wer hiervon Gebrauch machen m"ochte, sollte eine kurze Mitteilung
% an \url{admin@jupiter} mit Angabe des Loginnamens und der eigenen IP--Adresse
% schicken.

% Ein Tip: Wer sein Server--Home--Directory in den Home--Directory--Bereich eines
% normalen Benutzers auf der eigenen Maschine mounten m"ochte, sollte daf"ur
% sorgen, da"s die UID und GID auf beiden Rechnern "ubereinstimmen. Das vermeidet
% Probleme mit File--Permissions. 


\subsubsection{Private Homepages}

Jeder Benutzer kann in seinem HOME--Bereich auf dem Benutzerserver ein
Unterverzeichnis mit dem Namen \path{public_html} anlegen (auf  
\url{jupiter.schunter.etc.tu-bs.de} sollte dieses bereits existieren), das der
\glossar WWW-Server als Do\-ku\-men\-ten--Verzeichnis f"ur pers"onliche \glossar
WWW--Seiten interpretiert. Eine dort liegende Datei mit dem Namen
\path{index.html} wird als \glossar Homepage interpretiert (alternativ ist
auch \path{index.htm} m"oglich). Der URL dieser \glossar Homepage besteht aus
dem Namen des \glossar WWW-Servers, dem Tilde-Zeichen (\verb#~#) und dem Login
des Benutzers (Beispiel: \url{http://www.schunter.etc.tu-bs.de/~user/}). Es
sollte euch bewusst sein, da"s dieses Verzeichnis mittels \glossar WWW-Server
weltweit eingesehen werden kann.

Das Erzeugen eines Unterverzeichnisses \path{public_html} geht (unter Linux)
so:
\begin{itemize}
  \item Einloggen auf \url{jupiter.schunter.etc.tu-bs.de} mittels Kommando:
    \texttt{telnet jupiter} oder: \verb#ftp jupiter# und anschlie"sender
    \glossar Authentifizierung
  \item Kommando: \verb#mkdir ~/public_html# zur Erzeugung des
    Unterverzeichnisses, in dem die \glossar WWW--Seiten liegen sollen 
  \item Kommando: \verb#chmod 755 ~/public_html#, damit jedermann (!)
    dieses Verzeichnis lesen kann
  \item Kommando: \verb#chmod 711 ~#, damit jedermann ein vorhandenes
    Verzeichnis im HOME--Bereich finden kann
\end{itemize}

Das Verzeichnis kann aber auch vom PC unter Windows per \glossar ftp
oder sftp mit
graphischer Oberfl"ache angelegt werden, z.B. mit WS\underline{\ \
}FTP oder WinSCP. Auch
dazu m"usst ihr euch auf \url{jupiter} einloggen. Ein Verzeichnis k"onnt ihr dort
per Knopfdruck anlegen.

Legt im Verzeichnis \path{public_html} eine Datei mit dem Namen
\path{index.html} an (absoluter Pfad:
\path{~/public_html/index.html}). Diese Datei findet man mit dem URL:
\url{http://www.schunter.etc.tu-bs.de/~/user/}

Nat"urlich m"ussen auch alle Dateien in diesem Verzeichnis zum Lesen freigegeben
werden:
\begin{itemize}
  \item Kommando: \verb#chmod 644 ~/public_html/*#
\end{itemize}

Zus"atzlich kann im selben Verzeichnis (\path{~/public_html/}) eine Datei
\path{description.txt} mit einer kurzen Beschreibung des Inhalts der \glossar
Homepage (eine Zeile, maximal 50 Zeichen) angelegt werden.
% Diese wird dann auf
%der "Ubersichtsseite der privaten \glossar Homepages im SchunterNet mit angezeigt.

%%% Local Variables: 
%%% mode: latex
%%% TeX-master: "../Netzeinfuehrung"
%%% End: 
