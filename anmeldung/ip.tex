
\subsection{IP--Adressen und Namen}
\label{ip}

Damit ein Rechner im Internet identifiziert werden kann, ben"otigt er
eine weltweit eindeutige IP--Adresse, die euch von der \glossar
Netzadministration zugeteilt wird. F"ur das SchunterNet stehen Adressen in den
Bereichen 134.169.168.xxx und 134.169.169.xxx zur Verf"ugung. 

Da diese Zahlenkolonnen recht unhandlich sind, erh"alt jeder Rechner im
Wohnheim zus"atzlich einen eindeutigen Namen, welcher aus einem Host-- und einem
Domain--Teil besteht. Die Domain unseres Wohnheims ist {\tt
  schunter.etc.tu-bs.de}. Der Hostname kann frei gew"ahlt werden und darf aus
Kleinbuchstaben (keine Umlaute) und Zahlen bestehen, muss aber mit einem
Buchstaben beginnen. Nat"urlich darf er im Wohnheim auch nur
h"ochstens einmal vorkommen. (Bsp.: scotty logan5 \dots).
Falls es euren gew"ahlten Rechnernamen schon gibt, m"usst ihr
einen anderen Namen w"ahlen.

Aus Gr"unden der Handhabbarkeit wird eine Bezeichnung empfohlen, die
"ubersichtlich, reproduzier-- und merkbar ist. "Uberlange Zeichenfolgen, die
keinen Sinn ergeben, sind das zum Beispiel nicht.

Um einen Zugang auf den Benutzerserver (Linux--PC) zu erhalten (z.B.\  um dort
eine \glossar Homepage einzurichten oder die eingegangene \glossar E-Mail
abzuholen), ben"otigt ihr noch einen Loginnamen. Auch dieser kann relativ frei
gew"ahlt werden, darf jedoch nicht mehr als 8 Zeichen enthalten und sollte in
einem gewissen Zusammenhang mit dem zugeh"origen Benutzer stehen (Vorname,
Spitzname etc.) Bei der Wahl des Loginnamens sollte man beachten, da"s dieser
au"serdem den ersten Teil der E-Mail--Adresse darstellen wird, also
\url{loginname@schunter.etc.tu-bs.de}. Daher gilt hier die obige Empfehlung
f"ur die Zeichenfolge des Rechnernamens besonders.
