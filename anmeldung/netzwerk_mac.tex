
\subsection{Netzwerkkarte und MAC--Adresse}
\label{sec:netzw-und-mac}
In jedem Zimmer befindet sich eine Anschlussdose, üblicherweise bei
in der Nähe der Steckdosen und des Fernsehanschlusses.
\rm Zum Anschluss ist ein Netzwerkadapter (Ethernet--Karte) mit RJ45--Buch\-se und
ein Anschlusskabel (Twisted Pair, RJ45, \textbf{nicht} crossed over) der
ben"otigten L"ange erforderlich. Es  kann, falls nicht schon
vorhanden, in der \glossar Sprechstunder des SchunterNet erworben werden.
Die Bandbreite der Netzkarte bzw.\  der Netzanbindung kann 10~MBit/s
(10BaseT) oder 100~MBit/s (100BaseT) betragen. 
%Macs haben seit 1999 10/100BaseT Ethernet eingebaut, das gilt auch
%f"ur iMacs/iBooks/PowerBooks.

Jede Ethernet--Karte bzw.\ --Chip hat eine weltweit eindeutige Hardware--(oder
\glossar MAC--)Adresse, die fest in der Karte eingestellt ist. Bei "alteren
Ethernet--Karten ist sie auf einen der Chips aufgeklebt, bei neueren
erf"ahrt man sie entweder aus der mitgelieferten Dokumentation, oder
durch ein Service--Programm.%, welches auf der mitgelieferten Diskette
%zu finden ist. Dazu wird der Rechner im MS--DOS--Mode gestartet (echtes DOS ist
%wichtig, also keine MS--DOS--Eingabeaufforderung oder DOS--Box unter Windows!)
%und das Programm \verb#A:\setup# aufgerufen. Unter "`View
%Configuration"' steht die Hardwareadresse (6 Byte, hexadezimal, jeweils durch
%Doppelpunkt getrennt).
Die Adresse kann z.B. so aussehen: \texttt{00:12:6b:9f:20:cc}
Je nach Betriebssystem gibt es mehrere Möglichkeiten, die MAC-Adresse
herauszufinden.
\newpage
\subsubsection*{Windows XP}
Unter Windows XP wählt man \fbox{Startmenü} $\Rightarrow$
\fbox{Ausführen}.  In das sich öffnende Fenster tippt man  \texttt{cmd}
ein und bestätigt dies durch einen Klick auf \fbox{Ok}:
\centergraphics[width=\textwidth]{bilder/cmd_xp}
 In die sich
nun öffnende Eingabeaufforderung gibt man den Befehl \texttt{ipconfig /all}
ein: %Illustrierende Bilder
\centergraphics[width=\textwidth]{bilder/mac_xp}
Nun muss man aufpassen, dass man nicht die falsche MAC-Adresse, z.B
der WLAN-Karte erwischt. Gesucht wird die der Ethernetkarte. Im
Screenshot wäre dies zum Beispiel der \texttt{,,Ethernetadapter LAN-Verbindung
2''}. Die MAC-Adresse selbst verbirgt sich hinter der psysikalischen
Adresse, hier also: \texttt{,,0B-00-27-FC-09-51''}.%\newpage
%
\subsubsection*{Windows Vista/7}
 \begin{minipage}{0.9\linewidth}
Unter Windows Vista und Windows7 wählt man \fbox{Startmenü} und gibt in das Suchfeld 
 \texttt{cmd} ein. In den Ergebnissen wählt man dann die
 \texttt{cmd.exe} aus:
 %\ \\

\centergraphics[width=0.8\textwidth]{bilder/win7start}  %\end{figure}

%\begin{minipage}{1.0\linewidth}
%\end{minipage}
 In die sich nun öffnende Eingabeaufforderung gibt man den Befehl \texttt{ipconfig /all}
ein: %Illustrierende Bilder
 %\begin{minipage}{1.0\linewidth}
\centergraphics[width=0.8\linewidth]
{bilder/mac_7}
\end{minipage}\newpage%\ \\ \\
Nun muss man aufpassen, dass man nicht die falsche MAC-Adresse, z.B
der WLAN-Karte erwischt. Gesucht wird die der Ethernetkarte. Im
Screenshot wäre dies zum Beispiel der \texttt{,,Ethernetadapter Local Area Connection''}. 
Die MAC-Adresse selbst verbirgt sich hinter der psysikalischen
Adresse, hier %also: \texttt{,,0B-00-27-FC-09-51''}.
also: \texttt{,, 00:15:58:ce:53:14''}.
%\end{minipage}
%\hfill 
%\newpage
%In die sich nun öffnende Eingabeaufforderung gibt man den Befehl 
% Leute wegen Screenshots nerven
% ggf. in der Uni machen?
\subsubsection*{Mac OS X}
% dito
% In einer \glossar Sprechstunde sah ich, wie eine Benutzerin die MAC über den
% Apple System Profiler herausfand, würde das gerne mal anhand des
% Minis nachvollziehen und fotografisch aufbereiten :)
%\begin{minipage}{1.0\linewidth}
Unter MacOS X kann die MAC-Adresse wie folgt ermittelt werden:\\
Man öffnet zunächst im \fbox{Apfel-Menü} die \fbox{Systemeinstellungen} und dadrin
\fbox{Netzwerk}.  Hier sucht man nun nach der Netzwerkkarte,
üblicherweise \fbox{Ethernet (integriert)}  und klickt dadrin dann auf
\fbox{Weitere Optionen}. Dort wählt man dann den Reiter
\fbox{Ethernet}. Die dort angezeigte Ethernet-ID ist die gesuchte
MAC-Adresse.

%\end{minipage}
%mit den ,,Apple System Profiler''
%ermittelt werden: %screenshot mini
%Unter älteren Systemen (Mac OS 9.22 und dessen Vorgänger) kann 
%die MAC--Adresse ermittelt werden, indem man das
%Kontrollfeld "`TCP/IP"' aufruft und dessen INFO-Button anw"ahlt. Die hier
%angegebene "`Hardware Adresse"' ist die MAC--Adresse. Alternativ kann die
%MAC--Adresse auch mit dem "`Apple System Profiler"' im Apple--Men"u ermittelt
%werden. 
\newpage
\subsubsection*{Unix/Linux}
%\begin{minipage}{1.0\linewidth}
\small{Vorbemerkung: Je nach Distribution kann es kleinere Unterschiede
geben. Die folgende Anleitung wurde für Debian und Ubuntu getestet und
sollte sich leicht an andere Distributionen anpassen lassen.} Unter
\glossar UNIX/Linux öffnet man zunächst ein Terminal. Dort gibt man
dann den Befehl \texttt{ifconfig} ein. Unter Umständen muss der
absolute Pfad \texttt{ifconfig} benutzt werden. Die Ausgabe könnte so
aussehen: %insert screenshot mit englischer und deutscher Form!
\centergraphics{bilder/mac_linux}
Aufpassen muss man nun, dass man nicht die falsche MAC-Adresse
notiert, z.B. von der WLAN-Karte.
Gesucht ist die MAC-Adresse der Ethernetkarte. Diese fängt zumeist mit
,,eth'' an, hier also ,,eth1''.  Davon ist nun die MAC-Adresse
gesucht, auch ,,HWAddr'' oder ,,Hardwareadresse'' genannt, hier
also: \texttt{,, 00:15:58:ce:53:14''}. 
%\end{minipage}

%%% Local Variables: 
%%% mode: latex
%%% TeX-master: "../Netzeinfuehrung"
%%% End: 
