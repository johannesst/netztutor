\subsection{Nutzervertrag}

Um einen Zugang zum \emph{SchunterNet} zu erhalten, muss zun"achst ein
Antrag gestellt werden. Dieser dient zum einen der Registrierung der
pers"onlichen und technischen Daten der Nutzer bzw.\  deren Computer im
\snev und im Gauss--IT--Zentrum der TU zum Zwecke der \glossar Administration und
zum anderen als Nutzervertrag der Anerkennung der Teilnahmeregeln durch den
Nutzer. Gleichzeitig erfolgt die Aufnahme in den \snev als passives Mitglied.
% Schön wäre es... Anpassung an den Stand Umbau Haus3?
\bf Üblicherweise erhält man den  Antrag beim Einzug von der
Hausverwaltung oder bei der wöchentlichen \glossar Sprechstunde. Er ist vollst"andig ausgef"ullt und
unterschrieben in den Briefkasten des Vereins am Clubhaus einzuwerfen. Eine
PDF-Version ist auch unter
\url{http://www.schunternet.de/SchunterNet/Verein/Dokumente/} zu
finden.

\rm Es folgen einige Hinweise zum Ausf"ullen des Antrags und zu den Hintergr"unden.


%%% Local Variables: 
%%% mode: latex
%%% TeX-master: "../Netzeinfuehrung"
%%% End: 
